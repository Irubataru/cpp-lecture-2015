\usepackage{array}

% ------------------------------------------------ %
% tikz functions to place markers on beamer slides %
% ------------------------------------------------ %

\makeatletter
\tikzset{%
  remember picture with id/.style={%
    remember picture,
    overlay,
    save picture id=#1,
  },
  save picture id/.code={%
    \edef\pgf@temp{#1}%
    \immediate\write\pgfutil@auxout{%
      \noexpand\savepointas{\pgf@temp}{\pgfpictureid}}%
  },
  if picture id/.code args={#1#2#3}{%
    \@ifundefined{save@pt@#1}{%
      \pgfkeysalso{#3}%
    }{
      \pgfkeysalso{#2}%
    }
  }
}

\def\savepointas#1#2{%
  \expandafter\gdef\csname save@pt@#1\endcsname{#2}%
}

\def\tmk@labeldef#1,#2\@nil{%
  \def\tmk@label{#1}%
  \def\tmk@def{#2}%
}

\tikzdeclarecoordinatesystem{pic}{%
  \pgfutil@in@,{#1}%
  \ifpgfutil@in@%
    \tmk@labeldef#1\@nil
  \else
    \tmk@labeldef#1,(0pt,0pt)\@nil
  \fi
  \@ifundefined{save@pt@\tmk@label}{%
    \tikz@scan@one@point\pgfutil@firstofone\tmk@def
  }{%
  \pgfsys@getposition{\csname save@pt@\tmk@label\endcsname}\save@orig@pic%
  \pgfsys@getposition{\pgfpictureid}\save@this@pic%
  \pgf@process{\pgfpointorigin\save@this@pic}%
  \pgf@xa=\pgf@x
  \pgf@ya=\pgf@y
  \pgf@process{\pgfpointorigin\save@orig@pic}%
  \advance\pgf@x by -\pgf@xa
  \advance\pgf@y by -\pgf@ya
  }%
}
\newcommand\tikzsupermark[2][]{%
\tikz[remember picture with id=#2] #1;}
\makeatother

\newcommand{\MarkText}[3][]{%
\begin{tikzpicture}[overlay,remember picture]%
  \path (pic cs:#2,{(0,\paperheight)}) +(0,.7\baselineskip) coordinate (a);
  \path (pic cs:#3,{(0,-\paperheight)}) +(0,-.3\baselineskip) coordinate (b);
  \draw [ultra thick,
    if picture id={#2}{red}{line width=1cm,green,opacity=.5},
    #1]
  (a) -- ($(b)!(a)!($(b)+(1,0)$)$);
\end{tikzpicture}%
}%

\newcommand{\tikzmark}[1]{\tikz[overlay,remember picture] \coordinate (#1);}

% -------------- %
% Special frames %
% -------------- %

\newcommand{\LiveFrame}{%
\begin{frame}[plain]
  \nointerlineskip
  \begin{tikzpicture}[overlay,remember picture]
    \fill[FeebleWeek] (current page.north east) rectangle (current page.south west);
    \node at ([yshift=-.25em] current page.center) [font=\LARGE, WhiteTrash] {Live Example};
  \end{tikzpicture}
\end{frame}
}

\newcommand{\CPPEleven}{%
\nointerlineskip
\begin{tikzpicture}[overlay,remember picture]
  \node[anchor=south east] at ( current page.south east) [color=Tropiteal] {\{C++11\}};
\end{tikzpicture}
}

\newcommand{\CPPFourteen}{%
\nointerlineskip
\begin{tikzpicture}[overlay,remember picture]
  \node[anchor=south east] at ( current page.south east) [color=Tropiteal] {\{C++14\}};
\end{tikzpicture}
}

\newcommand{\BadPractice}{%
\nointerlineskip
\begin{tikzpicture}[overlay,remember picture]
  \node[anchor=south east] at ( current page.south east) [color=Marty] {Bad Practice};
\end{tikzpicture}
}


\newcommand{\Library}[2][0pt]{%
\nointerlineskip
\begin{tikzpicture}[overlay,remember picture]
  \node[anchor=south east] at ([xshift=#1] current page.south east) [color=Marty,scale=0.9]
    [font=\fontspec{Yanone Kaffeesatz Regular}] {\#include<#2>};
\end{tikzpicture}
}

% ----------------------------------------------------- %
% Various tikz based functions to style up the document %
% ----------------------------------------------------- %

\newcommand{\underlinemark}[3][line width=1.5pt, FeebleWeek]{%
\draw[transform canvas={yshift=-.15cm}, #1]
  ([xshift=-1pt] #2) -- ([xshift=2pt] #3);
}

\newcommand{\Highlight}[3][opacity=0.4,AtomicBikini]{%
\nointerlineskip
\begin{tikzpicture}[overlay,remember picture]
  \path (pic cs:#2,{(0,0)}) coordinate (a);
  \path (pic cs:#3,{(0,0)}) coordinate (b);
  \fill[#1] (a -| current page.north west) -- ++(0,1.4ex) -- ++(\paperwidth,0)
  -- (b -| current page.north east) -- ++(0,-.4ex) -- ++(-\paperwidth,0) -- cycle;
\end{tikzpicture}
}

\newcommand{\Dimtext}[2]%
{
  { \transparent{0.7}
  \begin{tikzpicture}[overlay, remember picture]
    \fill[white] ( #1 -| current page.north west) -- ++(0,.8em) -- ++(\paperwidth,0) -- (#2 -| current page.north east)
   -- ++(0,-.5em) -- ++(-\paperwidth,0) -- cycle;
  \end{tikzpicture}
  }
}

% -------------- %
% Array commands %
% -------------- %

\newcolumntype{L}[1]{>{\raggedright\let\newline\\\arraybackslash\hspace{0pt}}m{#1}}
\newcolumntype{C}[1]{>{\centering\let\newline\\\arraybackslash\hspace{0pt}}m{#1}}
\newcolumntype{R}[1]{>{\raggedleft\let\newline\\\arraybackslash\hspace{0pt}}m{#1}}
