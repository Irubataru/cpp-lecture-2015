\documentclass[14pt,a4paper,dvipsnames,usenames]{beamer}
\usepackage[T1]{fontenc}
\usepackage{url}
\usepackage{xcolor}
\usepackage{amsmath,varwidth}
\usepackage{tikz,tikz-uml,xparse}
\usepackage{listings,algpseudocode}
\usepackage[quiet]{mathspec}

\setmainfont[
  ItalicFont={Yanone Kaffeesatz Light},
  Scale=1.3,
  LetterSpace=2.0
]{Yanone Kaffeesatz Bold}

\setmonofont{Hack}

% --------------------- %
% Beamer theme settings %
% --------------------- %

\usetheme{CodeCourse}
\setbeamertemplate{navigation symbols}{} %Remove beamer navigation symbols
\setbeamertemplate{bibliography item}{\insertbiblabel}

% ---------------------------- %
% Colour definitions for theme %
% ---------------------------- %

\definecolor{Tropiteal}{RGB}{0,168,198}
\definecolor{TealDrop}{RGB}{64,192,203}
\definecolor{WhiteTrash}{RGB}{249,242,231}
\definecolor{AtomicBikini}{RGB}{174,226,57}
\definecolor{FeebleWeek}{RGB}{143,190,0}
\definecolor{ICantExpress}{RGB}{28,20,13}
\definecolor{Love}{RGB}{209,67,52}
\definecolor{UnrealFoodPills}{RGB}{250,105,0}
\definecolor{Marty}{RGB}{250,42,0}
\definecolor{ConradGold}{RGB}{255,221,0}

\definecolor{sbase03}{HTML}{002B36}
\definecolor{sbase02}{HTML}{073642}
\definecolor{sbase01}{HTML}{586E75}
\definecolor{sbase00}{HTML}{657B83}
\definecolor{sbase0}{HTML}{839496}
\definecolor{sbase1}{HTML}{93A1A1}
\definecolor{sbase2}{HTML}{EEE8D5}
\definecolor{sbase3}{HTML}{FDF6E3}
\definecolor{syellow}{HTML}{B58900}
\definecolor{sorange}{HTML}{CB4B16}
\definecolor{sred}{HTML}{DC322F}
\definecolor{smagenta}{HTML}{D33682}
\definecolor{sviolet}{HTML}{6C71C4}
\definecolor{sblue}{HTML}{268BD2}
\definecolor{scyan}{HTML}{2AA198}
\definecolor{sgreen}{HTML}{859900}

% -------------------------------------------------- %
% Beamer notes to be itemized rather than enumerated %
% -------------------------------------------------- %

\makeatletter
\def\beamer@setupnote{%
  \gdef\beamer@notesactions{%
    \beamer@outsideframenote{%
      \beamer@atbeginnote%
      \beamer@notes%
      \ifx\beamer@noteitems\@empty\else
      \begin{itemize}\itemsep=0pt\parskip=0pt%
        \beamer@noteitems%
      \end{itemize}%
      \fi%
      \beamer@atendnote%
    }%
    \gdef\beamer@notesactions{}%
  }
}
\makeatother

% -------------------------- %
% Standard LaTeX definitions %
% -------------------------- %

\DeclareGraphicsExtensions{.pdf}
\renewcommand\textbullet{\ensuremath{\bullet}}

% ----------------------------------- %
% Various tikz styles and definitions %
% ----------------------------------- %

\usetikzlibrary{calc,intersections,positioning,matrix,chains,scopes}
\usetikzlibrary{decorations.pathmorphing,decorations.pathreplacing,shapes}
\usetikzlibrary{decorations.text}
\usetikzlibrary{arrows.meta,bending}

\tikzset{
  table/.style={
    matrix of nodes,
    row sep=-\pgflinewidth,
    column sep=-\pgflinewidth,
    nodes={rectangle,draw,text width=6ex,align=center,scale=0.8},
    text depth=0.1ex,
    text height=1.25ex,
    nodes in empty cells
  },
  pointy arrow/.style={
    {Stealth[flex]}-,
    line width=1pt
  },
  simple uml node/.style={
    draw, rectangle,
    line width=1pt,
    rounded corners,
    fill=TealDrop,
    text width=3cm,
    align=center,
    font=\footnotesize
  },
  uml arrow/.style={
    -{Stealth},
    line width=1pt
  }
}
% ---------------------------------- %
% My default listings C++ code style %
% ---------------------------------- %

\lstset{%
  language=C++, basicstyle=\ttfamily\lst@ifdisplaystyle\footnotesize\fi,
  keywordstyle=\color{sgreen}, identifierstyle=\color{sblue}, 
  commentstyle=\color{sbase1}, stringstyle=\color{sorange},
  numberstyle=\color{sviolet}, showstringspaces=false,  
  breaklines=true,
  tabsize=5,
  morekeywords={%
    MyClass,unique_ptr,shared_ptr,weak_ptr,auto_ptr,
    make_unique, make_shared, default_delete,
    dynamic_pointer_cast, const_pointer_cast,
    scoped_ptr,QSharedPointer,QScopedPointer,QWeakPointer,
    ftring, SmartPtr, Shape, Circle, ShapeFactory, CircleFactory,
    Derived, Base
  }
}

\lstalias[]{gnumake}[gnu]{make}

\def\inline{\lstinline[basicstyle=\ttfamily\normalsize]}

% ---------------------------------------------------------------------------------------------------- %
% Various command for inline code instead of lstinline so that I don't have to update the keyword list %
% ---------------------------------------------------------------------------------------------------- %

\newcommand{\object}[1]{{\ttfamily \color{OliveGreen}#1}}
\newcommand{\function}[1]{{\ttfamily \color{RoyalBlue}#1}}
\newcommand{\std}[1]{{\ttfamily {\color{RoyalBlue} std::}{\color{OliveGreen}#1}}}
\newcommand{\boost}[1]{{\ttfamily {\color{RoyalBlue} boost::}{\color{OliveGreen}#1}}}

\newcommand{\uniqueptr}{\std{unique\_ptr}}
\newcommand{\sharedptr}{\std{shared\_ptr}}
\newcommand{\weakptr}{\std{weak\_ptr}}
\newcommand{\autoptr}{\std{auto\_ptr}}

\usepackage{array}

% ------------------------------------------------ %
% tikz functions to place markers on beamer slides %
% ------------------------------------------------ %

\makeatletter
\tikzset{%
  remember picture with id/.style={%
    remember picture,
    overlay,
    save picture id=#1,
  },
  save picture id/.code={%
    \edef\pgf@temp{#1}%
    \immediate\write\pgfutil@auxout{%
      \noexpand\savepointas{\pgf@temp}{\pgfpictureid}}%
  },
  if picture id/.code args={#1#2#3}{%
    \@ifundefined{save@pt@#1}{%
      \pgfkeysalso{#3}%
    }{
      \pgfkeysalso{#2}%
    }
  }
}

\def\savepointas#1#2{%
  \expandafter\gdef\csname save@pt@#1\endcsname{#2}%
}

\def\tmk@labeldef#1,#2\@nil{%
  \def\tmk@label{#1}%
  \def\tmk@def{#2}%
}

\tikzdeclarecoordinatesystem{pic}{%
  \pgfutil@in@,{#1}%
  \ifpgfutil@in@%
    \tmk@labeldef#1\@nil
  \else
    \tmk@labeldef#1,(0pt,0pt)\@nil
  \fi
  \@ifundefined{save@pt@\tmk@label}{%
    \tikz@scan@one@point\pgfutil@firstofone\tmk@def
  }{%
  \pgfsys@getposition{\csname save@pt@\tmk@label\endcsname}\save@orig@pic%
  \pgfsys@getposition{\pgfpictureid}\save@this@pic%
  \pgf@process{\pgfpointorigin\save@this@pic}%
  \pgf@xa=\pgf@x
  \pgf@ya=\pgf@y
  \pgf@process{\pgfpointorigin\save@orig@pic}%
  \advance\pgf@x by -\pgf@xa
  \advance\pgf@y by -\pgf@ya
  }%
}
\newcommand\tikzsupermark[2][]{%
\tikz[remember picture with id=#2] #1;}
\makeatother

\newcommand{\MarkText}[3][]{%
\begin{tikzpicture}[overlay,remember picture]%
  \path (pic cs:#2,{(0,\paperheight)}) +(0,.7\baselineskip) coordinate (a);
  \path (pic cs:#3,{(0,-\paperheight)}) +(0,-.3\baselineskip) coordinate (b);
  \draw [ultra thick,
    if picture id={#2}{red}{line width=1cm,green,opacity=.5},
    #1]
  (a) -- ($(b)!(a)!($(b)+(1,0)$)$);
\end{tikzpicture}%
}%

\newcommand{\tikzmark}[1]{\tikz[overlay,remember picture] \coordinate (#1);}

% -------------- %
% Special frames %
% -------------- %

\newcommand{\LiveFrame}{%
\begin{frame}[plain]
  \nointerlineskip
  \begin{tikzpicture}[overlay,remember picture]
    \fill[FeebleWeek] (current page.north east) rectangle (current page.south west);
    \node at ([yshift=-.25em] current page.center) [font=\LARGE, WhiteTrash] {Live Example};
  \end{tikzpicture}
\end{frame}
}

\newcommand{\CPPEleven}{%
\nointerlineskip
\begin{tikzpicture}[overlay,remember picture]
  \node[anchor=south east] at ( current page.south east) [color=Tropiteal] {\{C++11\}};
\end{tikzpicture}
}

\newcommand{\CPPFourteen}{%
\nointerlineskip
\begin{tikzpicture}[overlay,remember picture]
  \node[anchor=south east] at ( current page.south east) [color=Tropiteal] {\{C++14\}};
\end{tikzpicture}
}

\newcommand{\BadPractice}{%
\nointerlineskip
\begin{tikzpicture}[overlay,remember picture]
  \node[anchor=south east] at ( current page.south east) [color=Marty] {Bad Practice};
\end{tikzpicture}
}

\newcommand{\ClassLine}{%
\nointerlineskip
\begin{tikzpicture}[overlay,remember picture]
  \draw[transform canvas={yshift=-1.7cm},thin] (current page.north west) -- (current page.north east);
\end{tikzpicture}
}

\newcommand{\Library}[2][0pt]{%
\nointerlineskip
\begin{tikzpicture}[overlay,remember picture]
  \node[anchor=south east] at ([xshift=#1] current page.south east) [color=Marty,scale=0.9]
    [font=\fontspec{Yanone Kaffeesatz Regular}] {\#include<#2>};
\end{tikzpicture}
}

% ----------------------------------------- %
% Various Commands for inline codecolouring %
% ----------------------------------------- %

\newcommand{\inobj}[1]{%
  {\color{Tropiteal}#1}}

\newcommand{\inkey}[1]{%
  {\color{FeebleWeek}#1}}

\newcommand{\innorm}[1]{%
  {\color{ICantExpress}#1}}

\newcommand{\std}[1]{%
  \inobj{std}\innorm{::}\inkey{#1}}

% ----------------------------------------------------- %
% Various tikz based functions to style up the document %
% ----------------------------------------------------- %

\newcommand{\underlinemark}[3][line width=1.5pt, FeebleWeek]{%
\draw[transform canvas={yshift=-.15cm}, #1]
  ([xshift=-1pt] #2) -- ([xshift=2pt] #3);
}

\newcommand{\StrikeThrough}[2][Marty]{%
  \tikz[baseline=(strokethrough.base)]{%
    \node[inner sep=0pt, outer sep=0pt] (strokethrough) {\strut{}#2};
    \draw[line width=2pt,#1] ([xshift=-3pt] strokethrough.west) -- ([xshift=3pt] strokethrough.east);
  }
}

\newcommand{\Highlight}[3][opacity=0.4,AtomicBikini]{%
\nointerlineskip
\begin{tikzpicture}[overlay,remember picture]
  \path (pic cs:#2,{(0,0)}) coordinate (a);
  \path (pic cs:#3,{(0,0)}) coordinate (b);
  \fill[#1] (a -| current page.north west) -- ++(0,1.4ex) -- ++(\paperwidth,0)
  -- (b -| current page.north east) -- ++(0,-.4ex) -- ++(-\paperwidth,0) -- cycle;
\end{tikzpicture}
}

\newcommand{\Dimtext}[2]%
{
  { \transparent{0.7}
  \begin{tikzpicture}[overlay, remember picture]
    \fill[white] ( #1 -| current page.north west) -- ++(0,.8em) -- ++(\paperwidth,0) -- (#2 -| current page.north east)
   -- ++(0,-.5em) -- ++(-\paperwidth,0) -- cycle;
  \end{tikzpicture}
  }
}

% -------------- %
% Array commands %
% -------------- %

\newcolumntype{L}[1]{>{\raggedright\let\newline\\\arraybackslash\hspace{0pt}}m{#1}}
\newcolumntype{C}[1]{>{\centering\let\newline\\\arraybackslash\hspace{0pt}}m{#1}}
\newcolumntype{R}[1]{>{\raggedleft\let\newline\\\arraybackslash\hspace{0pt}}m{#1}}



\usefonttheme{serif}

%\setbeameroption{show notes}

\newcommand{\StrikeThrough}[2][Marty]{%
  \tikz[baseline=(strokethrough.base)]{%
    \node[inner sep=0pt, outer sep=0pt] (strokethrough) {\strut{}#2};
    \draw[line width=2pt,#1] ([xshift=-3pt] strokethrough.west) -- ([xshift=3pt] strokethrough.east);
  }
}

\title[C++ Day2]{Introduction to the C++\newline{}Programming Language\newline{}\newline{}\fontsize{16pt}{16pt}\selectfont{}Day 2}
\author{\texorpdfstring{%
    Aleksandra Rylund Glesaaen\newline\fontsize{12pt}{12pt}\selectfont\texttt{aleksandra@glesaaen.com}%
  }{%
    Aleksandra Rylund Glesaaen}}
\date{September 29th 2015}

\begin{document}

\begin{frame}
\titlepage
\end{frame}

\begin{frame}
  \frametitle{What will we learn?}

  \begin{itemize}
    \setlength\itemsep{.5em}
    \item \StrikeThrough{Basic C++ syntax}
    \item \StrikeThrough{Control structures}
    \item Functions \hspace{.5cm}{\color{Marty}(today)}
    \item Structs and classes \hspace{.25cm}{\color{Tropiteal}(Wednesday and Thursday)}
    \item Templates and STL \hspace{.5cm}{\color{Tropiteal}(Thursday and Friday)}
    \item Exceptions \hspace{.5cm}{\color{Tropiteal}(Friday)}
  \end{itemize}
\end{frame}

\begin{frame}[fragile]
  \frametitle{Today's topics}

  \tableofcontents
  
\end{frame}

\section{Scope}

\frame[plain]{\sectionpage}

\begin{frame}[fragile]
  \frametitle{Variable Visibility}

  A scope is an area of visibility of a defined name

  \vspace{.3cm}
  \begin{columns}
  \column{0.45\textwidth}
  \begin{tikzpicture}
    \node [
      draw, line width=1pt,
      text width=\textwidth, text depth=1.5cm,
      FeebleWeek
    ] (scope1) {%
      Scope \#1
    };
    \node[anchor=west] at ([yshift=-.25em]scope1.west) {var \;a, b};
    \node [
      below=.5cm of scope1,
      draw, line width=1pt,
      text width=\textwidth, text depth=1.5cm,
      Marty
    ] (scope2) {%
      Scope \#2
    };
    \node[anchor=west] at ([yshift=-.25em]scope2.west) {var \;b, c};
  \end{tikzpicture}
  \column{0.55\textwidth}
  
  \begin{itemize}
    \setlength\itemsep{1em}
    \item a is not defined in\\scope \#2
    \item c is not defined in\\scope \#1
    \item the name b is not the same memory address in the two scopes
  \end{itemize}
  
  \end{columns}
  
\end{frame}

\begin{frame}[fragile]
  \frametitle{Variable Visibility}

  Scopes can also be nested

  \vspace{.5cm}
  \begin{columns}
  \column{0.45\textwidth}
  \begin{tikzpicture}
    \node [
      draw, line width=1pt,
      text width=\textwidth, text depth=4cm,
      FeebleWeek
    ] (scope1) {%
      Scope \#1
    };
    \node[anchor=west] at ([yshift=-2em]scope1.north west) (scope1var) {var \;a, b};
    \node [
      below=.25em of scope1var.south west,
      xshift=.5em,
      draw, line width=1pt,
      text width=0.7\textwidth, text depth=1.5cm,
      anchor=north west,
      Marty
    ] (scope2) {%
      Scope \#2
    };
    \node[anchor=west] at ([yshift=-2em]scope2.north west) (scope2var) {var \;c, d};
  \end{tikzpicture}
  \column{0.55\textwidth}
  
  \begin{itemize}
    \setlength\itemsep{1em}
    \item a and b are both available in scope \#2
    \item c and d are not in\\scope \#1
  \end{itemize}
  
  \end{columns}

\end{frame}

\begin{frame}[fragile]
  \frametitle{Variable Visibility}

  One can also overshadow names in nested scopes

  \vspace{.5cm}
  \begin{columns}
  \column{0.45\textwidth}
  \begin{tikzpicture}
    \node [
      draw, line width=1pt,
      text width=\textwidth, text depth=4cm,
      FeebleWeek
    ] (scope1) {%
      Scope \#1
    };
    \node[anchor=west] at ([yshift=-2em]scope1.north west) (scope1var) {var \;a};
    \node [
      below=.25em of scope1var.south west,
      xshift=.5em,
      draw, line width=1pt,
      text width=0.7\textwidth, text depth=1.5cm,
      anchor=north west,
      Marty
    ] (scope2) {%
      Scope \#2
    };
    \node[anchor=west] at ([yshift=-2em]scope2.north west) (scope2var) {var \;a};
  \end{tikzpicture}
  \column{0.05\textwidth}
  \column{0.45\textwidth}

  The a is not the same in the two scopes

  \vspace{1em}
  a from Scope \#1 is unavailable inside of Scope \#2
  
  \end{columns}
\end{frame}

\begin{frame}[fragile]
  \frametitle{Scope and memory management}

  A variable only takes up memory while it is "in scope"

  \vspace{1em}
  It is deleted when it goes "out of scope"
  
\end{frame}

\begin{frame}[fragile]
  \frametitle{Blocks \;$\approx$\; Scope}

  Blocks define a new scope

  \vspace{.5em}
  \begin{lstlisting}[escapeinside={(*}{*)}]
{ (*\tikzmark{scope1 begin}*)
  bool dancing = true;
  for (;;) { (*\tikzmark{scope2 begin}*)
    // ...

  } (*\tikzmark{scope2 end}*)
} (*\tikzmark{scope1 end}*)

{ (*\tikzmark{scope3 begin}*)
  int baboons = 10;
  // ...
} (*\tikzmark{scope3 end}*)
  \end{lstlisting}

  \nointerlineskip
  \begin{tikzpicture}[overlay,remember picture]
    \coordinate (first layer) at ( current page.north);
    \coordinate (second layer) at ([xshift=.75cm] current page.north);

    \draw[decorate,decoration={brace,amplitude=5pt,aspect=0.15},line width=1pt]
      ($(scope1 begin -| first layer) + (0,1ex)$) -- (scope1 end -| first layer)
      node[pos=0.15,right=.5em,scale=0.75] {Scope 1};
    \draw[decorate,decoration={brace,amplitude=5pt,aspect=0.5},line width=1pt]
      ($(scope2 begin -| second layer) + (0,1ex)$) -- (scope2 end -| second layer)
      node[pos=0.5,right=.25em,scale=0.75] {Scope 2};
    \draw[decorate,decoration={brace,amplitude=5pt,aspect=0.5},line width=1pt]
      ($(scope3 begin -| first layer) + (0,1ex)$) -- (scope3 end -| first layer)
      node[pos=0.5,right=.5em,scale=0.75] {Scope 3};
  \end{tikzpicture}

\end{frame}

\section{Functions}

\frame[plain]{\sectionpage}

\begin{frame}
  \frametitle{DRY, KISS, YAGNI, Occam, ...}

  Functions is a tool to divide one big problem\\into many small ones

  \vspace{1em}
  DRY - Don't Repeat Yourself

  \vspace{1em}
  KISS - Keep It Simple, 
  \tikz[baseline={(stupid.base)}]{
    \node[inner sep=2pt] (stupid) {Stupid};
    \draw[line width=2pt,Marty] ([yshift=.2cm] stupid.south west) -- ([yshift=-.1cm] stupid.north east);
  } Silly

  \vspace{1em}
  If you copy-paste while programming,\\you are doing it {\color{Marty}WRONG}
  
\end{frame}

\begin{frame}
  \frametitle{What is a function?}

  A function is a unit in your program that

  \vspace{1em}
  \begin{itemize}
    \setlength\itemsep{1em}
    \item Takes input from the caller {\color{Tropiteal}\fontsize{8pt}{8pt}\selectfont[optional]}
    \item {\color{Marty}\underline{Does something}}
    \item Returns the result to the caller {\color{Tropiteal}\fontsize{8pt}{8pt}\selectfont[optional]}
  \end{itemize}
  
\end{frame}

\begin{frame}[fragile]
  \frametitle{The Function Signature}

  \begin{lstlisting}[escapeinside={(*}{*)},basicstyle=\ttfamily\fontsize{11pt}{11pt}\selectfont]
(*\tikzmark{type begin}*)double (*\tikzmark{name begin}*)integerPower((*\tikzmark{sig begin}*)double x, int n(*\tikzmark{sig end}*)) { (*\tikzmark{body}*)... }
  \end{lstlisting}

  \only<2>{
  \nointerlineskip
  \begin{tikzpicture}[overlay,remember picture]
    \draw[pointy arrow] ([shift={(2ex,-.25em)}] type begin) .. controls +(0,-1cm) and +(-.5cm,0) .. +(.5cm,-1cm)
      node[right, scale=0.8,align=left] {The return type\\of the function};
    \draw[pointy arrow] ([shift={(4ex,1.25ex)}] name begin) -- +(0,1cm)
      node[above,xshift=-1cm] [align=left,scale=0.8] {The name of the function\\Similar to a variable name};
    \draw[transform canvas={yshift=-.25em},decorate,decoration={brace,amplitude=5pt,mirror},line width=1pt]
      (sig begin) -- (sig end);
    \draw[transform canvas={yshift=-.5cm},pointy arrow] ($(sig begin) !.5! (sig end)$) -- +(0,-2cm)
      node[below,xshift=-1cm,scale=0.8] {The input parameters of the function};
    \draw[pointy arrow] ([shift={(1ex,1.25ex)}] body) -- +(0,1cm)
      node[above,scale=0.8,align=center] {The function body\\{\fontsize{10pt}{10pt}\selectfont(creates a scope)}};
  \end{tikzpicture}
  }
  
\end{frame}

\begin{frame}[fragile]
  \frametitle{Return types}

  Most types, including qualifiers, can be returned from a function
  {\fontsize{10pt}{10pt}\selectfont{}but their meanings might not be that straight forward}

  \vspace{1em}
  To return something from a function, use the {\color{FeebleWeek}return} keyword
  
  \begin{onlyenv}<2>
    \vspace{.5em}
    \begin{lstlisting}[basicstyle=\ttfamily]
      int highFive()
      {
        return 5;
      }
    \end{lstlisting}
  \end{onlyenv}

\end{frame}

\begin{frame}[fragile]
  \frametitle{Return types}

  \vspace{.5em}
  {\fontsize{18}{18}\selectfont{}\color{FeebleWeek}reference / pointer}

  \vspace{.5em}
  returning references and pointers can be very useful, just make sure the variables haven't gone out of scope

  \vspace{.5em}
  \begin{lstlisting}
int & highFive()
{
  int five = 5;
  return five;
}
  \end{lstlisting}

  \vspace{.5em}
  this will not work
  
\end{frame}

\begin{frame}[fragile]
  \frametitle{Return types}

  \vspace{.5em}
  {\fontsize{18}{18}\selectfont{}\color{FeebleWeek}void}

  \vspace{1em}
  void is the return type when you don't want to return anything

  \vspace{.5em}
  \begin{lstlisting}
void sayHello()
{
  std::cout << "Hello!" << std::endl;
}
  \end{lstlisting}


\end{frame}

\begin{frame}[fragile]
  \frametitle{Return types}

  \vspace{.5em}
  {\fontsize{18}{18}\selectfont{}\color{FeebleWeek}array}

  \vspace{1em}
  You can't return an array in the naive way, but you can return the underlying pointer, but there is the scope thing
  again

\end{frame}

\begin{frame}[fragile]
  \frametitle{Return types}

  \vspace{.5em}
  {\fontsize{18}{18}\selectfont{}\color{FeebleWeek}const}

  \vspace{.5em}
  returning const prevents things like rvalue assignments
  
  \vspace{.5em}
  \begin{lstlisting}[escapeinside={(*}{*)},morekeywords={MyType}]
MyType function(int,double);

int main()
{
  // ...
  function(4,5.) = x;
}
  \end{lstlisting}

\end{frame}

\begin{frame}[fragile]
  \frametitle{Call-by-value vs Call-by-reference}

  Call-by-value is when you give your argument as a {\color{Marty}non-reference} type

  \vspace{.5em}
  \begin{itemize}
    \setlength\itemsep{.5em}
    \item Can only get feedback from the return value
    \item Function body can't manipulate input parameters
    \item Input parameters are copied in memory
  \end{itemize}

\end{frame}

\begin{frame}[fragile]
  \frametitle{Call-by-value vs Call-by-reference}

  \begin{lstlisting}[basicstyle=\ttfamily]
double half(double input)
{
  return input/2;
}

int main()
{
  double var = 5;
  double half_of_var = half(var);
}
  \end{lstlisting}
  
\end{frame}

\begin{frame}[fragile]
  \frametitle{Call-by-value vs Call-by-reference}

  Call-by-reference is when you give your argument as a {\color{FeebleWeek}reference} type

  \vspace{.5em}
  \begin{itemize}
    \setlength\itemsep{.5em}
    \item The function can manipulate the arguments
    \item No unnecessary copying of variables
    \item A const reference is like a read-only argument\\
      {\fontsize{10pt}{10pt}\selectfont{}\{but remember that copying base types is basically free\}}
  \end{itemize}
  
\end{frame}

\begin{frame}[fragile]
  \frametitle{Call-by-value vs Call-by-reference}

  \begin{lstlisting}[basicstyle=\ttfamily]
void half(double & input)
{
  input /= 2;
}

int main()
{
  double var = 5;
  half(var);
}
  \end{lstlisting}
  
\end{frame}

\begin{frame}[fragile]
  \frametitle{Arrays as input}

  \begin{lstlisting}
double sum(const double arr[], int size) { ... }
  \end{lstlisting}

  \vspace{1em}
  Arrays are automatically called by reference

  \vspace{1em}
  Can leave the array dimension unspecified
  
\end{frame}

\begin{frame}[fragile]
  \frametitle{Multi-dimensional arrays}

  \begin{lstlisting}
double print(const int array[][5][10]) { ... }
  \end{lstlisting}

  \vspace{1em}
  Can {\color{Marty}only} leave the innermost size unspecified
  
\end{frame}

\begin{frame}[fragile]
  \frametitle{Recursive functions}

  Functions can of course call other functions,\\including themselves

  \vspace{1em}
  A function calling itself is called {\color{Tropiteal}recursive}

  \vspace{1em}
  Recursive functions is like an advanced loop
  
\end{frame}

\begin{frame}[fragile]
  \frametitle{Recursive functions - example}

  \only<2>{
  \Highlight{alt return begin}{alt return end}
  }

  \begin{lstlisting}[escapeinside={(*}{*)},basicstyle=\ttfamily]
unsigned factorial(unsigned n)
{
  if (n < 2) { (*\tikzsupermark{alt return begin}*)
    return 1;
  } (*\tikzsupermark{alt return end}*)

  return n * factorial(n-1);
}
  \end{lstlisting}

  \only<2>{
  \nointerlineskip
  \begin{tikzpicture}[overlay,remember picture]
    \path (pic cs:alt return end,{(0,0)}) coordinate (a);
    \node[anchor=south west, inner sep=0pt] at ([shift={(5cm,1.5ex)}] a) [scale=0.8] {Alternate return path};
  \end{tikzpicture}
  }
\end{frame}

\begin{frame}[fragile]
  \frametitle{Static variables}

  Static variables aren't deleted when they go out of scope, but when the program exits

  \vspace{.5em}
  Their initialisation only happens the first time the block is run

  \vspace{.5em}
  \begin{lstlisting}[escapeinside={(*}{*)}]
int countCalls()
{
  static int times_called = 0;
  ++times_called;
  return times_called;
}
  \end{lstlisting}
  
\end{frame}

\begin{frame}[fragile]
  \frametitle{Definition vs Declaration}

  Every function must be declared before it is called

  \vspace{1em}
  But that doesn't mean you have to stick all your functions at the top of your C++ file
  
\end{frame}

\begin{frame}[fragile]
  \frametitle{Definition vs Declaration}

  Function declaration:

  \vspace{.5em}
  \begin{lstlisting}[escapeinside={(*}{*)}]
void function(int, const double&(*\tikzmark{decl name}*));
  \end{lstlisting}

  {\fontsize{10pt}{10pt}\selectfont
    Declare that the function exist, and define its signature, but not what it does
  }

  \nointerlineskip
  \begin{tikzpicture}[overlay,remember picture]
    \draw[pointy arrow] ([yshift=1.25ex] decl name) .. controls +(0,.5cm) and +(-.3cm,0) .. +(.5cm,.5cm)
      node[right,scale=0.6] {Argument name optional};
  \end{tikzpicture}

  \vspace{1em}
  Function definition:

  \vspace{.5em}
  \begin{lstlisting}[escapeinside={(*}{*)}]
void function(int x, const double & d)
{
  // ...
}
  \end{lstlisting}

  {\fontsize{10pt}{10pt}\selectfont
    Define what the function does when called
  }
  
\end{frame}

\begin{frame}[fragile]
  \frametitle{Definition vs Declaration - example}

  \begin{lstlisting}[escapeinside={(*}{*)},basicstyle=\ttfamily\fontsize{12pt}{12pt}\selectfont]
double square(double); (*\tikzmark{declaration}*)

int main()
{
  double three = 3.;
  auto nine = square(three);
}

double square(double d) (*\tikzmark{definition}*)
{
  return d*d;
}
  \end{lstlisting}

  \nointerlineskip
  \begin{tikzpicture}[overlay,remember picture]
    \draw[pointy arrow] ([shift={(.5cm,.5ex)}] declaration) -- +(1.5cm,0)
      coordinate (arrow end)
      node[right,scale=0.75] {Declaration};
    \draw[pointy arrow] ([shift={(.5cm,.5ex)}] definition) -- +(1.5cm,0)
      node[right,scale=0.75] {Definition};
  \end{tikzpicture}

\end{frame}

\begin{frame}[fragile]
  \frametitle{Function overloading}

  Possible to create multiple functions with\\the same name but different argument list

  \vspace{.5em}
  This is called {\color{Tropiteal}overloading}

  \vspace{.5em}
  \begin{lstlisting}
int sum(int, int);
int sum(int, int, int);
double sum(double,double);
double sum(double,double,double);
  \end{lstlisting}

  \vspace{.5em}
  Helps create a consistent interface
  
\end{frame}

\begin{frame}[fragile]
  \frametitle{Function overloading - example}

  \begin{onlyenv}<1>
  \begin{lstlisting}[escapeinside={(*}{*)}]
double norm(double a, double b)
{
  return a*a + b*b;
}

double norm(double a, double b, double c)
{
  return a*a + b*b + c*c;
}
  \end{lstlisting}
  \end{onlyenv}

  \begin{onlyenv}<2>
  \begin{lstlisting}[escapeinside={(*}{*)}]
double norm(double a, double b)
{
  return a*a + b*b;
}

double norm(double a, double b, double c)
{
  return norm(a,b) + c*c;
}
  \end{lstlisting}
  \end{onlyenv}
  
\end{frame}

\begin{frame}[fragile]
  \frametitle{Variadic functions}

  Can also make a function with an undefined\\number of arguments

  \vspace{1em}
  These are called {\color{Tropiteal}variadic} functions

  \vspace{1em}
  But variadic functions in C aren't very pretty,\\so we will wait until we talk about templates

\end{frame}

\begin{frame}[fragile]
  \frametitle{Variadic functions - sneak peek}

  \begin{lstlisting}[escapeinside={(*}{*)}]
template <typename... Arguments>
double sum(double val1, Arguments... values)
{
  return val1 + sum(values...);
}

template<>
double sum(double d)
{
  return d;
}

sum(-5.2, 12.5);
sum(1, 4.5, 2.6, 9.4);
  \end{lstlisting}

  \CPPEleven

\end{frame}

\begin{frame}[fragile]
  \frametitle{Variadic functions - sneak peek}

  \begin{lstlisting}[escapeinside={(*}{*)}]
template <typename... Arguments>
double average(Arguments... values)
{
  auto number_of_arguments = sizeof...(values);
  return sum(values...)/number_of_arguments;
}

average(5, 9, 0);
average(1, 5, 1, 8, 3, 2, 2, 9);
  \end{lstlisting}

  \CPPEleven

\end{frame}

\begin{frame}[fragile]
  \frametitle{Default argument values}

  A variant of function overloading is giving the arguments default values

  \vspace{1em}
  This should be done in the function declaration

  \vspace{1em}
  \begin{lstlisting}[escapeinside={(*}{*)},basicstyle=\ttfamily\fontsize{12pt}{12pt}\selectfont]
double integerPower(double, int = 2);
  \end{lstlisting}
  
\end{frame}

\begin{frame}[fragile]
  \frametitle{Lambda functions}

  \begin{lstlisting}[escapeinside={(*}{*)},basicstyle=\ttfamily]
[(*\tikzmark{capture}*)...]((*\tikzmark{arg begin}*)double x, int n(*\tikzmark{arg end}*)){ (*\tikzmark{body}*)... }
  \end{lstlisting}

  \only<2>{
  \nointerlineskip
  \begin{tikzpicture}[overlay,remember picture]
    \draw[pointy arrow] ([shift={(1.4ex,2ex)}] capture) .. controls +(0,1cm) and +(-.75cm,0) .. +(.5cm,1.5cm)
      node[right,align=left,scale=0.8]
        {The capture list\\Gives the object more internal variables available in scope\\Allows for closures in C++};
    \draw[transform canvas={yshift=-.25em},decorate,decoration={brace,amplitude=5pt,mirror},line width=1pt]
      (arg begin) -- (arg end);
    \draw[transform canvas={yshift=-.5cm},pointy arrow] ($(arg begin) !.5! (arg end)$) -- +(0,-1cm)
      node[below,xshift=-1cm,scale=0.8] {The argument list};
    \draw[pointy arrow] ([shift={(1.4ex,-.75ex)}] body) -- +(0,-2cm)
      node[below,xshift=-1cm,scale=0.8] {The function body};
  \end{tikzpicture}
  }

  \CPPEleven
  
\end{frame}

\begin{frame}[fragile]
  \frametitle{Lambda functions}

  The lambda function is a function literal,\\so it must be assigned to a variable

  \vspace{1em}
  \begin{lstlisting}[escapeinside={(*}{*)}]
auto sum = [](int a, int b) { return a + b; };

int summed_value = sum(6,9);
  \end{lstlisting}

  \vspace{1em}
  The return type is inferred from the code

  \CPPEleven
  
\end{frame}

\begin{frame}[fragile]
  \frametitle{Lambda functions}

  The argument types can also be inferred from context

  \vspace{1em}
  \begin{lstlisting}[escapeinside={(*}{*)}]
auto sum = [](auto a, auto b) { return a + b; };

int summed_value = sum(6, 9);
double summed_floats = sum(5.6, 9.2);
  \end{lstlisting}

  \vspace{1em}
  This is a pet example of templates at work

  \CPPFourteen
  
\end{frame}

\LiveFrame

\section{Libraries}

\frame[plain]{\sectionpage}

\begin{frame}
  \frametitle{Using other people's work}

  There is no reason to reinvent the wheel\\every time you write a program

  \vspace{1em}
  You could of course write your own \lstinline!cos! function,\\but why would you do that?

  \vspace{.2em}
  {\footnotesize
    \{you would probably never get it as efficient and safe either\}
  }
  
\end{frame}

\begin{frame}[fragile]
  \frametitle{The {\color{Marty}\#include} statement}

  The include command copies the content\\of the specified file into the current file
  
\end{frame}

\begin{frame}[fragile]
  \frametitle{The {\color{Marty}\#include} statement}

  \nointerlineskip
  \begin{tikzpicture}[overlay,remember picture]
    \node[draw, line width=.5pt,anchor=north] at ([shift={(-3cm,1.5cm)}]current page.center) (header) [scale=0.5] {%
      \begin{minipage}{8cm}
        \begin{lstlisting}

int sum(int, int);
int sum(int,int,int);
double sum(double,double);

int highFive();
void sayHello();
        \end{lstlisting}
      \end{minipage}
    };

    \node[fill=WhiteTrash] at (header.north) [font=\ttfamily,scale=0.4] {header.hpp};

    \node[anchor=south west] at ([yshift=.5cm] header.north west) (pre) [scale=0.75] {Before preprocessing};

    \node[draw, line width=.5pt] at ([shift={(-3cm,-2.5cm)}]current page.center) (source) [scale=0.5] {%
      \begin{minipage}{8cm}
        \begin{lstlisting}

#include "header.hpp"

int main()
{
  auto total = sum(highFive(), 9);

  // ...

  for (auto i = 0; i < 10; ++i) {
    sayHello();
  }
}
        \end{lstlisting}
      \end{minipage}
    };

    \node[fill=WhiteTrash] at (source.north) [font=\ttfamily,scale=0.4] {main.cpp};


    \node[draw, line width=.5pt,anchor=north] at ([shift={(3cm,1.5cm)}]current page.center) (source post) [scale=0.5] {%
      \begin{minipage}{8cm}
        \begin{lstlisting}

int sum(int, int);
int sum(int,int,int);
double sum(double,double);

int highFive();
void sayHello();

int main()
{
  auto total = sum(highFive(), 9);

  // ...

  for (auto i = 0; i < 10; ++i) {
    sayHello();
  }
}
        \end{lstlisting}
      \end{minipage}
    };

    \node[fill=WhiteTrash] at (source post.north) [font=\ttfamily,scale=0.4] {main.cpp};

    \node[anchor=south west] at (pre.south -| source post.west) [scale=0.75] {After preprocessing};

    \draw[dashed] ([yshift=-2cm] current page.north) -- +(0,-7cm);
  \end{tikzpicture}

\end{frame}

\begin{frame}[fragile]
  \frametitle{Header files and source files}

  For larger projects one normally organises

  \vspace{1em}

  \begin{center}
    \tikz{%
      \node (decs) {declarations};
      \node[right=2cm of decs] (header) {header files};
      \draw[->,>=stealth,line width=1pt,shorten >=.3cm,shorten <=.3cm] (decs.east) -- (header.west);
    }
    
    \vspace{1em}
    \tikz{%
      \node (defs) {definitions};
      \node[right=2cm of defs] (source) {source files};
      \draw[->,>=stealth,line width=1pt,shorten >=.3cm,shorten <=.3cm] (defs.east) -- (source.west);
    }
  \end{center}
  
\end{frame}

\begin{frame}[fragile]
  \frametitle{Header guards - Motivation}

  Declaring a function more than once is illegal,\\but with all the includes it is hard to keep track
  
\end{frame}

\begin{frame}[fragile]
  \frametitle{{\color{Marty}Error:} Nested includes}

  \nointerlineskip
  \begin{tikzpicture}[overlay,remember picture]
    \node[draw, line width=.5pt,anchor=north] at ([shift={(0,2.5cm)}]current page.center) (header1) [scale=0.5] {%
      \begin{minipage}{10cm}
        \begin{lstlisting}

int sum(int, int);
int sum(int,int,int);
double sum(double,double);
        \end{lstlisting}
      \end{minipage}
    };

    \node[fill=WhiteTrash] at (header1.north) [font=\ttfamily,scale=0.4] {sum\_functions.hpp};

    \node[draw, line width=.5pt,anchor=north, below=.4cm of header1.south] (header2) [scale=0.5] {%
      \begin{minipage}{10cm}
        \begin{lstlisting}

#include "sum_functions.hpp"

int highFive();
void sayHello();
        \end{lstlisting}
      \end{minipage}
    };

    \node[fill=WhiteTrash] at (header2.north) [font=\ttfamily,scale=0.4] {utilities.hpp};

    \node[draw, line width=.5pt,anchor=north, below=.4cm of header2.south] (main) [scale=0.5] {%
      \begin{minipage}{10cm}
        \begin{lstlisting}

#include "sum_functions.hpp"
#include "utilities.hpp"

int main()
{
  auto total = sum(highFive(), 9);

  // ...

  for (auto i = 0; i < 10; ++i) {
    sayHello();
  }
}
        \end{lstlisting}
      \end{minipage}
    };

    \node[fill=WhiteTrash] at (main.north) [font=\ttfamily,scale=0.4] {main.cpp};

  \end{tikzpicture}
  
\end{frame}

\begin{frame}[fragile]
  \frametitle{{\color{Marty}Error:} Circular includes}

  \nointerlineskip
  \begin{tikzpicture}[overlay,remember picture]
    \node[draw, line width=.5pt,anchor=north] at ([shift={(0,2cm)}]current page.center) (header1) [scale=0.7] {%
      \begin{minipage}{10cm}
        \begin{lstlisting}

#include "utilities.hpp"

int sum(int, int);
int sum(int,int,int);
double sum(double,double);
        \end{lstlisting}
      \end{minipage}
    };

    \node[fill=WhiteTrash] at (header1.north) [font=\ttfamily,scale=0.6] {sum\_functions.hpp};

    \node[draw, line width=.5pt,anchor=north, below=.75cm of header1.south] (header2) [scale=0.7] {%
      \begin{minipage}{10cm}
        \begin{lstlisting}

#include "sum_functions.hpp"

int highFive();
void sayHello();
        \end{lstlisting}
      \end{minipage}
    };

    \node[fill=WhiteTrash] at (header2.north) [font=\ttfamily,scale=0.6] {utilities.hpp};
  \end{tikzpicture}
  
\end{frame}

\begin{frame}[fragile]
  \frametitle{Header guards}

  These issues can be resolved using header guards

  \begin{overlayarea}{\textwidth}{6cm}
      \begin{onlyenv}<1>
        \vspace{.6cm}
          \begin{center}
              \begin{tikzpicture}
                \node[draw, line width=.5pt,anchor=north] (header1) [scale=0.7] {%
                  \begin{minipage}{10cm}
                    \begin{lstlisting}
    
    #ifndef SUM_FUNCTIONS_H
    #define SUM_FUNCTIONS_H
    
    int sum(int, int);
    int sum(int,int,int);
    double sum(double,double);
    
    #endif /* SUM_FUNCTIONS_H */
                    \end{lstlisting}
                  \end{minipage}
                };
    
                \node[fill=WhiteTrash] at (header1.north) [font=\ttfamily,scale=0.6] {sum\_functions.hpp};
                \node at ([yshift=.8cm] header1.north) [scale=0.8] {Old Style};
              \end{tikzpicture}
          \end{center}
      \end{onlyenv}
    
      \begin{onlyenv}<2>
          \begin{center}
              \begin{tikzpicture}
                \node[draw, line width=.5pt,anchor=north] (header1) [scale=0.7] {%
                  \begin{minipage}{10cm}
                    \begin{lstlisting}
    
    #pragma once
    
    int sum(int, int);
    int sum(int,int,int);
    double sum(double,double);
                    \end{lstlisting}
                  \end{minipage}
                };
    
                \node[fill=WhiteTrash] at (header1.north) [font=\ttfamily,scale=0.6] {sum\_functions.hpp};
                \node at ([yshift=.8cm] header1.north) [scale=0.8] {New Style};
              \end{tikzpicture}
          \end{center}
      \end{onlyenv}
  \end{overlayarea}
  
\end{frame}

\begin{frame}[fragile]
  \frametitle{Header guards}

  The old style:

  \vspace{.5em}
  {\color{Marty}Cons:} Need a unique name for all headers

  \vspace{1em}
  The new style:

  \vspace{.5em}
  {\color{FeebleWeek}Pros:} Easy to use, no extra names

  \vspace{.5em}
  {\color{Marty}Cons:} Compiler support not guaranteed\\
  \hspace{1.2cm}{\footnotesize\{but the common compilers support it\}}



\end{frame}

\begin{frame}[fragile]
  \frametitle{{\color{Marty}\#include\raisebox{-1pt}{""}} \;vs\; {\color{Marty}\#include<>}}

  {\color{Marty}\#include\raisebox{-1pt}{""}}\\[.1cm]
  {\fontspec{Yanone Kaffeesatz Regular}
    Looks for the header files relative to the location of the source file
  }

  \vspace{1cm}
  {\color{Marty}\#include<>}\\[.1cm]
  {\fontspec{Yanone Kaffeesatz Regular}
    Looks for the header files in the system include folders, e.g.\\
    {\color{Tropiteal}/usr/include}, {\color{Tropiteal}/usr/local/include}, ...\\[.1cm]
    Can add more include folders with the {\color{Tropiteal}-I} compiler flags
  }

\end{frame}

\begin{frame}[fragile]
  \frametitle{Namespaces}

  You can further organise your code by putting\\all your functions and classes into namespaces

  \vspace{1em}
  Namespaces {\color{FeebleWeek}scope} the names you create when declaring functions and classes

  \vspace{1em}
  We have already encountered the {\color{Tropiteal}std namespace} introduced by the standard libraries
  
\end{frame}

\begin{frame}[fragile]
  \frametitle{Namespaces}

\begin{lstlisting}[escapeinside={(*}{*)}]
namespace Summers {

  int sum(int,int);

  namespace Printers {

    void prettyPrintSum(int,int);
  }
}

int main()
{
  auto val = Summers::sum(5,1);

  Summers::Printers::prettyPrintSum(10,5);
}
\end{lstlisting}
  
\end{frame}

\begin{frame}[fragile]
  \frametitle{Namespaces}

  Without namespaces you would have to check if any names in external libraries
  clashed with your every time you included a library

  \vspace{1em}
  C libraries often have absurdly long method names

  \vspace{1em}
  \begin{lstlisting}
gsl_vector * vec = gsl_vector_alloc(5);
gsl_multiroot_fsolver * s = gsl_multiroot_fsolver_alloc();
  \end{lstlisting}
  
\end{frame}

\begin{frame}[fragile]
  \frametitle{Namespaces}

  You can include a name from a namespace\\with the {\color{FeebleWeek}using} keyword

  \vspace{.5em}
  \begin{lstlisting}[escapeinside={(*}{*)}]
using std::cout, std::endl;
  \end{lstlisting}

  \vspace{1em}
  Or an entire namespace {\color{Marty}(but never do this!)}

  \begin{lstlisting}[escapeinside={(*}{*)}]
using namespace Summers;
  \end{lstlisting}
  
\end{frame}

\begin{frame}[fragile]
  \frametitle{{\color{FeebleWeek}inline} functions}

  To define a function in the header you\\have to declare it {\color{FeebleWeek}inline}

  \vspace{.5em}
  \begin{lstlisting}[escapeinside={(*}{*)}]
inline int sum(int a, int b)
{
  return a + b;
}
  \end{lstlisting}

  \vspace{.5em}
  If not the linker will complain about duplicate definitions

  
\end{frame}

\section{Compiling and Linking}

\frame[plain]{\sectionpage}

\begin{frame}[fragile]
  \frametitle{Steps of building a program}

  \vspace*{-.5cm}
  \begin{center}
    \begin{tikzpicture}
      \node[simple uml node] (source) {Source Files\\[5pt]{\ttfamily [.cpp]}};
      \node[simple uml node] [below=of source] (object) {Object Files\\[5pt]{\ttfamily [.o]}};
      \node[simple uml node] [below=of object] (exec) {Executable\\[5pt]{\ttfamily [.out]}};
      \draw[uml arrow] (source.south) -- (object.north);
      \draw[uml arrow] (object.south) -- (exec.north);
    
      \uncover<1-2>{
        \coordinate (mid source object) at ($(source) !.5! (object)$);
        \node[simple uml node,fill=AtomicBikini] at ([xshift=5cm] mid source object) (preproc) {Preprocessor};
        \node[simple uml node,fill=AtomicBikini] [below=of preproc] (compiler) {Compiler};
        \draw[uml arrow, dashed]  (mid source object) -- (preproc.west);
        \draw[uml arrow, dashed] (preproc.south) -- (compiler.north);

        \uncover<2>{
          \node[above=1.5cm of preproc.north west,anchor=west] [text width=7cm, scale=0.6] (preproc comm) {%
            Copies {\ttfamily\#include} statements\\
            Removes comments, etc.
          };
          \draw[dashed,-{Stealth[bend]},shorten >=0pt]
            (preproc.north west) .. controls +(-.2cm,.2cm) and +(-.5cm,-.5cm) .. (preproc comm.south west);

          \node[below=1.5cm of compiler.south west,anchor=west] [text width=6cm, scale=0.6] (compiler comm) {%
            Translates source code to machine code and creates flags for the linker
          };
          \draw[dashed,-{Stealth[bend]}]
            (compiler.south east) .. controls +(.3cm,-.3cm) and +(.3cm,.3cm) .. ([xshift=-.4cm] compiler comm.north east);
        }
      }

      \uncover<3-4>{
        \coordinate (mid object exec) at ($(object) !.5! (exec)$);
        \node[simple uml node,fill=AtomicBikini] at ([xshift=5cm] mid object exec) (linker) {Linker};
        \draw[uml arrow, dashed] (mid object exec) -- (linker.west);

        \uncover<4>{
          \node[above=2cm of linker.north west, anchor=west] [text width=6cm, scale=0.6] (linker comm) {%
            Puts the executable\\together, filling in function definitions in the various source files
          };
          \draw[dashed,-{Stealth[bend]},shorten >=0pt]
            (linker.north west) .. controls +(-.2cm,.2cm) and +(-.5cm,-.5cm) .. (linker comm.south west);
        }
      }
    \end{tikzpicture}
  \end{center}
\end{frame}

\begin{frame}[fragile]
  \frametitle{Steps of building a program}

  All of the above steps happen if you write

  \vspace{.5em}
  \begin{lstlisting}[language=bash,alsoletter={+}]
g++ -o program source1.cpp source2.cpp
  \end{lstlisting}

  \vspace{1em}
  Can use the \lstinline[language=bash]!-c! flag to compile only

  \vspace{.5em}
  \begin{lstlisting}[language=bash,escapeinside={(*}{*)},alsoletter={+}]
g++ -o source1.o -c source1.cpp (*\tikzmark{compile 1}*)
g++ -o source2.o -c source2.cpp (*\tikzmark{compile 2}*)
g++ -o program source1.o source2.o (*\tikzmark{linking}*)
  \end{lstlisting}

  \nointerlineskip
  \begin{tikzpicture}[overlay,remember picture]
    \coordinate (compile begin) at ([shift={(.5cm,.3ex)}] compile 1);
    \draw[pointy arrow] (compile begin) -- +(1.8cm,0)
      coordinate (compile end)
      node[right,scale=0.75] {compiling};
    \coordinate (linking begin) at ([shift={(.5cm,.3ex)}] linking);
    \draw[pointy arrow] (linking begin) -- (linking begin -| compile end)
      node[right,scale=0.75] {linking};
  \end{tikzpicture}

\end{frame}

\begin{frame}[fragile]
  \frametitle{Quick introduction to makefiles}

  \only<1>{
  We use a build manager to automatise this process

  \vspace{1em}
  Examples:

  make, cmake, QMake, various IDE's, ...
  }

  \begin{onlyenv}<2>
  A makefile is simply a list of ingredients and results for the various stages of compiling

  \vspace{.5em}
  \begin{lstlisting}[language=make,basicstyle=\ttfamily]
result : ingredients
    method to obtain
  \end{lstlisting}
  \end{onlyenv}
  
\end{frame}

\begin{frame}[fragile]
  \frametitle{Quick introduction to makefiles}

  \begin{lstlisting}[language=gnumake,alsoletter={<,+},morekeywords={&<}]
program : source1.o source2.o
  g++ -o program source1.o source2.o

source1.o : source1.cpp
  g++ -o source1.o -c source1.cpp

source2.o : source2.cpp
  g++ -o source2.o -c source2.cpp
  \end{lstlisting}
\end{frame}

\begin{frame}[fragile]
  \frametitle{Quick introduction to makefiles}

  \begin{lstlisting}[language=gnumake,alsoletter={<,+},morekeywords={&<,SRCS,OBJS}]
SRCS := $(wildcard *.cpp)
OBJS := $(SRCS:%.cpp=obj/%.o)

program : $(OBJS)
  g++ $(OBJS) -o $@

obj/%.o : %.cpp | obj
  g++ -c $< -o $@

obj :
  @mkdir -p obj
  \end{lstlisting}
\end{frame}

\begin{frame}[fragile]
  \frametitle{Compiling vs Linking}

  When compiling the code is converted to machine code, but function definitions might still be missing

  \vspace{1em}
  These "holes" are filled when the program is linked and an executable is created
  
\end{frame}

\begin{frame}[fragile]
  \frametitle{Various types of errors}

  There are in essence {\large\color{Marty}3} types of errors

  \only<1>{\vspace{1em}}\only<2->{\vspace{.5em}}
  { \setbeamercolor*{item}{fg=Marty}
  \begin{enumerate}
    \only<1>{\setlength\itemsep{1em}}\only<2->{\setlength\itemsep{.5em}}
    \item Compile errors \tikzmark{compile error}
    \only<2>{
      \vspace{.3em}
      \begin{itemize}
        \item Syntactic errors
        \item Template lookup errors
        \item Type conversion errors
      \end{itemize}
    }
    \item Linking errors \tikzmark{linking error}
    \only<3>{
      \vspace{.3em}
      \begin{itemize}
        \item Multiple definitions
        \item Function definition not found
      \end{itemize}
    }
    \item Runtime errors \tikzmark{runtime error}
    \only<4>{
      \vspace{.3em}
      \begin{itemize}
        \item Unexpected behaviour
        \item Memory issues
        \item Infinite loops
      \end{itemize}
    }
  \end{enumerate}
  }

  \nointerlineskip
  \begin{tikzpicture}[overlay,remember picture]
    \only<2>{
      \draw[transform canvas={shift={(3cm,1.5ex)}},decorate,decoration={brace,amplitude=5pt},line width=1pt]
        (compile error) -- (linking error -| compile error)
        node[midway,right=.2cm,scale=0.8,text width=5cm] {%
          Normally easy to find with a good coding environment, comes with practice
        };
    }
    \only<3>{
      \draw[transform canvas={shift={(3cm,1.5ex)}},decorate,decoration={brace,amplitude=5pt},line width=1pt]
        (linking error -| compile error) -- (runtime error -| compile error)
        node[midway,right=.2cm,scale=0.8,text width=5cm] {%
          A bit harder to find but in essence easy, also might require some practice
        };
    }
    \only<4>{
      \draw[transform canvas={shift={(3cm,1.5ex)}},decorate,decoration={brace,amplitude=5pt},line width=1pt]
        (runtime error -| compile error) -- +(0,-8ex)
        node[midway,right=.2cm,scale=0.8,text width=4cm] {%
          This is the real killer, need more advanced tools
        };
    }
  \end{tikzpicture}
  
\end{frame}

\section{Debugging}

\frame[plain]{\sectionpage}

\begin{frame}[fragile]
  \frametitle{Runtime errors}

  \begin{lstlisting}[escapeinside={(*}{*)}]
int sumArray(int array[], unsigned size)
{
  unsigned index = 0;
  int result = array[index];

  do {
    ++index;
    result += array[index];
  } while (index < size);

  return result;
}
  \end{lstlisting}
  
\end{frame}

\begin{frame}[fragile]
  \frametitle{Runtime errors}

  \begin{lstlisting}[escapeinside={(*}{*)}]
double volumeOfCone(double r, double h)
{
  static const double pi = 4*std::atan(1);
  return static_cast<double>(1/3)*pi*r*r*h;
}
  \end{lstlisting}
  
\end{frame}

\begin{frame}[fragile]
  \frametitle{Runtime errors}

  \begin{lstlisting}
void print(int ** array, unsigned size)
{
  for (int i = 0; i < size; ++i) {
    for (int j = 0; i < size; ++i)
      std::cout << array[i][j] << " ";

    std::cout << std::endl;
  }
}
  \end{lstlisting}
  
\end{frame}

\begin{frame}[fragile]
  \frametitle{Runtime errors}

  \begin{lstlisting}[basicstyle=\ttfamily\fontsize{8pt}{8pt}\selectfont]
// Allocate memory
// Return: whether allocation was successful
bool allocate(int *, unsigned size);

void initialise(int * array, unsigned size, bool set_to_zero)
{
  //Only set to zero if allocation was successful
  if (set_to_zero && allocate(array,size)) {
    for (auto i = 0; i < size; ++i)
      array[i] = 0;
  }
}
  \end{lstlisting}
  
\end{frame}

\begin{frame}[fragile]
  \frametitle{Runtime errors}

  \begin{lstlisting}[escapeinside={(*}{*)}]
unsigned factorial(unsigned n)
{
  unsigned result = 1;
  while (n > 1) {
    result *= --n;
  }

  return result;
}
  \end{lstlisting}
  
\end{frame}

\begin{frame}[fragile]
  \frametitle{Runtime errors}

  \begin{lstlisting}[escapeinside={(*}{*)}]
unsigned f(unsigned n){return !n?1:--n*f(n);}
  \end{lstlisting}
  
\end{frame}

\begin{frame}[fragile]
  \frametitle{"printf debugging"}

  Print the current state of the variables

  \hspace{1em}
  \begin{lstlisting}[basicstyle=\ttfamily\fontsize{8pt}{8pt}\selectfont,escapeinside={(*}{*)}]
int sumArray(int array[], unsigned size)
{
  unsigned index = 0;
  int result = array[index];

  do {
    std::cout << index << std::endl; (*\tikzmark{print1}*)
    ++index;
    result += array[index];
  } while (index < size);

  return result;
}
  \end{lstlisting}

  \nointerlineskip
  \begin{tikzpicture}[overlay,remember picture]
    \draw[pointy arrow] ([shift={(.2cm,.3ex)}] print1) -- +(2cm,0)
      node[right,scale=0.8] {Here?};
  \end{tikzpicture}
  
\end{frame}

\begin{frame}[fragile]
  \frametitle{"printf debugging"}

  Print the current state of the variables

  \hspace{1em}
  \begin{lstlisting}[basicstyle=\ttfamily\fontsize{8pt}{8pt}\selectfont,escapeinside={(*}{*)}]
int sumArray(int array[], unsigned size)
{
  unsigned index = 0;
  int result = array[index];

  do {
    ++index;
    std::cout << index << std::endl; (*\tikzmark{print2}*)
    result += array[index];
  } while (index < size);

  return result;
}
  \end{lstlisting}

  \nointerlineskip
  \begin{tikzpicture}[overlay,remember picture]
    \draw[pointy arrow] ([shift={(.2cm,.3ex)}] print2) -- +(2cm,0)
      node[right,scale=0.8] {Here?};
  \end{tikzpicture}
  
\end{frame}

\begin{frame}[fragile]
  \frametitle{"printf debugging"}

  {\color{FeebleWeek}\large Advantage:}
  \begin{itemize}
    \item Requires no additional knowledge
  \end{itemize}

  \vspace{1em}
  {\color{Marty}\large{}Disadvantage:}
  { \setbeamercolor*{item}{fg=Marty}
  \begin{itemize}
    \item A very static way of debugging
    \item Have to recompile every time
    \item If you realise something mid debugging,\\there is no way out
    \item Important information can be lost in output
  \end{itemize}
  }

\end{frame}

\begin{frame}[fragile]
  \frametitle{Assert statements}

  An {\color{Tropiteal}assert} is a runtime test that terminate the program if it fails

  \vspace{1em}
  Enabled by including the {\ttfamily\color{sorange}<cassert>} library

  \vspace{1em}
  Can be disabled using the {\ttfamily\color{sorange}NDEBUG} preprocessor flag\\

  \vspace{.5em}
  \begin{tikzpicture}
    \node (code) {%
      \begin{minipage}{3.5cm}
  \begin{lstlisting}[escapeinside={(*}{*)}]
#define NDEBUG
  \end{lstlisting}
      \end{minipage}
    };
    \node[right=of code] (compiler) {%
      \begin{minipage}{6cm}
  \begin{lstlisting}[language=bash,alsoletter={+},deletekeywords={source}]
g++ source.cpp -DNDEBUG    
  \end{lstlisting}
      \end{minipage}
    };

    \node at ([xshift=-.8cm] compiler.west) [scale=.75] {or};
  \end{tikzpicture}

  
\end{frame}

\begin{frame}[fragile]
  \frametitle{Assert statements}

  \begin{lstlisting}[basicstyle=\ttfamily\fontsize{9pt}{9pt}\selectfont,escapeinside={(*}{*)}]
//Uncomment to disable
//#define NDEBUG
#include<cassert>

int sumArray(int array[], unsigned size)
{
  unsigned index = 0;
  int result = array[index];

  do {
    ++index;
    assert(index < size); (*\tikzmark{assert}*)
    result += array[index];
  } while (index < size);

  return result;
}
  \end{lstlisting}

  \nointerlineskip
  \begin{tikzpicture}[overlay,remember picture]
    \draw[pointy arrow] ([shift={(.2cm,.35ex)}] assert) -- +(2cm,0)
      node[right,scale=0.75] {Check before accessing};
  \end{tikzpicture}
  
\end{frame}

\begin{frame}[fragile]
  \frametitle{Assert statements}

  This is safer than printing as we can terminate the program, but it is still not very dynamic

  \vspace{1.5em}
  To achieve a more dynamic debugging process,\\we will introduce dedicated debuggers

\end{frame}

\LiveFrame

\section{Programming Practices}

\frame[plain]{\sectionpage}

\begin{frame}
  \frametitle{Good Programming Practices}

  \begin{itemize}
    \setlength\itemsep{0.5em}
    \item Don't Repeat Yourself (stay DRY)
    \item Write functions with a single responsibility
    \item Make use of namespaces to keep the global scope clean
    \item {\color{Marty}Never} include the std namespace
    \item Use debuggers to find runtime errors
  \end{itemize}
  
\end{frame}

\begin{frame}[fragile]
  \frametitle{Useless refactoring}

  This is utterly useless...

  \vspace{.5em}
  \begin{lstlisting}
int runProgram();

int main()
{
  return runProgram();
}

int runProgram()
{
  //Everything that was in main
}
  \end{lstlisting}
  
\end{frame}

\section{Recap}

\frame[plain]{\sectionpage}

\begin{frame}
  \frametitle{Recap Day 2}

  \begin{itemize}
    \setlength\itemsep{0.75em}
    \item Variable visibility and lifetime is governed by its scope
    \item A function is a unit that {\color{FeebleWeek}does something}
    \item Variables can be passed by value or reference
    \item {\color{FeebleWeek}static} variables outlive their scope
    \item One can separate definition and declaration
  \end{itemize}
  
\end{frame}

\begin{frame}
  \frametitle{Recap Day 2}

  \begin{itemize}
    \setlength\itemsep{0.75em}
    \item Function declarations can be {\color{sorange}\#include} 'd
    \item There are two steps to building a program: {\color{Tropiteal}compiling} and {\color{FeebleWeek}linking}
    \item We have three types of errors:\\{\color{Marty}compiling}, {\color{Marty}linking} and {\color{Marty}runtime}
    \item Dedicated debuggers can be used to find\\runtime errors
  \end{itemize}
  
\end{frame}

\end{document}
