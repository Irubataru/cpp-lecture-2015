\documentclass[14pt,a4paper,dvipsnames,usenames]{beamer}
\usepackage[T1]{fontenc}
\usepackage{url}
\usepackage{xcolor}
\usepackage{amsmath,varwidth}
\usepackage{tikz,tikz-uml,xparse}
\usepackage{listings,algpseudocode}
\usepackage[quiet]{mathspec}

\setmainfont[
  ItalicFont={Yanone Kaffeesatz Light},
  Scale=1.3,
  LetterSpace=2.0
]{Yanone Kaffeesatz Bold}

\setmonofont{Hack}

% --------------------- %
% Beamer theme settings %
% --------------------- %

\usetheme{CodeCourse}
\setbeamertemplate{navigation symbols}{} %Remove beamer navigation symbols
\setbeamertemplate{bibliography item}{\insertbiblabel}

% ---------------------------- %
% Colour definitions for theme %
% ---------------------------- %

\definecolor{Tropiteal}{RGB}{0,168,198}
\definecolor{TealDrop}{RGB}{64,192,203}
\definecolor{WhiteTrash}{RGB}{249,242,231}
\definecolor{AtomicBikini}{RGB}{174,226,57}
\definecolor{FeebleWeek}{RGB}{143,190,0}
\definecolor{ICantExpress}{RGB}{28,20,13}
\definecolor{Love}{RGB}{209,67,52}
\definecolor{UnrealFoodPills}{RGB}{250,105,0}
\definecolor{Marty}{RGB}{250,42,0}
\definecolor{ConradGold}{RGB}{255,221,0}

\definecolor{sbase03}{HTML}{002B36}
\definecolor{sbase02}{HTML}{073642}
\definecolor{sbase01}{HTML}{586E75}
\definecolor{sbase00}{HTML}{657B83}
\definecolor{sbase0}{HTML}{839496}
\definecolor{sbase1}{HTML}{93A1A1}
\definecolor{sbase2}{HTML}{EEE8D5}
\definecolor{sbase3}{HTML}{FDF6E3}
\definecolor{syellow}{HTML}{B58900}
\definecolor{sorange}{HTML}{CB4B16}
\definecolor{sred}{HTML}{DC322F}
\definecolor{smagenta}{HTML}{D33682}
\definecolor{sviolet}{HTML}{6C71C4}
\definecolor{sblue}{HTML}{268BD2}
\definecolor{scyan}{HTML}{2AA198}
\definecolor{sgreen}{HTML}{859900}

% -------------------------------------------------- %
% Beamer notes to be itemized rather than enumerated %
% -------------------------------------------------- %

\makeatletter
\def\beamer@setupnote{%
  \gdef\beamer@notesactions{%
    \beamer@outsideframenote{%
      \beamer@atbeginnote%
      \beamer@notes%
      \ifx\beamer@noteitems\@empty\else
      \begin{itemize}\itemsep=0pt\parskip=0pt%
        \beamer@noteitems%
      \end{itemize}%
      \fi%
      \beamer@atendnote%
    }%
    \gdef\beamer@notesactions{}%
  }
}
\makeatother

% -------------------------- %
% Standard LaTeX definitions %
% -------------------------- %

\DeclareGraphicsExtensions{.pdf}
\renewcommand\textbullet{\ensuremath{\bullet}}

% ----------------------------------- %
% Various tikz styles and definitions %
% ----------------------------------- %

\usetikzlibrary{calc,intersections,positioning,matrix,chains,scopes}
\usetikzlibrary{decorations.pathmorphing,decorations.pathreplacing,shapes}
\newcommand{\tikzmark}[1]{\tikz[overlay,remember picture] \node (#1) {};}

\tikzumlset{fill class=RoyalBlue!20}

\tikzstyle{startnode} = [circle, draw, fill=Bittersweet!20, minimum size=10pt, inner sep = 0]
\tikzstyle{stopnode} = [startnode]
\tikzstyle{process} = [rectangle, minimum width=2cm, minimum height=.7cm, text centered, draw, fill=OliveGreen!20, align=center, font=\scriptsize]
\tikzstyle{dangerprocess} = [process, fill=BrickRed!20, draw=BrickRed, thick]
\tikzstyle{flowarrow} = [->, >=stealth, thick]

% ---------------------------------- %
% My default listings C++ code style %
% ---------------------------------- %

\lstset{%
  language=C++, basicstyle=\scriptsize\ttfamily, 
  keywordstyle=\color{OliveGreen}, identifierstyle=\color{RoyalBlue}, 
  commentstyle=\color{Brown}, stringstyle=\color{Bittersweet}, showstringspaces=false,  
  breaklines=true, prebreak=\mbox{{\color{Bittersweet}\tiny\ $\hookleftarrow$}}, 
  postbreak=\mbox{{\color{Bittersweet}\tiny$\to$\ }}, tabsize=5,
  morekeywords={%
    MyClass,unique_ptr,shared_ptr,weak_ptr,auto_ptr,
    make_unique, make_shared, default_delete,
    dynamic_pointer_cast, const_pointer_cast,
    scoped_ptr,QSharedPointer,QScopedPointer,QWeakPointer,
    ftring, SmartPtr, Shape, Circle, ShapeFactory, CircleFactory,
    Derived, Base
  }
}

\def\inline{\lstinline[basicstyle=\ttfamily\normalsize]}

% ---------------------------------------------------------------------------------------------------- %
% Various command for inline code instead of lstinline so that I don't have to update the keyword list %
% ---------------------------------------------------------------------------------------------------- %

\newcommand{\object}[1]{{\ttfamily \color{OliveGreen}#1}}
\newcommand{\function}[1]{{\ttfamily \color{RoyalBlue}#1}}
\newcommand{\std}[1]{{\ttfamily {\color{RoyalBlue} std::}{\color{OliveGreen}#1}}}
\newcommand{\boost}[1]{{\ttfamily {\color{RoyalBlue} boost::}{\color{OliveGreen}#1}}}

\newcommand{\uniqueptr}{\std{unique\_ptr}}
\newcommand{\sharedptr}{\std{shared\_ptr}}
\newcommand{\weakptr}{\std{weak\_ptr}}
\newcommand{\autoptr}{\std{auto\_ptr}}

\usepackage{array}

% ------------------------------------------------ %
% tikz functions to place markers on beamer slides %
% ------------------------------------------------ %

\makeatletter
\tikzset{%
  remember picture with id/.style={%
    remember picture,
    overlay,
    save picture id=#1,
  },
  save picture id/.code={%
    \edef\pgf@temp{#1}%
    \immediate\write\pgfutil@auxout{%
      \noexpand\savepointas{\pgf@temp}{\pgfpictureid}}%
  },
  if picture id/.code args={#1#2#3}{%
    \@ifundefined{save@pt@#1}{%
      \pgfkeysalso{#3}%
    }{
      \pgfkeysalso{#2}%
    }
  }
}

\def\savepointas#1#2{%
  \expandafter\gdef\csname save@pt@#1\endcsname{#2}%
}

\def\tmk@labeldef#1,#2\@nil{%
  \def\tmk@label{#1}%
  \def\tmk@def{#2}%
}

\tikzdeclarecoordinatesystem{pic}{%
  \pgfutil@in@,{#1}%
  \ifpgfutil@in@%
    \tmk@labeldef#1\@nil
  \else
    \tmk@labeldef#1,(0pt,0pt)\@nil
  \fi
  \@ifundefined{save@pt@\tmk@label}{%
    \tikz@scan@one@point\pgfutil@firstofone\tmk@def
  }{%
  \pgfsys@getposition{\csname save@pt@\tmk@label\endcsname}\save@orig@pic%
  \pgfsys@getposition{\pgfpictureid}\save@this@pic%
  \pgf@process{\pgfpointorigin\save@this@pic}%
  \pgf@xa=\pgf@x
  \pgf@ya=\pgf@y
  \pgf@process{\pgfpointorigin\save@orig@pic}%
  \advance\pgf@x by -\pgf@xa
  \advance\pgf@y by -\pgf@ya
  }%
}
\newcommand\tikzsupermark[2][]{%
\tikz[remember picture with id=#2] #1;}
\makeatother

\newcommand{\MarkText}[3][]{%
\begin{tikzpicture}[overlay,remember picture]%
  \path (pic cs:#2,{(0,\paperheight)}) +(0,.7\baselineskip) coordinate (a);
  \path (pic cs:#3,{(0,-\paperheight)}) +(0,-.3\baselineskip) coordinate (b);
  \draw [ultra thick,
    if picture id={#2}{red}{line width=1cm,green,opacity=.5},
    #1]
  (a) -- ($(b)!(a)!($(b)+(1,0)$)$);
\end{tikzpicture}%
}%

\newcommand{\tikzmark}[1]{\tikz[overlay,remember picture] \coordinate (#1);}

% -------------- %
% Special frames %
% -------------- %

\newcommand{\LiveFrame}{%
\begin{frame}[plain]
  \nointerlineskip
  \begin{tikzpicture}[overlay,remember picture]
    \fill[FeebleWeek] (current page.north east) rectangle (current page.south west);
    \node at ([yshift=-.25em] current page.center) [font=\LARGE, WhiteTrash] {Live Example};
  \end{tikzpicture}
\end{frame}
}

\newcommand{\CPPEleven}{%
\nointerlineskip
\begin{tikzpicture}[overlay,remember picture]
  \node[anchor=south east] at ( current page.south east) [color=Tropiteal] {\{C++11\}};
\end{tikzpicture}
}

\newcommand{\CPPFourteen}{%
\nointerlineskip
\begin{tikzpicture}[overlay,remember picture]
  \node[anchor=south east] at ( current page.south east) [color=Tropiteal] {\{C++14\}};
\end{tikzpicture}
}

\newcommand{\BadPractice}{%
\nointerlineskip
\begin{tikzpicture}[overlay,remember picture]
  \node[anchor=south east] at ( current page.south east) [color=Marty] {Bad Practice};
\end{tikzpicture}
}

\newcommand{\ClassLine}{%
\nointerlineskip
\begin{tikzpicture}[overlay,remember picture]
  \draw[transform canvas={yshift=-1.7cm},thin] (current page.north west) -- (current page.north east);
\end{tikzpicture}
}

\newcommand{\Library}[2][0pt]{%
\nointerlineskip
\begin{tikzpicture}[overlay,remember picture]
  \node[anchor=south east] at ([xshift=#1] current page.south east) [color=Marty,scale=0.9]
    [font=\fontspec{Yanone Kaffeesatz Regular}] {\#include<#2>};
\end{tikzpicture}
}

% ----------------------------------------- %
% Various Commands for inline codecolouring %
% ----------------------------------------- %

\newcommand{\inobj}[1]{%
  {\color{Tropiteal}#1}}

\newcommand{\inkey}[1]{%
  {\color{FeebleWeek}#1}}

\newcommand{\innorm}[1]{%
  {\color{ICantExpress}#1}}

\newcommand{\std}[1]{%
  \inobj{std}\innorm{::}\inkey{#1}}

% ----------------------------------------------------- %
% Various tikz based functions to style up the document %
% ----------------------------------------------------- %

\newcommand{\underlinemark}[3][line width=1.5pt, FeebleWeek]{%
\draw[transform canvas={yshift=-.15cm}, #1]
  ([xshift=-1pt] #2) -- ([xshift=2pt] #3);
}

\newcommand{\StrikeThrough}[2][Marty]{%
  \tikz[baseline=(strokethrough.base)]{%
    \node[inner sep=0pt, outer sep=0pt] (strokethrough) {\strut{}#2};
    \draw[line width=2pt,#1] ([xshift=-3pt] strokethrough.west) -- ([xshift=3pt] strokethrough.east);
  }
}

\newcommand{\Highlight}[3][opacity=0.4,AtomicBikini]{%
\nointerlineskip
\begin{tikzpicture}[overlay,remember picture]
  \path (pic cs:#2,{(0,0)}) coordinate (a);
  \path (pic cs:#3,{(0,0)}) coordinate (b);
  \fill[#1] (a -| current page.north west) -- ++(0,1.4ex) -- ++(\paperwidth,0)
  -- (b -| current page.north east) -- ++(0,-.4ex) -- ++(-\paperwidth,0) -- cycle;
\end{tikzpicture}
}

\newcommand{\Dimtext}[2]%
{
  { \transparent{0.7}
  \begin{tikzpicture}[overlay, remember picture]
    \fill[white] ( #1 -| current page.north west) -- ++(0,.8em) -- ++(\paperwidth,0) -- (#2 -| current page.north east)
   -- ++(0,-.5em) -- ++(-\paperwidth,0) -- cycle;
  \end{tikzpicture}
  }
}

% -------------- %
% Array commands %
% -------------- %

\newcolumntype{L}[1]{>{\raggedright\let\newline\\\arraybackslash\hspace{0pt}}m{#1}}
\newcolumntype{C}[1]{>{\centering\let\newline\\\arraybackslash\hspace{0pt}}m{#1}}
\newcolumntype{R}[1]{>{\raggedleft\let\newline\\\arraybackslash\hspace{0pt}}m{#1}}


\usefonttheme{serif}

\pgfdeclarelayer{background}
\pgfdeclarelayer{foreground}
\pgfsetlayers{background,main,foreground}

%\setbeameroption{show notes}

\title[C++ Day4]{Introduction to the C++\newline{}Programming Language\newline{}\newline{}\fontsize{16pt}{16pt}\selectfont{}Day 4}
\author{\texorpdfstring{%
    Aleksandra Rylund Glesaaen\newline\fontsize{12pt}{12pt}\selectfont\texttt{aleksandra@glesaaen.com}%
  }{%
    Aleksandra Rylund Glesaaen}}
\date{October 1st 2015}

\begin{document}

\frame{\titlepage}

\begin{frame}
  \frametitle{What will we learn?}

  \begin{itemize}
    \setlength\itemsep{.5em}
    \item \StrikeThrough{Basic C++ syntax}
    \item \StrikeThrough{Control structures}
    \item \StrikeThrough{Functions}
    \item Structs and classes \hspace{.25cm}{\color{Marty}(Wednesday and today)}
    \item Templates and STL \hspace{.5cm}{\color{Marty}(today and Friday)}
    \item Exceptions \hspace{.5cm}{\color{Tropiteal}(Friday)}
  \end{itemize}
\end{frame}

\begin{frame}
  \frametitle{Today's topics}

  \tableofcontents
  
\end{frame}

\section{Inheritance}

\frame[plain]{\sectionpage}

\begin{frame}
  \frametitle{Subtypes and supertypes}

  Inheritance lets you create new classes based on existing ones

  \vspace{1em}
  Subtypes/children are more specialised than their supertypes/parents

  \vspace{1em}
  Children inherit all the variables and methods of their parents

\end{frame}

\begin{frame}[fragile]
  \frametitle{Simple Example}

  \begin{lstlisting}[escapeinside={(*}{*)},morekeywords={Animal,Bird,Fish}]
class Animal (*\tikzmark{animal begin}*)
{
public:
  void eat();
}; (*\tikzmark{animal end}*)

class Bird : public Animal (*\tikzmark{bird begin}*)
{
public:
  void fly();
}; (*\tikzmark{bird end}*)
  \end{lstlisting}

  \nointerlineskip
  \begin{tikzpicture}[overlay,remember picture]
    \coordinate (layer one) at ([xshift=1.5cm] current page.north);

    \coordinate (animal top) at ([yshift=1.25ex] animal begin);
    \coordinate (animal bottom) at ([yshift=-.5ex] animal end);

    \coordinate (bird top) at ([yshift=1.25ex] bird begin);
    \coordinate (bird bottom) at ([yshift=-.5ex] bird end);

    \only<1>{
      \draw[decorate,decoration={brace,amplitude=5pt},line width=1pt]
        (layer one |- animal top) -- (layer one |- animal bottom)
        node[midway,right=.5em,align=left,scale=0.75] {%
          Every animal can eat
        };
    }
    \only<2>{
      \draw[decorate,decoration={brace,amplitude=5pt},line width=1pt]
        (layer one |- bird top) -- (layer one |- bird bottom)
        node[midway,right=.5em,align=left,scale=0.75] {%
          Every bird is an animal\\[5pt]
          \hspace{.1cm}\tikz[baseline=-.6ex] \draw[-{Stealth},line width=1pt] (0,0) -- (.5cm,0); Every bird can eat\\[10pt]
          Every bird can fly {\fontsize{8pt}{8pt}\selectfont\{well...\}}
        };
    }
    \only<3>{
      \draw[decorate,decoration={brace,amplitude=5pt},line width=1pt]
        (layer one |- animal top) -- (layer one |- bird bottom)
        node[midway,right=.5em,align=left,scale=0.75] {%
          Not every animal is a bird\\[5pt]
          \hspace{.1cm}\tikz[baseline=-.6ex] \draw[-{Stealth},line width=1pt] (0,0) -- (.5cm,0); Not all animals can fly
        };
    }
  \end{tikzpicture}
  
\end{frame}

\begin{frame}[fragile]
  \frametitle{Simple Example}

  \begin{onlyenv}<2>
  \nointerlineskip
  \begin{tikzpicture}[overlay,remember picture]
    \path (pic cs:cow fly,{(0,0)}) coordinate (a);
    \fill[opacity=.15,Marty] (a -| current page.north west) -- ++(0,1.1ex) -- ++(\paperwidth,0)
    -- (a -| current page.north east) -- ++(0,-.4ex) -- ++(-\paperwidth,0) -- cycle;
    \node at (a -| current page.east) [scale=0.75,anchor=east,xshift=-1cm,yshift=.1cm] {{\color{Marty}Error:} cows can't fly};
  \end{tikzpicture}
  \end{onlyenv}
  
  \begin{overlayarea}{\textwidth}{6cm}
  \begin{onlyenv}<1>
  \begin{lstlisting}[escapeinside={(*}{*)},morekeywords={Animal,Bird,Fish}]
#include"animals.hpp"

int main()
{
  Animal cow;
  cow.eat();

  Bird penguin;
  penguin.eat();
  penguin.fly();
}
  \end{lstlisting}
  \end{onlyenv}

  \begin{onlyenv}<2>

  \begin{lstlisting}[escapeinside={(*}{*)},morekeywords={Animal,Bird,Fish}]
#include"animals.hpp"

int main()
{
  Animal cow;
  cow.eat();
  cow.fly(); (*\tikzsupermark{cow fly}*)

  Bird penguin;
  penguin.eat();
  penguin.fly();
}
  \end{lstlisting}
  \end{onlyenv}
  \end{overlayarea}

\end{frame}

\begin{frame}
  \frametitle{Inheritance and access}

  {\large\centering
    My {\color{Marty}\Large\{}\tikzmark{begin} \hspace{2cm} \tikzmark{end}{\color{Marty}\Large\}} are not my friends\\[5pt]
    \only<1>{\fontsize{8pt}{8pt}\selectfont (they might be bastards)}
    \only<2>{\fontsize{8pt}{8pt}\selectfont (they are definitely up to no good)}
    \only<3>{\fontsize{8pt}{8pt}\selectfont (they weren't chosen by me)}
  \par}
  
  \nointerlineskip
  \begin{tikzpicture}[overlay,remember picture]
    \coordinate (middle) at ($(begin) !.5! (end)$);
    \only<1>{
      \node[yshift=.5ex] at (middle) [font=\large] {children\strut{}};
    }
    \only<2>{
      \node[yshift=.5ex] at (middle) [font=\large] {siblings\strut{}};
    }
    \only<3>{
      \node[yshift=.5ex] at (middle) [font=\large] {parents\strut{}};
    }

  \end{tikzpicture}

\end{frame}

\begin{frame}
  \frametitle{Inheritance and access}

  Inherited classes do not have access to the base\\class' private members

  \vspace{1em}
  Here is where the {\color{FeebleWeek}protected} access level enters

  \vspace{1em}
  {\color{FeebleWeek}protected} members are

  \begin{itemize}
    \setlength\itemsep{.3em}
    \item {\color{Marty}private} to the outside
    \item {\color{Tropiteal}public} for subclasses
  \end{itemize}

\end{frame}

\begin{frame}[fragile]
  \frametitle{Inheritance and access}

  There is also an access level at the inheritance point

  \vspace{1em}
  \begin{lstlisting}[escapeinside={(*}{*)},basicstyle=\ttfamily]
class Bird : public(*\tikzmark{public}*) Animal {}
  \end{lstlisting}

  \nointerlineskip
  \begin{tikzpicture}[overlay,remember picture]
    \draw[pointy arrow] ([shift={(-3ex,-.5ex)}] public) .. controls +(0,-.75cm) and +(-.25cm,.25cm) .. +(1cm,-.5cm)
      node[below right,anchor=north west,inner sep=4pt,xshift=-1ex] {This one};
  \end{tikzpicture}

\end{frame}

\begin{frame}[fragile]
  \frametitle{Inheritance and access}

  This is the access level the inherited members will have in the new subclass

  \vspace{.5em}
  \begin{onlyenv}<1>
  \begin{lstlisting}[escapeinside={(*}{*)},morekeywords={Animal,Bird}]
class Animal
{
public:
  void eat();
};

class Bird : public Animal {}
  \end{lstlisting}

  \vspace{.5em}
  \lstinline!eat()! is a public method in \,\lstinline[morekeywords={Bird}]!Bird!
  \end{onlyenv}

  \begin{onlyenv}<2>
  \begin{lstlisting}[escapeinside={(*}{*)},morekeywords={Animal,Bird}]
class Animal
{
public:
  void eat();
};

class Bird : private Animal {}
  \end{lstlisting}

  \vspace{.5em}
  \lstinline!eat()! is a private method in \,\lstinline[morekeywords={Bird}]!Bird!
  \end{onlyenv}

\end{frame}

\begin{frame}[fragile]
  \frametitle{Inheritance and access}

  Think of {\color{FeebleWeek}public} inheritance as\\[5pt]
  \hspace{1cm}{\color{Tropiteal}\{child\}} is a {\color{Tropiteal}\{parent\}}

  \vspace{1cm}
  Think of {\color{Marty}private} inheritance as\\[5pt]
  \hspace{1cm}{\color{Tropiteal}\{child\}} is implemented in terms of {\color{Tropiteal}\{parent\}}

  \vspace{1cm}
  Don't think too hard about {\color{Marty}protected} inheritance

\end{frame}

\begin{frame}[fragile]
  \frametitle{Constructors}

  \only<1>{
  Constructors are not inherited {\footnotesize\{not even the copy constructor\}}

  \vspace{1cm}
  But a parent's constructor can (and will) be called
  }

  \begin{onlyenv}<2-3>
  \begin{lstlisting}[escapeinside={(*}{*)}]
class Derived : public Base
{
public:
  Derived() {} (*\tikzmark{default constructor}*)

  Derived(int x)
    : var {x} {} (*\tikzmark{derived int}*)

  Derived(int x)
    : Base {x} {} (*\tikzmark{base int}*)
};
  \end{lstlisting}

  \only<3>{
  \nointerlineskip
  \begin{tikzpicture}[overlay,remember picture]
    \draw[pointy arrow] ([yshift=.35ex] default constructor) -- +(2.5cm,0)
      coordinate (default end)
      node[right,align=left,scale=0.75] {%
        Base default\\constructor called
      };
    \coordinate (derived int begin) at ([yshift=.35ex] derived int);
    \draw[pointy arrow] (derived int begin) -- (derived int begin -| default end)
      node[right,align=left,scale=0.75] {%
        Base default\\constructor called
      };
    \coordinate (base int begin) at ([yshift=.35ex] base int);
    \draw[pointy arrow] (base int begin) -- (base int begin -| default end)
      node[right,align=left,scale=0.75] {%
        Base constructor\\
        accepting an int called
      };
  \end{tikzpicture}
  }

  \end{onlyenv}

  \begin{onlyenv}<4->

    The copy constructor can be defined as

    \vspace{1em}
  \begin{lstlisting}[escapeinside={(*}{*)}]
class Derived : public Base
{
public:
  Derived() {}

  Derived(const Derived & copy)
    : Base {copy}(*\tikzmark{copy}*) {}
};
  \end{lstlisting}

  \nointerlineskip
  \begin{tikzpicture}[overlay,remember picture]
    \draw[pointy arrow] ([shift={(-2ex,-.5ex)}] copy) .. controls +(0,-.5cm) and +(-.5cm,0) .. +(1cm,-1cm)
      node[right,scale=0.75] {%
        We will get back to why this works
      };
  \end{tikzpicture}

  \end{onlyenv}
  
\end{frame}

\begin{frame}[fragile]
  \frametitle{Destructors}

  The destructor is also not inherited

  \vspace{1cm}
  But it will also be called when the instance is deleted\\
  {\footnotesize\{more on this later\}}
  
\end{frame}

\begin{frame}[fragile]
  \frametitle{Inheritance vs composition}

  Inheritance is {\Large\color{FeebleWeek}great},\\[5pt]
but remember that all of it so far can also be accomplished with composition
  
\end{frame}

\begin{frame}[fragile]
  \frametitle{Inheritance vs composition}

  \begin{lstlisting}[morekeywords={Animal,Bird},basicstyle=\fontsize{12pt}{12pt}\selectfont\ttfamily]
class Bird
{
public:
  void eat()
  {
    base.eat();
  }

private:
  Animal base;
};
  \end{lstlisting}

\end{frame}

\begin{frame}[fragile]
  \frametitle{Inheritance vs composition}

  Always consider your options

  \vspace{1cm}
  Composition is used to describe\\[5pt]
  \hspace{1cm}{\color{Tropiteal}\{A\}} is implemented in terms of {\color{Tropiteal}\{B\}}\\[5pt]
  \hspace{1cm}{\color{Tropiteal}\{A\}} has a {\color{Tropiteal}\{B\}}\\[5pt]

\end{frame}

\begin{frame}[fragile]
  \frametitle{Friendship under inheritance}

  {\large \centering
    Friends of my {\color{Marty}\Large\{}\tikzmark{begin} \hspace{2cm} \tikzmark{end}{\color{Marty}\Large\}} are not my
    friends\\[5pt]
    \only<1>{\fontsize{8pt}{8pt}\selectfont (they might be creeps)}
    \only<2>{\fontsize{8pt}{8pt}\selectfont (they are all annoying)}
    \only<3>{\fontsize{8pt}{8pt}\selectfont (they are probably criminals)}
  \par}
  
  \nointerlineskip
  \begin{tikzpicture}[overlay,remember picture]
    \coordinate (middle) at ($(begin) !.5! (end)$);
    \only<1>{
      \node[yshift=.5ex] at (middle) [font=\large] {parents\strut{}};
    }
    \only<2>{
      \node[yshift=.5ex] at (middle) [font=\large] {children\strut{}};
    }
    \only<3>{
      \node[yshift=.5ex] at (middle) [font=\large] {friends\strut{}};
    }
    
  \end{tikzpicture}
  
\end{frame}

\begin{frame}[fragile]
  \frametitle{Multiple inheritance}

  It is also possible to inherit from multiple classes

  \vspace{.3em}
  \begin{center}
  \fontspec{Yanone Kaffeesatz Regular}
  \begin{tikzpicture}[
    class/.style={
      draw, rounded corners,
      line width=.5pt,
      minimum width=3cm,
      text height=1.5ex,
      fill=TealDrop!60!WhiteTrash
    },
    inherit/.style={
      line width=1pt,
      {Triangle[open]}-,
      shorten <=3pt
    }]
    \matrix (scheme) [
      matrix of nodes,
      nodes in empty cells,
      column sep=1.7cm,
      row sep=1.5cm,
      ]
    {
      & & \\
      & & \\
    };
    \node[class] at (scheme-1-1) (bird) {Bird};
    \node[class] at (scheme-1-3) (robot) {Robot};
    \node[class] at (scheme-2-2) {RoboPenguin}
      edge[inherit] (bird)
      edge[inherit] (robot);
    \end{tikzpicture}
  \end{center}

  \begin{overlayarea}{\textwidth}{2cm}

  \begin{onlyenv}<1>
  \begin{lstlisting}[escapeinside={(*}{*)}]
class RoboPenguin : public Bird, public Robot {};
  \end{lstlisting}
  \end{onlyenv}

  \only<2>{\fontspec{Yanone Kaffeesatz Regular}
    {\color{Marty}Note,} the two base classes shouldn't declare the same members
  }
  \end{overlayarea}
  
  
\end{frame}

\section{Polymorphism}

\frame[plain]{\sectionpage}

\begin{frame}[fragile,plain]

  {\Large polymorphism} \hspace{.2cm} {\fontspec{Yanone Kaffeesatz Light}-noun}\\[2pt]
  {\fontspec{Doulos SIL}/ˌpɒlɪˈmɔːfɪz(ə)m/}

  \vspace{.4cm}
  {\fontsize{12pt}{12pt}\selectfont
  From ancient Greek\\[1pt]
  {\fontspec{Roboto}πολύς} {\fontspec{Yanone Kaffeesatz Light}(polús), meaning "many" or "much", and}\\
  {\fontspec{Roboto}μορφή} {\fontspec{Yanone Kaffeesatz Light}(morphé), meaning "form" or "shape"}
  }

  \vspace{.8cm}
  {\fontspec{Yanone Kaffeesatz Light}\footnotesize \raisebox{1ex}{a)}}
  the condition of occurring in several different forms

  \vspace{.6cm}
  {\fontspec{Yanone Kaffeesatz Light}\footnotesize \raisebox{1ex}{b)}}
  the ability to assume different forms or shapes
  
\end{frame}

\begin{frame}[fragile]
  \frametitle{What is polymorphism?}

  A base class can always be replaced with\\a child class instance

  \vspace{.8em}
  If given by-value:\\
  \hspace{.2cm} the derived class will be cast to the base class

  \vspace{.8em}
  If given by-reference:\\
  \hspace{.2cm} the derived class will {\color{FeebleWeek}keep its identity}

\end{frame}

\begin{frame}[fragile]
  \frametitle{Polymorphism - example}

  \begin{onlyenv}<1>
  \begin{center}
  \begin{tikzpicture}[
    frame/.style={
      draw,
      fill=WhiteTrash
    }]

    \node[frame,scale=0.75] (base class) \bgroup%
      \begin{minipage}{10cm}
        \begin{lstlisting}[morekeywords={string}]
class Logger
{
public:
  virtual void log(std::string) = 0;
};
        \end{lstlisting}
      \end{minipage}\egroup;

    \node[fill=WhiteTrash] at (base class.north) [font=\ttfamily,scale=0.6] {Base class};

  \end{tikzpicture}
  \end{center}
  \end{onlyenv}

  \begin{onlyenv}<2>
  \vspace*{-.5cm}
  \begin{center}
  \begin{tikzpicture}[
    frame/.style={
      draw,
      fill=WhiteTrash
    }]
    \node[frame,scale=0.6] (console class) \bgroup%
      \begin{minipage}{10.6cm}
        \begin{lstlisting}[morekeywords={string,override}]
class ConsoleLogger : public Logger
{
public:
  virtual void log(std::string message) override
  {
    std::cout << message << std::endl;
  }
};
        \end{lstlisting}
      \end{minipage}\egroup;

    \node[fill=WhiteTrash] at (console class.north) [font=\ttfamily,scale=0.6] {Derived \#1};

    \node[frame,scale=0.6,below=.5cm of console class] (file class) \bgroup%
      \begin{minipage}{10.6cm}
        \begin{lstlisting}[morekeywords={string,override,ofstream}]
class FileLogger : public Logger
{
private:
  std::ofstream ofs;

public:
  FileLogger(std::string filename)
    : ofs {filename} {}

  virtual void log(std::string message) override
  {
    ofs << message << std::endl;
  }
};
        \end{lstlisting}
      \end{minipage}\egroup;

    \node[fill=WhiteTrash] at (file class.north) [font=\ttfamily,scale=0.6] {Derived \#2};
  \end{tikzpicture}
  \end{center}
  \end{onlyenv}

  \begin{onlyenv}<3>
  \begin{center}
  \begin{tikzpicture}[
    frame/.style={
      draw,
      fill=WhiteTrash
    }]
    \node[frame,scale=0.75] (in use) \bgroup%
      \begin{minipage}{12cm}
        \begin{lstlisting}[morekeywords={Logger,FileLogger,ConsoleLogger}]
void complicatedOperation(/* args */, Logger & logger)
{
  // ...
  logger.log("Stuff happens");
}

int main()
{
  ConsoleLogger console_log;
  complicatedOperation(console_log);

  FileLogger file_log {"program.log"};
  complicatedOperation(file_log);
}
        \end{lstlisting}
      \end{minipage}\egroup;

    \node[fill=WhiteTrash] at (in use.north) [font=\ttfamily,scale=0.6] {Implementation};
  \end{tikzpicture}
  \end{center}
  \end{onlyenv}
\end{frame}

\begin{frame}[fragile]
  \frametitle{The {\color{FeebleWeek}virtual} keyword}

  The {\color{FeebleWeek}virtual} keyword means that the function in question can be overridden

  \vspace{1em}
  If not present the base class' implementation will be called even when we have a by-reference instance

\end{frame}

\begin{frame}[fragile]
  \frametitle{The {\color{FeebleWeek}virtual} keyword}

  \nointerlineskip
  \begin{tikzpicture}[overlay,remember picture]
    \fill[opacity=0.25,FeebleWeek] ([yshift=1.33cm] current page.west) -- ++(0,.95ex) -- ++(\paperwidth,0) -- ++(0,-1ex) -- cycle;
  \end{tikzpicture}

  \begin{tikzpicture}
  \node[anchor=north,inner sep=0pt] {%
  \begin{minipage}{7cm}
  \begin{lstlisting}[escapeinside={(*}{*)},morekeywords={string},basicstyle=\fontsize{7pt}{7pt}\selectfont\ttfamily]
class Base
{
public:
  std::string name() const
  {
    return {"Base"};
  }
};
  \end{lstlisting}
\end{minipage}};
\node[anchor=north,inner sep=0pt] at (5.5cm,0) {%
  \begin{minipage}{7cm}
  \begin{lstlisting}[morekeywords={string},basicstyle=\fontsize{7pt}{7pt}\selectfont\ttfamily]
class Derived : public Base
{
public:
  std::string name() const
  {
    return {"Derived"};
  }
};
  \end{lstlisting}
\end{minipage}};
  \end{tikzpicture}

  \begin{lstlisting}[morekeywords={string,Base,Derived},
    escapeinside={(*}{*)},
    basicstyle=\fontsize{8pt}{8pt}\selectfont\ttfamily]
int main()
{
  Derived dclass;
  Base & bref = dclass;

  dclass.name(); (*\tikzmark{dclass call}*)

  bref.name(); (*\tikzmark{bref call}*)
}
  \end{lstlisting}

  \nointerlineskip
  \begin{tikzpicture}[overlay,remember picture]
    \coordinate (dclass begin) at ([yshift=.3ex] dclass call);
    \draw[pointy arrow,line width=.6pt] (dclass begin) -- +(2cm,0)
      coordinate (dclass end)
      node[right,scale=0.75] {\lstinline!"Derived"!};
    \coordinate (bref begin) at ([yshift=.3ex] bref call);
    \draw[pointy arrow,line width=.6pt] (bref begin) -- (bref begin -| dclass end)
      node[right,scale=0.75] {\lstinline!"Base"!};
  \end{tikzpicture}

\end{frame}

\begin{frame}[fragile]
  \frametitle{The {\color{FeebleWeek}virtual} keyword}

  \nointerlineskip
  \begin{tikzpicture}[overlay,remember picture]
    \fill[opacity=0.25,FeebleWeek] ([yshift=1.33cm] current page.west) -- ++(0,.95ex) -- ++(\paperwidth,0) -- ++(0,-1ex) -- cycle;
  \end{tikzpicture}

  \begin{tikzpicture}
  \node[anchor=north,inner sep=0pt] {%
  \begin{minipage}{7cm}
  \begin{lstlisting}[morekeywords={string},basicstyle=\fontsize{7pt}{7pt}\selectfont\ttfamily]
class Base
{
public:
  virtual std::string name() const
  {
    return {"Base"};
  }
};
  \end{lstlisting}
\end{minipage}};
\node[anchor=north,inner sep=0pt] at (5.5cm,0) {%
  \begin{minipage}{7cm}
  \begin{lstlisting}[morekeywords={string},basicstyle=\fontsize{7pt}{7pt}\selectfont\ttfamily]
class Derived : public Base
{
public:
  virtual std::string name() const
  {
    return {"Derived"};
  }
};
  \end{lstlisting}
\end{minipage}};
    \end{tikzpicture}

  \begin{lstlisting}[morekeywords={string,Base,Derived},
    escapeinside={(*}{*)},
    basicstyle=\fontsize{8pt}{8pt}\selectfont\ttfamily]
int main()
{
  Derived dclass;
  Base & bref = dclass;

  dclass.name(); (*\tikzmark{dclass call}*)

  bref.name(); (*\tikzmark{bref call}*)
}
  \end{lstlisting}

  \nointerlineskip
  \begin{tikzpicture}[overlay,remember picture]
    \coordinate (dclass begin) at ([yshift=.3ex] dclass call);
    \draw[pointy arrow,line width=.6pt] (dclass begin) -- +(2cm,0)
      coordinate (dclass end)
      node[right,scale=0.75] {\lstinline!"Derived"!};
    \coordinate (bref begin) at ([yshift=.3ex] bref call);
    \draw[pointy arrow,line width=.6pt] (bref begin) -- (bref begin -| dclass end)
      node[right,scale=0.75] {\lstinline!"Derived"!};
  \end{tikzpicture}

\end{frame}

\begin{frame}[fragile]
  \frametitle{Overriding functions}

  Using polymorphism and overriding functions is a great way of writing more generalised code
  and decrease code duplication

  \vspace{1cm}
  Humans are awesome at abstraction, take advantage of that for your programming
  
\end{frame}

\begin{frame}[fragile]
  \frametitle{The {\color{FeebleWeek}override} keyword}

  {\color{FeebleWeek}virtual}\\[2pt]
  tell the base class that a derived class can override the function

  \vspace*{-.5cm}
  \begin{center}
    \begin{tikzpicture}
      \draw[{Stealth}-{Stealth},use as bounding box,line width=1.5pt] (0,0) -- (0,1.5cm);
      \node at (.2cm,.75cm) [anchor=west,scale=0.75,Marty] {Symmetry};
    \end{tikzpicture}
  \end{center}

  \vspace*{-.5cm}
  {\color{FeebleWeek}override}\\[2pt]
  tell the derived class that the function overrides a function from the base class

  \CPPEleven
  
\end{frame}

\begin{frame}[fragile]
  \frametitle{The {\color{FeebleWeek}override} keyword}

  \begin{lstlisting}[escapeinside={(*}{*)},morekeywords={string}]
class Base
{
public:
  virtual std::string name() const (*\tikzmark{base name}*)
  {
    return {"Base"};
  }
};

class Derived : public Base
{
public:
  virtual std::string name() (*\tikzmark{derived name}*)
  {
    return {"Derived"};
  }
};
  \end{lstlisting}

  \only<2>{
  \nointerlineskip
  \begin{tikzpicture}[overlay,remember picture]
    \draw[-{Stealth[bend]},line width=.7pt,shorten >=.4cm]
      ([yshift=.3ex] derived name) .. controls +(1cm,.5cm) and +(1cm,-.3cm) .. ([xshift=-5ex] base name)
      node[midway,right,scale=0.75] {This will {\color{Marty}not} override};
  \end{tikzpicture}
  }

  \CPPEleven
  
\end{frame}

\begin{frame}[fragile]
  \frametitle{The {\color{FeebleWeek}override} keyword}

  \begin{lstlisting}[escapeinside={(*}{*)},morekeywords={string,override}]
class Base
{
public:
  virtual std::string name() const
  {
    return {"Base"};
  }
};

class Derived : public Base
{
public:
  virtual std::string name()(*\tikzmark{override}*) override
  {
    return {"Derived"};
  }
};
  \end{lstlisting}

  \nointerlineskip
  \begin{tikzpicture}[overlay,remember picture]
    \draw[{Stealth}-,line width=0.75] ([shift={(-.5ex,-.5ex)}] override) -- +(.5cm,-.5cm)
      node [below,xshift=1cm,scale=0.75,Marty] {Compile error};
  \end{tikzpicture}

  \CPPEleven
  
\end{frame}

\begin{frame}[fragile]
  \frametitle{The {\color{FeebleWeek}override} keyword}

  \begin{lstlisting}[escapeinside={(*}{*)},morekeywords={string,override}]
class Base
{
public:
  virtual std::string name() const
  {
    return {"Base"};
  }
};

class Derived : public Base
{
public:
  virtual std::string name()(*\tikzmark{override}*) const override
  {
    return {"Derived"};
  }
};
  \end{lstlisting}

  \nointerlineskip
  \begin{tikzpicture}[overlay,remember picture]
    \draw[{Stealth}-,line width=0.75] ([shift={(-.5ex,-.5ex)}] override) -- +(.5cm,-.5cm)
      node [below,xshift=.35cm,scale=0.75,FeebleWeek] {OK};
  \end{tikzpicture}

  \CPPEleven
  
\end{frame}

\begin{frame}[fragile]
  \frametitle{The {\color{FeebleWeek}final} keyword}

  The {\color{FeebleWeek}final} keyword is used to {\color{Marty}disable} inheritance

  \vspace{.2cm}
  It can be used on both classes and member functions

  \vspace{.2cm}
  \begin{lstlisting}[escapeinside={(*}{*)},morekeywords={final}]
class Base
{
  virtual void finalMethod() final;
};

class Derived final : public Base
{
  virtual void finalMethod(); (*\tikzmark{final method}*)
};

class Child : public Derived {}; (*\tikzmark{final class}*)
  \end{lstlisting}

  \nointerlineskip
  \begin{tikzpicture}[overlay,remember picture]
    \coordinate (bend) at ([xshift=2.5cm] current page.north);
    \coordinate (method begin) at ([yshift=.3ex] final method);
    \coordinate (class begin) at ([yshift=.3ex] final class);

    \coordinate (method bend) at (method begin -| bend);
    \coordinate (class bend) at (class begin -| bend);
    \coordinate (bend mid) at ($(method bend) !.5! (class bend)$);

    \draw[{Stealth}-,rounded corners,line width=.7] (class begin) -- (class bend) -- (bend mid) -- +(.5cm,0);
    \draw[{Stealth}-,rounded corners,line width=.7] (method begin) -- (method bend) -- (bend mid) -- +(.5cm,0)
      node[right,scale=0.75,Marty] {Compile error};
    
  \end{tikzpicture}
  
  \CPPEleven
  
\end{frame}

\begin{frame}[fragile]
  \frametitle{Pure virtual functions}

  It is possible to create an {\color{Tropiteal}abstract} class, which has virtual functions with no implementation

  
  \begin{onlyenv}<2>
  \vspace{.5cm}
  \begin{lstlisting}[escapeinside={(*}{*)}]
class Shape
{
public:
  virtual double area() const = 0;
};
  \end{lstlisting}
    
  \vspace{.5cm}
  {\fontspec{Yanone Kaffeesatz Regular}
    Every shape has an area, but the area of a generic shape is undefinable
  }
  \end{onlyenv}
  
\end{frame}

\begin{frame}[fragile]
  \frametitle{Interfaces}

  The sum of methods a class has is called its interface

  \vspace{1cm}
  It is common to define the topmost class of a hierarchy as a pure interface, or a pure abstract class
  
\end{frame}

\begin{frame}[fragile]
  \frametitle{Destructors again}

  When making use of polymorphism it is important to mark all destructors as virtual
  
  \vspace{.5cm}
  This way a derived class can be destructed from a base reference

  \begin{overlayarea}{\textwidth}{3cm}
  \begin{onlyenv}<1>
  \begin{lstlisting}[escapeinside={(*}{*)},basicstyle=\ttfamily\fontsize{8pt}{8pt}\selectfont]
class Base
{
public:
  ~Base() {...}
};

class Derived : public Base {...};

std::unique_ptr<Base> base_ptr = std::make_unique<Derived>();(*\tikzmark{non virtual}*)
  \end{lstlisting}

  \nointerlineskip
  \begin{tikzpicture}[overlay,remember picture]
    \draw[pointy arrow] ([shift={(-2cm,2ex)}] non virtual) -- +(0,1cm)
      node[above,scale=0.75,align=center] {%
        {\ttfamily\color{FeebleWeek} Derived}'s destructor\\
        will {\color{Marty}not} be called
      };
  \end{tikzpicture}
  \end{onlyenv}
  \begin{onlyenv}<2>
  \begin{lstlisting}[escapeinside={(*}{*)},basicstyle=\ttfamily\fontsize{8pt}{8pt}\selectfont]
class Base
{
public:
  virtual ~Base() {...}
};

class Derived : public Base {...};

std::unique_ptr<Base> base_ptr = std::make_unique<Derived>();(*\tikzmark{non virtual}*)
  \end{lstlisting}

  \nointerlineskip
  \begin{tikzpicture}[overlay,remember picture]
    \draw[pointy arrow] ([shift={(-2cm,2ex)}] non virtual) -- +(0,1cm)
      node[above,scale=0.75,align=center] {%
        {\ttfamily\color{FeebleWeek} Derived}'s destructor\\
        will {\color{FeebleWeek}now} be called
      };
  \end{tikzpicture}
  \end{onlyenv}
  \end{overlayarea}

\end{frame}

\begin{frame}[fragile]
  \frametitle{Casting}

  Casting from Derived to Base is called upcasting\\[5pt]
  This is always allowed for public inheritance

  \vspace{1cm}
  Casting from Base to Derived is called downcasting\\[5pt]
  Only allowed using an explicit cast operation
  
\end{frame}

\begin{frame}[fragile]
  \frametitle{Dynamic casting}

  For downcasting we need a dynamic cast

  \vspace{.5em}
  \begin{lstlisting}[morekeywords={Base,Derived}]
int main()
{
  Derived derived_object;
  Base & bref = derived object;

  Derived & dref = dynamic_cast<Derived&>(bref);
}
  \end{lstlisting}

  \vspace{.5em}
  Dynamic cast checks at runtime whether the downcast is actually allowed
  
\end{frame}

\begin{frame}[fragile]
  \frametitle{Slicing}

  Passing a derived class by value results in slicing

  \vspace{1.5cm}
  Upcasting happens automatically,\tikzmark{complain} and all the class specialisations are stripped away

  \nointerlineskip
  \begin{tikzpicture}[overlay,remember picture]
    \node[anchor=south] (a) at ([yshift=2ex] complain) [scale=0.4,Marty,align=center] {without the\\compiler complaining};
    \draw[Marty] ([shift={(0,0)}] a.south west) .. controls +(.6cm,-.1cm) and +(0,.3cm) .. ([yshift=1ex] complain);
    \draw[Marty] ([shift={(0,0)}] a.south east) .. controls +(-.6cm,-.1cm) and +(0,.3cm) .. ([yshift=1ex] complain);
  \end{tikzpicture}

  
\end{frame}

\begin{frame}[fragile]
  \frametitle{{\color{Marty}The deadly diamond of death}}

  \begin{center}
  \fontspec{Yanone Kaffeesatz Regular}
  \begin{tikzpicture}[
    class/.style={
      draw, rounded corners,
      line width=.5pt,
      minimum width=2.5cm,
      text height=1.5ex,
      fill=Marty!60!WhiteTrash
    },
    inherit/.style={
      line width=1pt,
      {Triangle[open]}-,
      shorten <=3pt
    }]
    \matrix (scheme) [
      matrix of nodes,
      nodes in empty cells,
      column sep=2.5cm,
      row sep=2cm,
      ]
    {
      & & \\
      & & \\
      & & \\
    };
    \node[class] at (scheme-1-2) (file) {File};
    \node[class] at (scheme-2-1) (ifile) {IFile}
      edge[inherit] (file);
    \node[class] at (scheme-2-3) (ofile) {OFile}
      edge[inherit] (file);
    \node at (scheme-2-2) {\includegraphics[width=2cm]{Figures/toxic.pdf}};
    \node[class] at (scheme-3-2) {IOFile}
      edge[inherit] (ifile)
      edge[inherit] (ofile);
    \end{tikzpicture}
  \end{center}

\end{frame}

\begin{frame}[fragile]
  \frametitle{{\color{Marty}The deadly diamond of death}}

  There are multiple solutions available

  \vspace{.5em}
  \begin{itemize}
    \setlength\itemsep{.5em}
    \item Using {\color{FeebleWeek}virtual} inheritance\\
      \lstinline[basicstyle=\ttfamily\footnotesize]!class IFile : public virtual File {};!
    \item Have the base class be pure virtual
    \item Rewrite your class structure
  \end{itemize}

\end{frame}

\section{Templates}

\frame[plain]{\sectionpage}

\begin{frame}[fragile]
  \frametitle{Motivation}

  Say that you have overloaded a function to work for several types

  \vspace{.5em}
  \begin{lstlisting}[morekeywords={Logger}]
void valueLog(int val, Logger & logger)
{
  logger << "[value]: " << val << std::endl;
}

void valueLog(double val, Logger & logger)
{
  logger << "[value]: " << val << std::endl;
}

// ... 
  \end{lstlisting}
  
\end{frame}

\begin{frame}[fragile]
  \frametitle{Motivation}

  If the function bodies are identical
  {\fontsize{8pt}{8pt}\selectfont \{or could be made identical\}}\\
  this is {\color{Marty}code duplication}, and should be avoided

  \vspace{1cm}
  Can only be done with {\color{FeebleWeek}templates}

\end{frame}

\begin{frame}[fragile]
  \frametitle{Function templates}

  \begin{lstlisting}[escapeinside={(*}{*)},morekeywords={ValueType,Logger}]
template <typename(*\tikzmark{typename}*) ValueType>
void valueLog(ValueType val, Logger & logger)
{
  logger << "[value]: " << val << std::endl;
}
  \end{lstlisting}

  \vspace{1em}
  Compiles for every type that can be streamed to a Logger
  {\fontsize{8pt}{8pt}\selectfont\{which again could be a template function\}}

  \nointerlineskip
  \begin{tikzpicture}[overlay,remember picture]
    \only<1>{
    \draw[pointy arrow] ([shift={(-3ex,1.5ex)}] typename) .. controls +(0,.5cm) and +(-.5cm,0) .. +(1cm,.5cm)
      node[right,scale=0.75] {Can be replaced with \lstinline!class!};
    }
    \only<2>{
    \draw[pointy arrow] ([shift={(3ex,1.5ex)}] typename) .. controls +(0,.5cm) and +(-.5cm,0) .. +(1cm,.5cm)
      node[right,scale=0.75] {Name of the undefined type};
    }
  \end{tikzpicture}
  
\end{frame}

\begin{frame}[fragile]
  \frametitle{Generic programming}

  \only<1>{
  Using templates is another great tool for abstraction

  \vspace{1em}
  You can make assumptions on your template types,
  and the assumptions will be checked by the compiler for every instance
  }

  \begin{onlyenv}<2->
  \begin{lstlisting}[escapeinside={(*}{*)},morekeywords={Container}]
template <typename Container>
void print(const Container & container)
{
  std::cout << "{";
  for (auto i = 0; i < container.size(); ++i)
    std::cout << container[i] << ",";
    
  std::cout << "}";
}
  \end{lstlisting}

  \vspace{.5em}
  Only compiles if

  \begin{overlayarea}{\textwidth}{2cm}
  \only<2>{
  \begin{enumerate}\fontsize{12pt}{12pt}\selectfont\setcounter{enumi}{0}
    \item \lstinline[morekeywords={Container}]!Container! type has a \lstinline!size()! member function
  \end{enumerate}
  }
  \only<3>{
  \begin{enumerate}\fontsize{12pt}{12pt}\selectfont\setcounter{enumi}{1}
    \item Type of \lstinline!container.size()! is comparable to \lstinline!int!
  \end{enumerate}
  }
  \only<4>{
  \begin{enumerate}\fontsize{12pt}{12pt}\selectfont\setcounter{enumi}{2}
    \item \lstinline[morekeywords={Container}]!Container! has overloaded the access operator {\color{sblue}\ttfamily []}
  \end{enumerate}
  }
  \only<5>{
  \begin{enumerate}\fontsize{12pt}{12pt}\selectfont\setcounter{enumi}{3}
    \item The type returned by \lstinline!container[i]! can be streamed to \lstinline!std::cout!
  \end{enumerate}
  }
  \only<6>{
  \begin{enumerate}\fontsize{12pt}{12pt}\selectfont\setcounter{enumi}{4}
    \item Both member functions \lstinline!size()! and \lstinline![i]! are const
  \end{enumerate}
  }
  \end{overlayarea}
  \end{onlyenv}
\end{frame}

\begin{frame}[fragile]
  \frametitle{Return value templates}

  Can make functions that only differ in return type
  
  \vspace{.3em}
  \begin{lstlisting}[morekeywords={Type}]
template <typename Type>
Type null()
{
  return static_cast<Type>(0);
}
  \end{lstlisting}

  \vspace{.5em}
  But you have to tell the compiler which function to call

  \vspace{.3em}
  \begin{lstlisting}
auto zero = null<int>();
auto uzero = null<unsigned>();
auto first_char = null<char>();
  \end{lstlisting}
  
\end{frame}

\begin{frame}[fragile]
  \frametitle{Template specialisations}

  Changing the template function for a specific type is called template specialisation
  
  \vspace{.5em}
  \begin{columns}
  \column{.4\textwidth}
  \begin{lstlisting}[morekeywords={string}]
template <>
char null()
{
  return 'a';
}
  \end{lstlisting}
  \column{.4\textwidth}
  \begin{lstlisting}[morekeywords={string}]
template <>
std::string null()
{
  return {""};
}
  \end{lstlisting}
  \end{columns}
  
\end{frame}

\begin{frame}[fragile]
  \frametitle{Disabling specialisations}

  Similarly to how one can disable the constructors and assignment operators in class declaration,
  one can disable templates for specific types

  \vspace{.5em}
  \begin{lstlisting}[morekeywords={ostream}]
template <>
void* null() = delete;

template <>
std::ostream null() = delete;
  \end{lstlisting}

  \CPPEleven
  
\end{frame}

\begin{frame}[fragile]
  \frametitle{Class templates}

  We can also make template classes

  \vspace{.5em}
  \begin{lstlisting}[morekeywords={ValueType}]
template <typename ValueType>
class Rational
{
private:
  ValueType num, den;
};

  \end{lstlisting}
  
  \begin{lstlisting}[escapeinside={(*}{*)},morekeywords={complex,Rational}]
Rational<double> frac;
Rational< std::complex<double> >(*\tikzmark{stream op}*) compl_frac;
  \end{lstlisting}

  \nointerlineskip
  \begin{tikzpicture}[overlay,remember picture]
    \draw[pointy arrow] ([shift={(-1ex,-.5ex)}] stream op) -- +(0,-.4cm)
      node[below,scale=.75,xshift=-2.5cm] {{\color{Marty}Note:} please don't accidentally write the stream operator};
  \end{tikzpicture}
  
\end{frame}

\begin{frame}[fragile]
  \frametitle{Partial specialisation}

  Class templates can also have {\color{FeebleWeek}partial} template specialisations

  \vspace{.5em}
  \begin{lstlisting}[escapeinside={(*}{*)}]
template <typename Type>
class Foo {};

template <typename Type>
class Foo<Type*> {}; (*\tikzmark{pointer spec}*)
  \end{lstlisting}

  \vspace{.5em}
  Can't have that for functions\\
  {\fontspec{Yanone Kaffeesatz Regular}
    (but you could always make function objects)
  }

  \nointerlineskip
  \begin{tikzpicture}[overlay,remember picture]
    \draw[pointy arrow] ([shift={(.5cm,.3ex)}] pointer spec) -- +(1.5cm,0)
      node[right,scale=0.75] {Specialisation for pointer types};
  \end{tikzpicture}
  
\end{frame}

\begin{frame}[fragile]
  \frametitle{Use case: type traits}

  \begin{lstlisting}[escapeinside={(*}{*)},morekeywords={Type}]
template <typename Type>
class type_trait
{
public:
  using value_type = typename Type::value_type;(*\tikzmark{dependent name}*)
};

template <typename Type>
class type_trait<Type*>
{
public:
  using value_type = Type;
};
  \end{lstlisting}

  \only<2>{
  \nointerlineskip
  \begin{tikzpicture}[overlay,remember picture]
    \draw[pointy arrow] ([shift={(-3.5ex,-.75ex)}] dependent name) -- +(0,-.5cm)
      node[below,scale=0.75] {A {\color{Marty}dependent} name};
  \end{tikzpicture}
  }

\end{frame}

\begin{frame}[fragile]
  \frametitle{Separating declaration and definition}
  \fontspec{Yanone Kaffeesatz Regular}

  Separating template functions and classes into separate header and source files
  are unfortunately not that easy, and doing it the naive way will most likely result 
  in linking errors

  \vspace{1cm}
  For a solution either have a look at the C++ Super FAQ {\footnotesize (link on the homepage)} or
  write all template functions as inline functions {\footnotesize (which is what libraries do)}
  
\end{frame}

\begin{frame}[fragile]
  \frametitle{Template Meta Programming}

  It was discovered {\fontsize{8pt}{8pt}\selectfont\color{Marty}(by accident)} that the C++ template system and pattern lookup
  is itself Turing complete

  \vspace{1cm}
  It is possible to program with the template system
  
\end{frame}

\begin{frame}[fragile]
  \frametitle{Compile time factorials}

  \begin{lstlisting}[escapeinside={(*}{*)},basicstyle=\ttfamily\fontsize{8pt}{8pt}\selectfont,morekeywords={constexpr}]
template <unsigned N>
struct Factorial
{
  static constexpr unsigned value {N*Factorial<N-1>::value};
};

template <>
struct Factorial
{
  static constexpr unsigned value {0};
};
  \end{lstlisting}

  \begin{lstlisting}[escapeinside={(*}{*)},basicstyle=\ttfamily\fontsize{8pt}{8pt}\selectfont,morekeywords={Factorial}]
int main()
{
  std::cout << (*\tikzmark{runtime constant begin}*)Factorial<10>::value(*\tikzmark{runtime constant end}*) << std::endl;
}
  \end{lstlisting}

  \only<2>{
  \nointerlineskip
  \begin{tikzpicture}[overlay,remember picture]
      \draw[transform canvas={yshift=-.75ex},decorate,decoration={brace,amplitude=5pt,mirror},line width=1pt]
        (runtime constant begin) -- (runtime constant end)
        node [midway, below=.5ex, scale=0.75] {Runtime constant};
  \end{tikzpicture}
  }

\end{frame}

\section{Programming Practices}

\frame[plain]{\sectionpage}

\begin{frame}[fragile,plain]

  \vspace{2.5cm}
  \centering
  {\LARGE Learn to {\color{Marty}love} the compiler}\\
  \tikz[baseline={(heart.center)}] \node[rotate=90,Marty,scale=4] (heart) {<3};
  
\end{frame}

\begin{frame}[fragile]
  \frametitle{Warnings are also great}

  \begin{lstlisting}[language=bash,alsoletter={+},deletekeywords={dirs},basicstyle=\ttfamily\fontsize{8pt}{8pt}\selectfont]
g++ -pedantic -Wall -Wextra -Wcast-align -Wcast-qual
  -Wctor-dtor-privacy -Wdisabled-optimization -Wformat=2
  -Winit-self -Wlogical-op -Wmissing-declarations
  -Wmissing-include-dirs -Wnoexcept -Wold-style-cast
  -Woverloaded-virtual -Wredundant-decls -Wshadow
  -Wsign-conversion -Wsign-promo -Wstrict-null-sentinel
  -Wstrict-overflow=5 -Wswitch-default -Wundef -Werror
  -Wno-unused -o program code.cpp
  \end{lstlisting}

  \vspace{.5em}
  {\fontspec{Yanone Kaffeesatz Regular}
    Everything that helps you avoid runtime debugging is invaluable
  }
  
\end{frame}

\begin{frame}
  \frametitle{Good Programming Practices}

  \begin{itemize}
    \setlength\itemsep{0.5em}
    \item Always make {\color{FeebleWeek}virtual} destructors for classes meant for inheritance
    \item Use templates to avoid code duplication
    \item Abstractions is a tool for the educated mind
  \end{itemize}

  \begin{center}
  \begin{tikzpicture}
    \node [use as bounding box] (quote) {%
      \begin{minipage}{0.85\textwidth} \centering
        We can solve any problem by introducing an extra layer of indirection
      \end{minipage}
    };
    \node[yshift=.2cm] at (quote.north) [font=\fontspec{Yanone Kaffeesatz Regular},Tropiteal] {%
      Fundamental theorem of software engineering};
  \end{tikzpicture}
  \end{center}
  
\end{frame}

\section{Recap}

\frame[plain]{\sectionpage}

\begin{frame}
  \frametitle{Recap Day 4}

  \begin{itemize}
    \setlength\itemsep{0.75em}
    \item You can create new classes based on existing ones through inheritance
    \item Encapsulation is very strict, friendship is not transitive
    \item Public inheritance is like an "is-a" relationship
  \end{itemize}
  
\end{frame}

\begin{frame}
  \frametitle{Recap Day 4}

  \begin{itemize}
    \setlength\itemsep{0.75em}
    \item Polymorphism is when you let derived classes act as its base class
    \item Templates can be used to write type independent, generic programs
    \item Template metaprogramming is awesome
  \end{itemize}
  
\end{frame}

\end{document}
