\documentclass[14pt,a4paper,dvipsnames,usenames]{beamer}
\usepackage[T1]{fontenc}
\usepackage{url}
\usepackage{xcolor}
\usepackage{amsmath,varwidth}
\usepackage{tikz,tikz-uml,xparse}
\usepackage{listings,algpseudocode}
\usepackage[quiet]{mathspec}
\usepackage{textcomp}

\setmainfont[
  ItalicFont={Yanone Kaffeesatz Light},
  Scale=1.3,
  LetterSpace=2.0
]{Yanone Kaffeesatz Bold}

\setmonofont{Hack}

% --------------------- %
% Beamer theme settings %
% --------------------- %

\usetheme{CodeCourse}
\setbeamertemplate{navigation symbols}{} %Remove beamer navigation symbols
\setbeamertemplate{bibliography item}{\insertbiblabel}

% ---------------------------- %
% Colour definitions for theme %
% ---------------------------- %

\definecolor{Tropiteal}{RGB}{0,168,198}
\definecolor{TealDrop}{RGB}{64,192,203}
\definecolor{WhiteTrash}{RGB}{249,242,231}
\definecolor{AtomicBikini}{RGB}{174,226,57}
\definecolor{FeebleWeek}{RGB}{143,190,0}
\definecolor{ICantExpress}{RGB}{28,20,13}
\definecolor{Love}{RGB}{209,67,52}
\definecolor{UnrealFoodPills}{RGB}{250,105,0}
\definecolor{Marty}{RGB}{250,42,0}
\definecolor{ConradGold}{RGB}{255,221,0}

\definecolor{sbase03}{HTML}{002B36}
\definecolor{sbase02}{HTML}{073642}
\definecolor{sbase01}{HTML}{586E75}
\definecolor{sbase00}{HTML}{657B83}
\definecolor{sbase0}{HTML}{839496}
\definecolor{sbase1}{HTML}{93A1A1}
\definecolor{sbase2}{HTML}{EEE8D5}
\definecolor{sbase3}{HTML}{FDF6E3}
\definecolor{syellow}{HTML}{B58900}
\definecolor{sorange}{HTML}{CB4B16}
\definecolor{sred}{HTML}{DC322F}
\definecolor{smagenta}{HTML}{D33682}
\definecolor{sviolet}{HTML}{6C71C4}
\definecolor{sblue}{HTML}{268BD2}
\definecolor{scyan}{HTML}{2AA198}
\definecolor{sgreen}{HTML}{859900}

% -------------------------------------------------- %
% Beamer notes to be itemized rather than enumerated %
% -------------------------------------------------- %

\makeatletter
\def\beamer@setupnote{%
  \gdef\beamer@notesactions{%
    \beamer@outsideframenote{%
      \beamer@atbeginnote%
      \beamer@notes%
      \ifx\beamer@noteitems\@empty\else
      \begin{itemize}\itemsep=0pt\parskip=0pt%
        \beamer@noteitems%
      \end{itemize}%
      \fi%
      \beamer@atendnote%
    }%
    \gdef\beamer@notesactions{}%
  }
}
\makeatother

% -------------------------- %
% Standard LaTeX definitions %
% -------------------------- %

\DeclareGraphicsExtensions{.pdf}
\renewcommand\textbullet{\ensuremath{\bullet}}

% ----------------------------------- %
% Various tikz styles and definitions %
% ----------------------------------- %

\usetikzlibrary{calc,intersections,positioning,matrix,chains,scopes}
\usetikzlibrary{decorations.pathmorphing,decorations.pathreplacing,shapes}
\usetikzlibrary{decorations.text}
\usetikzlibrary{arrows.meta,bending}

\tikzset{
  table/.style={
    matrix of nodes,
    row sep=-\pgflinewidth,
    column sep=-\pgflinewidth,
    nodes={rectangle,draw,text width=6ex,align=center,scale=0.8},
    text depth=0.1ex,
    text height=1.25ex,
    nodes in empty cells
  },
  pointy arrow/.style={
    {Stealth[flex]}-,
    line width=1pt
  },
  simple uml node/.style={
    draw, rectangle,
    line width=1pt,
    rounded corners,
    fill=TealDrop,
    text width=3cm,
    align=center,
    font=\footnotesize
  },
  uml arrow/.style={
    -{Stealth},
    line width=1pt
  }
}
% ---------------------------------- %
% My default listings C++ code style %
% ---------------------------------- %

\lstset{%
  language=C++, basicstyle=\ttfamily\lst@ifdisplaystyle\footnotesize\fi,
  keywordstyle=\color{sgreen}, identifierstyle=\color{sblue}, 
  commentstyle=\color{sbase1}, stringstyle=\color{sorange},
  numberstyle=\color{sviolet}, showstringspaces=false,  
  breaklines=true,
  tabsize=5,
  morekeywords={%
    MyClass,unique_ptr,shared_ptr,weak_ptr,auto_ptr,
    make_unique, make_shared, default_delete,
    dynamic_pointer_cast, const_pointer_cast,
    scoped_ptr,QSharedPointer,QScopedPointer,QWeakPointer,
    ftring, SmartPtr, Shape, Circle, ShapeFactory, CircleFactory,
    Derived, Base
  }
}

\lstalias[]{gnumake}[gnu]{make}

\def\inline{\lstinline[basicstyle=\ttfamily\normalsize]}

% ---------------------------------------------------------------------------------------------------- %
% Various command for inline code instead of lstinline so that I don't have to update the keyword list %
% ---------------------------------------------------------------------------------------------------- %

\newcommand{\object}[1]{{\ttfamily \color{OliveGreen}#1}}
\newcommand{\function}[1]{{\ttfamily \color{RoyalBlue}#1}}
\newcommand{\std}[1]{{\ttfamily {\color{RoyalBlue} std::}{\color{OliveGreen}#1}}}
\newcommand{\boost}[1]{{\ttfamily {\color{RoyalBlue} boost::}{\color{OliveGreen}#1}}}

\newcommand{\uniqueptr}{\std{unique\_ptr}}
\newcommand{\sharedptr}{\std{shared\_ptr}}
\newcommand{\weakptr}{\std{weak\_ptr}}
\newcommand{\autoptr}{\std{auto\_ptr}}

\usepackage{array}

% ------------------------------------------------ %
% tikz functions to place markers on beamer slides %
% ------------------------------------------------ %

\makeatletter
\tikzset{%
  remember picture with id/.style={%
    remember picture,
    overlay,
    save picture id=#1,
  },
  save picture id/.code={%
    \edef\pgf@temp{#1}%
    \immediate\write\pgfutil@auxout{%
      \noexpand\savepointas{\pgf@temp}{\pgfpictureid}}%
  },
  if picture id/.code args={#1#2#3}{%
    \@ifundefined{save@pt@#1}{%
      \pgfkeysalso{#3}%
    }{
      \pgfkeysalso{#2}%
    }
  }
}

\def\savepointas#1#2{%
  \expandafter\gdef\csname save@pt@#1\endcsname{#2}%
}

\def\tmk@labeldef#1,#2\@nil{%
  \def\tmk@label{#1}%
  \def\tmk@def{#2}%
}

\tikzdeclarecoordinatesystem{pic}{%
  \pgfutil@in@,{#1}%
  \ifpgfutil@in@%
    \tmk@labeldef#1\@nil
  \else
    \tmk@labeldef#1,(0pt,0pt)\@nil
  \fi
  \@ifundefined{save@pt@\tmk@label}{%
    \tikz@scan@one@point\pgfutil@firstofone\tmk@def
  }{%
  \pgfsys@getposition{\csname save@pt@\tmk@label\endcsname}\save@orig@pic%
  \pgfsys@getposition{\pgfpictureid}\save@this@pic%
  \pgf@process{\pgfpointorigin\save@this@pic}%
  \pgf@xa=\pgf@x
  \pgf@ya=\pgf@y
  \pgf@process{\pgfpointorigin\save@orig@pic}%
  \advance\pgf@x by -\pgf@xa
  \advance\pgf@y by -\pgf@ya
  }%
}
\newcommand\tikzsupermark[2][]{%
\tikz[remember picture with id=#2] #1;}
\makeatother

\newcommand{\MarkText}[3][]{%
\begin{tikzpicture}[overlay,remember picture]%
  \path (pic cs:#2,{(0,\paperheight)}) +(0,.7\baselineskip) coordinate (a);
  \path (pic cs:#3,{(0,-\paperheight)}) +(0,-.3\baselineskip) coordinate (b);
  \draw [ultra thick,
    if picture id={#2}{red}{line width=1cm,green,opacity=.5},
    #1]
  (a) -- ($(b)!(a)!($(b)+(1,0)$)$);
\end{tikzpicture}%
}%

\newcommand{\tikzmark}[1]{\tikz[overlay,remember picture] \coordinate (#1);}

% -------------- %
% Special frames %
% -------------- %

\newcommand{\LiveFrame}{%
\begin{frame}[plain]
  \nointerlineskip
  \begin{tikzpicture}[overlay,remember picture]
    \fill[FeebleWeek] (current page.north east) rectangle (current page.south west);
    \node at ([yshift=-.25em] current page.center) [font=\LARGE, WhiteTrash] {Live Example};
  \end{tikzpicture}
\end{frame}
}

\newcommand{\CPPEleven}{%
\nointerlineskip
\begin{tikzpicture}[overlay,remember picture]
  \node[anchor=south east] at ( current page.south east) [color=Tropiteal] {\{C++11\}};
\end{tikzpicture}
}

\newcommand{\CPPFourteen}{%
\nointerlineskip
\begin{tikzpicture}[overlay,remember picture]
  \node[anchor=south east] at ( current page.south east) [color=Tropiteal] {\{C++14\}};
\end{tikzpicture}
}

\newcommand{\BadPractice}{%
\nointerlineskip
\begin{tikzpicture}[overlay,remember picture]
  \node[anchor=south east] at ( current page.south east) [color=Marty] {Bad Practice};
\end{tikzpicture}
}

\newcommand{\ClassLine}{%
\nointerlineskip
\begin{tikzpicture}[overlay,remember picture]
  \draw[transform canvas={yshift=-1.7cm},thin] (current page.north west) -- (current page.north east);
\end{tikzpicture}
}

\newcommand{\Library}[2][0pt]{%
\nointerlineskip
\begin{tikzpicture}[overlay,remember picture]
  \node[anchor=south east] at ([xshift=#1] current page.south east) [color=Marty,scale=0.9]
    [font=\fontspec{Yanone Kaffeesatz Regular}] {\#include<#2>};
\end{tikzpicture}
}

% ----------------------------------------- %
% Various Commands for inline codecolouring %
% ----------------------------------------- %

\newcommand{\inobj}[1]{%
  {\color{Tropiteal}#1}}

\newcommand{\inkey}[1]{%
  {\color{FeebleWeek}#1}}

\newcommand{\innorm}[1]{%
  {\color{ICantExpress}#1}}

\newcommand{\std}[1]{%
  \inobj{std}\innorm{::}\inkey{#1}}

% ----------------------------------------------------- %
% Various tikz based functions to style up the document %
% ----------------------------------------------------- %

\newcommand{\underlinemark}[3][line width=1.5pt, FeebleWeek]{%
\draw[transform canvas={yshift=-.15cm}, #1]
  ([xshift=-1pt] #2) -- ([xshift=2pt] #3);
}

\newcommand{\StrikeThrough}[2][Marty]{%
  \tikz[baseline=(strokethrough.base)]{%
    \node[inner sep=0pt, outer sep=0pt] (strokethrough) {\strut{}#2};
    \draw[line width=2pt,#1] ([xshift=-3pt] strokethrough.west) -- ([xshift=3pt] strokethrough.east);
  }
}

\newcommand{\Highlight}[3][opacity=0.4,AtomicBikini]{%
\nointerlineskip
\begin{tikzpicture}[overlay,remember picture]
  \path (pic cs:#2,{(0,0)}) coordinate (a);
  \path (pic cs:#3,{(0,0)}) coordinate (b);
  \fill[#1] (a -| current page.north west) -- ++(0,1.4ex) -- ++(\paperwidth,0)
  -- (b -| current page.north east) -- ++(0,-.4ex) -- ++(-\paperwidth,0) -- cycle;
\end{tikzpicture}
}

\newcommand{\Dimtext}[2]%
{
  { \transparent{0.7}
  \begin{tikzpicture}[overlay, remember picture]
    \fill[white] ( #1 -| current page.north west) -- ++(0,.8em) -- ++(\paperwidth,0) -- (#2 -| current page.north east)
   -- ++(0,-.5em) -- ++(-\paperwidth,0) -- cycle;
  \end{tikzpicture}
  }
}

% -------------- %
% Array commands %
% -------------- %

\newcolumntype{L}[1]{>{\raggedright\let\newline\\\arraybackslash\hspace{0pt}}m{#1}}
\newcolumntype{C}[1]{>{\centering\let\newline\\\arraybackslash\hspace{0pt}}m{#1}}
\newcolumntype{R}[1]{>{\raggedleft\let\newline\\\arraybackslash\hspace{0pt}}m{#1}}


\usefonttheme{serif}

%\setbeameroption{show notes}

\title[C++ Day3]{Introduction to the C++\newline{}Programming Language\newline{}\newline{}\fontsize{16pt}{16pt}\selectfont{}Day 3}
\author{\texorpdfstring{%
    Aleksandra Rylund Glesaaen\newline\fontsize{12pt}{12pt}\selectfont\texttt{aleksandra@glesaaen.com}%
  }{%
    Aleksandra Rylund Glesaaen}}
\date{September 30th 2015}

\begin{document}

\begin{frame}
\titlepage
\end{frame}

\begin{frame}
  \frametitle{What will we learn?}

  \begin{itemize}
    \setlength\itemsep{.5em}
    \item \StrikeThrough{Basic C++ syntax}
    \item \StrikeThrough{Control structures}
    \item \StrikeThrough{Functions}
    \item Structs and classes \hspace{.25cm}{\color{Marty}(today and Thursday)}
    \item Templates and STL \hspace{.5cm}{\color{Tropiteal}(Thursday and Friday)}
    \item Exceptions \hspace{.5cm}{\color{Tropiteal}(Friday)}
  \end{itemize}
\end{frame}

\begin{frame}
  \frametitle{Today's topics}

  \tableofcontents
  
\end{frame}

\section{Dynamic Memory Management}

\frame[plain]{\sectionpage}

\begin{frame}[plain]
  \nointerlineskip
  \begin{tikzpicture}[overlay,remember picture]
    \fill[Marty] (current page.north east) rectangle (current page.south west);
    \node at ([yshift=-.25em] current page.center) [font=\LARGE, WhiteTrash] {Disclaimer};
  \end{tikzpicture}
\end{frame}

\begin{frame}
  \frametitle{Compile time \:vs\: runtime}

  {\large\color{FeebleWeek}Compile time}\,:\\[3pt]
  Things known/decided when the program is compiled, before it is ever run, also known at runtime

  \tikzmark{middle}

  \vspace{1em}
  {\large\color{Marty}Runtime}\,:\\[3pt]
  Things known/decided as the program is run,\\not known at compile time
  
\end{frame}

\begin{frame}[fragile]
  \frametitle{Two types of memory management}

  So far we have programmed using automatic memory management:

  \vspace{.5em}
  \begin{itemize}
    \item The memory need of every individual object is known at compile time
  \end{itemize}

  \begin{lstlisting}[escapeinside={(*}{*)},basicstyle=\ttfamily]
double array[100(*\tikzmark{array size}*)];
  \end{lstlisting}

  \vspace{1em}
  \begin{itemize}
    \item Cannot change the size after initialisation
  \end{itemize}

  \nointerlineskip
  \begin{tikzpicture}[overlay,remember picture]
    \draw[pointy arrow] ([shift={(-1.4ex,-.25ex)}] array size) .. controls +(0,-1.2cm) and +(-1cm,0) .. +(2cm,.5ex)
      node[right,scale=0.8,align=left] {Known by the compiler,\\is a constant expression};
  \end{tikzpicture}
\end{frame}

\begin{frame}[fragile]
  \frametitle{Two types of memory management}

  Managing memory at runtime is called\\dynamic memory management

  \vspace{.5em}
  {\color{FeebleWeek}\large Advantage:}
  \begin{itemize}
    \item More flexible than automatic memory management
  \end{itemize}

  \vspace{.5em}
  {\color{Marty}\large{}Disadvantage:}
  { \setbeamercolor*{item}{fg=Marty}
  \begin{itemize}
    \item Compiler cannot optimise your code as well
  \end{itemize}
  }
\end{frame}

\begin{frame}[fragile]
  \frametitle{Constructing memory at runtime}

  Use {\color{FeebleWeek}new} expressions to create memory

  \vspace{.5em}

  \hspace{1em}{\ttfamily\color{sgreen}new} \hspace{.5em} type \hspace*{.5em} {\ttfamily\{} \:constructor arguments\:
  {\ttfamily\};}\tikzmark{constructor}

  \vspace{1em}
  The expression returns a pointer to the memory location that was created

  \vspace{1em}
  Memory allocated this way is not limited by its scope\\
  {\footnotesize\{but the associated pointer is\}}

  \only<2>{
  \nointerlineskip
  \begin{tikzpicture}[overlay,remember picture]
    \draw[pointy arrow] ([shift={(-1.5ex,1.75ex)}] constructor) -- +(0,1cm)
      node[above,align=left,scale=0.6,xshift=1cm] {C++11 Style\\Constructor Braces};
  \end{tikzpicture}
  }

\end{frame}

\begin{frame}[fragile]
  \frametitle{Constructing memory at runtime}

  \begin{lstlisting}[escapeinside={(*}{*)},basicstyle=\ttfamily]
int * iptr;
iptr = new int {5};

auto dptr = new double {5.1};
*dptr = 2.16;
  \end{lstlisting}

\end{frame}

\begin{frame}[fragile]
  \frametitle{Constructing memory at runtime}

  Might seem a bit pointless at the moment but it will be more important as we later discuss polymorphism and data ownership

  \vspace{1em}
  It also adds flexibility to our programming

\end{frame}

\begin{frame}
  \frametitle{No more free cleanup}

  Dynamic memory isn't limited by scope
  
  \vspace{1em}
  So how is it cleaned up when it is no longer needed?

  \uncover<2>{
  \vspace{1em}
  We have to do it \;\raisebox{2pt}{\ttfamily\bfseries\color{Marty}(;\_;)}
  }
  
\end{frame}

\begin{frame}[fragile]
  \frametitle{No more free cleanup}

  A fun little program

  \vspace{.5em}
  \begin{lstlisting}[escapeinside={(*}{*)},basicstyle=\ttfamily\fontsize{12pt}{12pt}\selectfont]
int main()
{
  while (true) {
    new int {0};
  }
}
  \end{lstlisting}

\end{frame}

\begin{frame}[fragile]
  \frametitle{No more free cleanup}

  Clean up memory with {\color{Marty}delete} expressions

  \vspace{.5em}
  \hspace{1em}{\ttfamily\color{sgreen}delete} \hspace{.2em} pointer to memory {\ttfamily;}

  \vspace{1.5em}
  {\color{FeebleWeek}new} and {\color{Marty}delete} must always come in pairs,\\
  otherwise you have memory leaks

\end{frame}

\begin{frame}[fragile]
  \frametitle{No more free cleanup}

  \begin{lstlisting}[escapeinside={(*}{*)}]
int main()
{
  auto dynamic_memory = new int {4};

  // Carry out the program

  delete dynamic_memory;
}
  \end{lstlisting}

\end{frame}

\begin{frame}[fragile]
  \frametitle{Creating arrays}

  More immediate value with arrays

  \vspace{1em}
  \begin{lstlisting}[escapeinside={(*}{*)},basicstyle=\ttfamily]
auto array = new int [10(*\tikzmark{array size}*)];
  \end{lstlisting}

  \begin{uncoverenv}<2>
  \vspace*{.5em}
  \begin{lstlisting}[escapeinside={(*}{*)},basicstyle=\ttfamily]]
// ...

delete [] array;
  \end{lstlisting} 
  \end{uncoverenv}

  \only<1>{
  \nointerlineskip
  \begin{tikzpicture}[overlay,remember picture]
    \draw[pointy arrow] ([shift={(-1ex,-.5ex)}] array size) .. controls +(0,-1cm) and +(1cm,0) .. +(-2cm, -1.5cm)
      node[right,scale=0.8,align=left,anchor=east] {Does not need to be a\\compile time constant};
  \end{tikzpicture}
  }
  
\end{frame}

\begin{frame}[fragile]
  \frametitle{Creating arrays}

  \begin{lstlisting}[escapeinside={(*}{*)}]
int main()
{
  unsigned size {1};
  std::cout << "Array size: ";
  std::cin >> size;

  auto array = new int [size];
  
  //...
  delete [] array;
}
  \end{lstlisting}

\end{frame}

\begin{frame}[fragile]
  \frametitle{Multi dimensional arrays}

  Can also do dynamic multi dimensional arrays

  \vspace{1em}
  Multi dimensional arrays are just arrays of arrays

  \vspace{.5em}
  Type of 2D {\color{FeebleWeek}int} array: \hspace{.5em} \lstinline!int**!

  \vspace{.5em}
  Type of 4D {\color{FeebleWeek}float} array: \hspace{.5em} \lstinline!float****!
  
\end{frame}

\begin{frame}[fragile]
  \frametitle{Multi dimensional arrays}

  \begin{lstlisting}[escapeinside={(*}{*)},basicstyle=\ttfamily\fontsize{9pt}{9pt}\selectfont]
float** createArray(unsigned size_x, unsigned size_y)
{
  auto array = new float* [size_x];

  for (auto i = 0; i < size_x; ++i) {
    array[i] = new float [size_y];

    for (auto j = 0; j < size_y; ++j) {
      array[i][j] = 0.;
    }
  }

  return array;
}
  \end{lstlisting}

\end{frame}

\begin{frame}[fragile]
  \frametitle{Multi dimensional arrays}

  \begin{lstlisting}[escapeinside={(*}{*)},basicstyle=\ttfamily\fontsize{9pt}{9pt}\selectfont]
void deleteArray(float ** array, unsigned size_x)
{
  for (auto i = 0; i < size_x; ++i) {
    delete [] array[i];
  }

  delete [] array;
}
  \end{lstlisting}

\end{frame}

\begin{frame}
  \frametitle{Welcome to Memory Leak City}

  {\ttfamily {\color{Tropiteal}population} += {\color{Tropiteal}you};}

  \vspace{1em}

  If you believe that you are able to perfectly manage your own memory you are {\color{Marty}wrong}
  {\fontsize{8pt}{8pt}\selectfont \{I've tried as well\}}

  \vspace{1em}
  There are so many things that can go wrong

  \vspace{1em}
  We will look at some options in the exercises and on the final day
  
\end{frame}

\begin{frame}
  \frametitle{Debugging memory leaks}

  There are many tools available for debugging memory issues, but one of the best known ones is {\color{Tropiteal}valgrind}

  \vspace{1em}
  Valgrind checks for things such as
  
  \vspace{.3em}
  \begin{itemize}
    \setlength\itemsep{.3em}
    \item Reading out of bounds
    \item Using undefined values
    \item Double freeing of memory
    \item Memory leaks
  \end{itemize}
  
\end{frame}

\section{Object Oriented Programming}

\frame[plain]{\sectionpage}

\begin{frame}
  \frametitle{What exactly is OO?}

  Object oriented programming is a programming paradigm where one focuses on objects rather than methods.

  \vspace{.5cm}
  One organises the code into objects and interfaces, defining how they interact with each other and how they can be manipulated.

\end{frame}

\begin{frame}
  \frametitle{What exactly is OO? - example}

  An {\color{FeebleWeek}Address Book} is

  \begin{itemize}
    \item made up of addresses
  \end{itemize}

  \vspace{.5em}
  And one can manipulate it in multiple ways

  \begin{itemize}
    \item Add addresses
    \item Remove addresses
    \item Search for addresses
    \item Copy your friends' books
  \end{itemize}

\end{frame}

\begin{frame}[fragile]
  \frametitle{What exactly is OO? - example}

  \begin{lstlisting}[
    escapeinside={(*}{*)},
    basicstyle=\ttfamily\fontsize{9pt}{9pt}\selectfont,
    morekeywords={AddressBook,string,Address,list}]
class AddressBook
{
public:
  void addAddress(Address new_address);
  void removeAddress(int address_id);
  Address search(std::string search_string) const;
  void copyAddresses(const AddressBook & other_book);

private:
  std::list<Address> addresses;
};
  \end{lstlisting}

\end{frame}

\begin{frame}[fragile]
  \frametitle{Aggregate containers}

  Can use the {\color{FeebleWeek}struct} keyword to declare a new\\type that is a combination of other types

  \vspace{.3em}
  \begin{lstlisting}[escapeinside={(*}{*)},morekeywords={string,Address}]
struct Address
{
  std::string name;
  std::string street_name;
  unsigned street_number;
  unsigned zip_code;
}; (*\tikzmark{semicolon}*)
  \end{lstlisting}

  \vspace{.3em}
  This is a great organisational tool to help\\express yourself when writing code

  \begin{onlyenv}<2>
    \nointerlineskip
    \begin{tikzpicture}[overlay,remember picture]
      \draw[pointy arrow,line width=.5pt] ([shift={(.2cm,.3ex)}] semicolon) -- +(4cm,0)
        node[right,scale=0.8,Marty] {Don't forget the semicolon};
    \end{tikzpicture}
  \end{onlyenv}

\end{frame}

\begin{frame}[fragile]
  \frametitle{Aggregate containers}
  \begin{overlayarea}{\textwidth}{7cm}
  \begin{onlyenv}<1>
  \begin{lstlisting}[escapeinside={(*}{*)},morekeywords={Coordinate}]
struct Coordinate (*\tikzmark{decl begin}*)
{
  double x, y, z;
}; (*\tikzmark{decl end}*)

int main()
{
  Coordinate edge;
  edge.x = 4.5;
  edge.y = 0.0;
  edge.z = 9.1;

  //...
}
  \end{lstlisting}
  \end{onlyenv}

  \begin{onlyenv}<2>
  \begin{lstlisting}[escapeinside={(*}{*)},morekeywords={Coordinate}]
struct Coordinate
{
  double x, y, z;
};

int main()
{
  Coordinate edge {4.5, 0.0, 9.1};(*\tikzmark{initialiser list}*)

  //...
}
  \end{lstlisting}
  \end{onlyenv}
  \end{overlayarea}

  \nointerlineskip
  \begin{tikzpicture}[overlay,remember picture]
    \only<1>{
      \coordinate (begin decl) at ([shift={(1cm,1ex)}] decl begin);
      \draw[decorate,decoration={brace,amplitude=5pt},line width=1pt]
        (begin decl) -- (decl end -| begin decl)
        node[midway,right=.2cm,scale=0.8,text width=5cm] {%
          Type declaration
        };
    }
    \only<2>{
      \draw[pointy arrow] ([shift={(-5.45ex,-.5ex)}] initialiser list) -- +(0,-1cm)
        node[below,align=center,scale=0.8] {Construction with initialiser list\\Order as in type declaration};
    }
  \end{tikzpicture}
  
\end{frame}

\begin{frame}[fragile]
  \frametitle{Class members - variables}

  The variables in type declarations are called\\member variables

  \vspace{.5em}
  When accessed they can be used as any other\\variable of the same type

  \vspace{.5em}
  \begin{lstlisting}[morekeywords={Coordinate},escapeinside={(*}{*)}]
int main()
{
  Coordinate endp {9.4, 8.2, -3.4};
  auto dist = norm(endp.x, endp.y, endp.z);(*\tikzmark{norm}*)
}
  \end{lstlisting}

  
  \begin{onlyenv}<2>
  \nointerlineskip
  \begin{tikzpicture}[overlay,remember picture]
    \draw[pointy arrow] ([shift={(-5.7cm,-.5ex)}] norm) .. controls +(0,-.5cm) and +(-.5cm,0) .. +(.5cm,-.75cm)
      node[right,align=left,scale=0.75] {%
        Could overload {\ttfamily\color{sblue} norm} to\\accept a {\ttfamily\color{sgreen} Coordinate} object
      };
  \end{tikzpicture}
  \end{onlyenv}
  
\end{frame}

\begin{frame}[fragile]
  \frametitle{Class members - functions}
  
  The Real\texttrademark \,Object Oriented Programming starts here

\end{frame}

\begin{frame}[fragile]
  \frametitle{Class members - functions}

  Classes can also have functions as members

  \Highlight[opacity=0.15,TealDrop]{memfunc begin}{memfunc end}
  
  \vspace{.5em}
  \begin{lstlisting}[escapeinside={(*}{*)},morekeywords={Coordinate}]
class Coordinate
{
public:
  double x, y, z;

  double norm() const (*\tikzsupermark{memfunc begin}*)
  {
    return std::sqrt(x*x + y*y + z*z);
  } (*\tikzsupermark{memfunc end}*)
};
  \end{lstlisting}

  \vspace{.5em}
  {\fontsize{12pt}{12pt}\selectfont The member variables are "in scope" of the member functions}

\end{frame}

\begin{frame}[fragile]
  \frametitle{Class members - functions}

  The function is bound to the object calling it

  \vspace{.5em}
  \begin{lstlisting}[escapeinside={(*}{*)},morekeywords={Coordinate}]
int main()
{
Coordinate edge {4, 2, 1};
Coordinate point {-1, 6, 2};

double length = edge.norm(); (*\tikzmark{edge call}*)
double distance = point.norm(); (*\tikzmark{point call}*)
}
  \end{lstlisting}

  \nointerlineskip
  \begin{tikzpicture}[overlay,remember picture]
    \coordinate (edge point) at ([yshift=.3ex] edge call);
    \draw[pointy arrow,line width=.5pt] (edge point) -- +(2cm,0)
      coordinate (edge end)
      node[right,scale=0.75,font=\ttfamily] {4.58...};
    \coordinate (point point) at ([yshift=.3ex] point call);
    \draw[pointy arrow,line width=.5pt] (point point) -- (point point -| edge end)
      node[right,scale=0.75,font=\ttfamily] {6.40...};
  \end{tikzpicture}
  
\end{frame}

\begin{frame}[fragile]
  \frametitle{Declaration and definition}

  Special syntax for defining member functions

  \begin{overlayarea}{\textwidth}{6cm}
  \begin{onlyenv}<1>
    \begin{lstlisting}[escapeinside={(*}{*)},morekeywords={Coordinate}]
class Coordinate
{
public:
  double x, y, z;

  double norm() const
  {
    return std::sqrt(x*x + y*y + z*z); (*\tikzmark{inline member}*)
  }
};
    \end{lstlisting}

    \nointerlineskip
    \begin{tikzpicture}[overlay,remember picture]
      \draw[pointy arrow] ([shift={(.5cm,.3ex)}] inline member) .. controls +(1cm,0) and +(1cm,0) .. +(-.5cm,-1.2cm)
        node[anchor=east,left,scale=0.8] {%
          Automatically {\color{FeebleWeek}inline} function
        };
    \end{tikzpicture}
  \end{onlyenv}

  \begin{onlyenv}<2>
    \begin{lstlisting}[escapeinside={(*}{*)},morekeywords={Coordinate}]
class Coordinate
{
public:
  double x, y, z;

  double norm() const;
};

double Coordinate::norm() const
{
  return std::sqrt(x*x + y*y + z*z);
}
    \end{lstlisting}
  \end{onlyenv}
  \end{overlayarea}
  
\end{frame}

\begin{frame}
  \frametitle{Encapsulation}

  Encapsulation is the concept of separating the outward functionality of a class from the inner workings of it
  
\end{frame}

\begin{frame}[fragile]
  \frametitle{Encapsulation - example}

  \begin{lstlisting}[escapeinside={(*}{*)},morekeywords={Coordinate}]
class Coordinate
{
private:
  //...

public:
  void setCartesian(double x, double y);
  void setPolar(double r, double phi);
  double norm() const;
};
  \end{lstlisting}

  \vspace{.5em}
  {\fontspec{Yanone Kaffeesatz Regular}
    It is not important for someone using the class whether the coordinate
    is stored in the polar or Cartesian coordinate system
  }
  
\end{frame}

\begin{frame}[fragile]
  \frametitle{Encapsulation - example}

  \begin{tikzpicture}
    \node[draw, line width=.5pt,anchor=north] (impl1) [scale=0.5] {%
      \begin{minipage}{9cm}
        \begin{lstlisting}[morekeywords={Coordinate}]

class Coordinate
{
private:
  double radius, angle;

public:
  void setCartesian(double x, double y)
  {
    radius = std::sqrt(x*x + y*y);
    angle = std::atan(y/x);
  }

  void setPolar(double r, double phi)
  {
    radius = r;
    angle = phi;
  }

  double norm() const
  {
    return radius;
  }
};
        \end{lstlisting}
      \end{minipage}
    };

    \node[fill=WhiteTrash] at (impl1.north) [font=\ttfamily,scale=0.4] {Implementation \#1};

    \node[draw, line width=.5pt] [right=of impl1.north east, anchor=north west] (impl2) [scale=0.5] {%
      \begin{minipage}{9.5cm}
        \begin{lstlisting}[morekeywords={Coordinate}]

class Coordinate
{
private:
  double x_comp, y_comp;

public:
  void setCartesian(double x, double y)
  {
    x_comp = x;
    y_comp = y;
  }

  void setPolar(double r, double phi)
  {
    x_comp = r * std::cos(phi);
    y_comp = r * std::sin(phi);
  }

  double norm() const
  {
    return std::sqrt(
      x_comp*x_comp + y_comp*y_comp);
  }
};
        \end{lstlisting}
      \end{minipage}
    };

    \node[fill=WhiteTrash] at (impl2.north) [font=\ttfamily,scale=0.4] {Implementation \#2};
  \end{tikzpicture}
  
\end{frame}

\begin{frame}[fragile]
  \frametitle{Encapsulation - example}

  For someone using the {\color{FeebleWeek}Coordinate} class\\these two implementations are identical

  \vspace{1em}
  Thus we can switch between the two without worrying that the rest of our program will change

  \vspace{1em}
  Encapsulation adds {\color{Marty}flexibility}
  
\end{frame}

\begin{frame}
  \frametitle{Access levels}

  Class access levels facilitate encapsulation in C++

  \vspace{.5em}
  There are \:{\large\color{FeebleWeek}3}\: access levels in C++

  \vspace{.5em}
  \begin{itemize}
    \setlength\itemsep{.5em}
    \item public\\
      {\fontsize{12pt}{12pt}\selectfont Accessible by everyone}
    \item private\\
      {\fontsize{12pt}{12pt}\selectfont Only accessible by other members and friends}
    \item protected\\
      {\fontsize{12pt}{12pt}\selectfont Accessible by children classes}
  \end{itemize}
  
\end{frame}

\begin{frame}[fragile]
  \frametitle{Access levels}

  Cannot access private members from the outside

  \vspace{.5em}
  \begin{lstlisting}[escapeinside={(*}{*)},morekeywords={Coordinate}]
class Coordinate
{
private:
  double x, y;
};

int main()
{
  Coordinate c;
  c.x = 5;
}
  \end{lstlisting}

  \vspace{.5em}
  {\color{Marty}This will not compile}
  
\end{frame}

\begin{frame}[fragile]
  \frametitle{class \,vs\, struct}

  The only difference between {\color{FeebleWeek}class} and {\color{FeebleWeek}struct} in C++

  \vspace{.5em}
  \hspace{1cm}Classes are {\color{Marty}private} by default

  \vspace{.5em}
  \hspace{1cm}Structs are {\color{FeebleWeek}public} by default

  \vspace{1em}
  \begin{onlyenv}<1>
  \begin{lstlisting}[escapeinside={(*}{*)},morekeywords={Container}]
struct Container
{
  double x; (*\tikzmark{struct x}*)
};
  \end{lstlisting}

  \nointerlineskip
  \begin{tikzpicture}[overlay,remember picture]
    \draw[pointy arrow] ([shift={(0,.3ex)}] struct x) -- +(2cm,0)
      node[right,scale=0.75] {x is public};
  \end{tikzpicture}
  \end{onlyenv}

  \begin{onlyenv}<2>
  \begin{lstlisting}[escapeinside={(*}{*)},morekeywords={Container}]
class Container
{
  double x; (*\tikzmark{class x}*)
};
  \end{lstlisting}

  \nointerlineskip
  \begin{tikzpicture}[overlay,remember picture]
    \draw[pointy arrow] ([shift={(0,.3ex)}] class x) -- +(2cm,0)
      node[right,scale=0.75] {x is private};
  \end{tikzpicture}
  \end{onlyenv}
  
\end{frame}

\begin{frame}[fragile]
  \frametitle{class \,vs\, struct}

  For readability one makes the distinction anyway

  \vspace{1em}
  {\large\color{FeebleWeek}struct}\\
  congregate data structure

  \vspace{1em}
  {\large\color{FeebleWeek}class}\\
  encapsulated type with an interface

\end{frame}

\begin{frame}[fragile]
  \frametitle{Friendship}

  {\color{FeebleWeek}friend}s of classes can access their private members

  \vspace{.5em}
  \begin{lstlisting}[escapeinside={(*}{*)},morekeywords={Coordinate}]
class Coordinate
{
friend double norm(const Coordinate &); 

private:
  double x, y;
};

double norm(const Coordinate & c) 
{
  return std::sqrt(c.x*c.x + c.y*c.y);
} 
  \end{lstlisting}
  
\end{frame}

\begin{frame}
  \frametitle{Constructor}

  The constructor is the function that is called when the object is initialised

  \vspace{1em}
  There are \,{\large\color{Tropiteal}3}\, types of default constructors
  
  \vspace{.5em}
  \begin{itemize}
    \setlength\itemsep{.5em}
    \item Default constructor\\
      {\fontspec{Yanone Kaffeesatz Regular} Calls the default constructor on all members}
    \item Copy constructor\\
      {\fontspec{Yanone Kaffeesatz Regular} Copies all nonstatic members}
    \item Move constructor {\footnotesize\color{Tropiteal}\{C++11\}}
  \end{itemize}
  
\end{frame}

\begin{frame}[fragile]
  \frametitle{Constructor}

  \begin{lstlisting}[escapeinside={(*}{*)},morekeywords={Coordinate}]
class Coordinate
{
private:
  double x, y;
};

int main()
{
  Coordinate c1; (*\tikzmark{call 1}*)

  Coordinate c2 {c1}; (*\tikzmark{call 2}*)

  Coordinate c3 = c2; (*\tikzmark{call 3}*)

  auto c4 = Coordinate {}; (*\tikzmark{call 4}*)
}
  \end{lstlisting}

  \nointerlineskip
  \begin{tikzpicture}[overlay,remember picture]
    \coordinate (call1 begin) at ([yshift=.3ex] call 1);
    \draw[pointy arrow] (call1 begin) -- +(3cm,0)
      coordinate (call1 end)
      node [right,scale=0.75] {Default constructor};
    \coordinate (call2 begin) at ([yshift=.3ex] call 2);
    \draw[pointy arrow] (call2 begin) -- (call2 begin -| call1 end)
      node [right,scale=0.75] {Copy constructor};
    \coordinate (call3 begin) at ([yshift=.3ex] call 3);
    \draw[pointy arrow] (call3 begin) -- (call3 begin -| call1 end)
      node [right,scale=0.75] {Copy constructor};
    \coordinate (call4 begin) at ([yshift=.3ex] call 4);
    \draw[pointy arrow] (call4 begin) -- (call4 begin -| call1 end)
      node [right,scale=0.75] {Default constructor};
  \end{tikzpicture}

\end{frame}

\begin{frame}[fragile]
  \frametitle{Constructor}

  We can change their behaviour

  \begin{overlayarea}{\textwidth}{6cm}
    \begin{onlyenv}<1>
  \begin{lstlisting}[escapeinside={(*}{*)},morekeywords={Coordinate}]
class Coordinate
{
public:
  Coordinate()
  {
    x = 0.0; (*\tikzmark{x val}*)
    y = 0.0; (*\tikzmark{y val}*)
  }

private:
  double x, y;
};
  \end{lstlisting}

  \nointerlineskip
  \begin{tikzpicture}[overlay,remember picture]
    \coordinate (default begin) at ([yshift=1.5ex] x val);
    \coordinate (default end) at ([yshift=-.5ex] y val);
    \draw[transform canvas={xshift=2cm},decorate,decoration={brace,amplitude=5pt},line width=1pt]
      (default begin) -- (default end)
      node[midway,right=.5em,scale=0.75,align=left] {%
        Should always initialise\\built in types with a default value
      };
  \end{tikzpicture}

    \end{onlyenv}

    \begin{onlyenv}<2>
  \begin{lstlisting}[escapeinside={(*}{*)},morekeywords={Coordinate}]
class Coordinate
{
public:
  Coordinate()
    : x {0.0},
      y {0.0}
  {}

private:
  double x, y;
};
  \end{lstlisting}
    \end{onlyenv}
  \end{overlayarea}

\end{frame}

\begin{frame}[fragile]
  \frametitle{Constructor}

  ...or we can declare new constructors

  \vspace{.3em}
  \begin{lstlisting}[escapeinside={(*}{*)},morekeywords={Coordinate}]
class Coordinate
{
public:
  Coordinate(double x0, double y0)
    : x {x0},
      y {y0}
  {}

private:
  double x, y;
};
  \end{lstlisting}


  \vspace{.3em}
  \uncover<2>{\fontspec{Yanone Kaffeesatz Regular}
    {\color{Marty}Note:} if you declare your own constructor, the default constructor will not be automatically generated any more
  }

\end{frame}

\begin{frame}[fragile]
  \frametitle{Constructor}

  \begin{lstlisting}[escapeinside={(*}{*)},morekeywords={Coordinate}]
int main()
{
  Coordinate c1 {5.2, 9.1}; (*\tikzmark{new construct}*)

  Coordinate c2; (*\tikzmark{default}*)
}
  \end{lstlisting}

  \nointerlineskip
  \begin{tikzpicture}[overlay,remember picture]
    \draw[pointy arrow] ([yshift=.3ex] new construct) -- +(1cm,0)
      coordinate (new end)
      node [right, scale=0.75] {Calls our new constructor};
    \coordinate (default begin) at ([yshift=.3ex] default);
    \draw[pointy arrow] (default begin) -- (default begin -| new end)
      node [right, scale=0.75] {{\color{Marty}Error:} no such constructor};
  \end{tikzpicture}

\end{frame}

\begin{frame}[fragile]
  \frametitle{Constructor - default}

  but we can reinstate the default constructors

  \begin{lstlisting}[escapeinside={(*}{*)},morekeywords={Coordinate}]
class Coordinate
{
public:
  Coordinate(double x0, double y0)
    : x {x0},
      y {y0}
  {}

  Coordinate() = default;
  Coordinate(const Coordinate &) = default;

private:
  double x, y;
};
  \end{lstlisting}


  \CPPEleven

\end{frame}

\begin{frame}[fragile]
  \frametitle{Constructor - delete}

  and we can delete them if we don't want them

  \begin{lstlisting}[escapeinside={(*}{*)},morekeywords={Coordinate,delete}]
class Coordinate
{
public:
  Coordinate(double x0, double y0)
    : x {x0},
      y {y0}
  {}

  Coordinate(const Coordinate &) = delete;
  Coordinate(Coordinate &&) = delete;

private:
  double x, y;
};
  \end{lstlisting}

  \CPPEleven

\end{frame}

\begin{frame}[fragile]
  \frametitle{Implicit conversions}

  A constructor taking only one argument can be used by the compiler for conversions
  
  \vspace{.3em}
  \begin{lstlisting}[escapeinside={(*}{*)}]
class SomeClass
{
public:
  SomeClass(double);
};
  \end{lstlisting}
  \begin{lstlisting}[escapeinside={(*}{*)},morekeywords={SomeClass}]
void someFunction(SomeClass);

int main()
{
  someFunction(2.45);
}
  \end{lstlisting}
  
\end{frame}

\begin{frame}[fragile]
  \frametitle{Implicit conversions}

  \begin{onlyenv}<1>
  These can be disabled by the {\color{FeebleWeek}explicit} keyword

  \vspace{.5em}
  \begin{lstlisting}[escapeinside={(*}{*)}]
class SomeClass
{
public:
  explicit SomeClass(double);
};
  \end{lstlisting}
  \end{onlyenv}

  \begin{onlyenv}<2>
  {\color{Marty}Pitfall:}\\Constructors with default arguments can also be used

  \vspace{.5em}
  \begin{lstlisting}[escapeinside={(*}{*)}]
class SomeClass
{
public:
  explicit SomeClass(double, double = 2.4);
};
  \end{lstlisting}
  \end{onlyenv}

\end{frame}

\begin{frame}
  \frametitle{lvalues and rvalues}

  \begin{center}
    \begin{tikzpicture}[every node/.style={scale=1.5}]
      \node[draw, line width=2pt, FeebleWeek] (rvalue) {lvalue};
      \node[right=of rvalue] (eq) {=};
      \node[draw, line width=2pt, Marty] [right=of eq] {rvalue};
    \end{tikzpicture}
  \end{center}

\end{frame}

\begin{frame}
  \frametitle{lvalues and rvalues}

  {\large\color{Marty} lvalue}\\
  An lvalue is an object that persists after a single expression, can be at the left hand side of an assignment operator

  \vspace{1em}
  {\large\color{Marty} rvalue}\\
  An rvalue is a temporary object that do not persist after the expression, can only be at the right hand side of an assignment
  operator
  
\end{frame}

\begin{frame}[fragile]
  \frametitle{lvalue and rvalue references}

  Normal references are lvalue references\\[3pt]
  \hspace{1em} {\color{FeebleWeek}type} {\ttfamily \&}
  

  \vspace{1em}
  rvalue references are a way to signal that we don't intend to use the object after that point\\[3pt]
  \hspace{1em} {\color{FeebleWeek}type} {\ttfamily \&\&}

  \vspace{.5em}
  Convert lvalue reference to rvalue reference with the \lstinline!std::move! function in 
  {\ttfamily\color{sorange}<utility>}
  
  \CPPEleven
\end{frame}

\begin{frame}[fragile]
  \frametitle{Move constructors}

  \Highlight[opacity=0.15,TealDrop]{move constructor begin}{move constructor end}

  \begin{lstlisting}[escapeinside={(*}{*)},morekeywords={MemoryManager}]
class MemoryManager
{
public:
  MemoryManager(const MemoryManager & copy)
    : d_ptr {new double {*(copy.d_ptr)}} {}

  MemoryManager(MemoryManager && move) (*\tikzsupermark{move constructor begin}*)
    : d_ptr {move.d_ptr}
  {
    move.d_ptr = nullptr;
  } (*\tikzsupermark{move constructor end}*)

private:
  double * d_ptr;
};
  \end{lstlisting}

  \CPPEleven
\end{frame}

\begin{frame}
  \frametitle{Destructor}

  The destructor is the function that is called when the object goes out of scope

  \vspace{.5em}
  It will always automatically call the destructor of all the class' members,
  but you can add additional functionality
  
\end{frame}

\begin{frame}[fragile]
  \frametitle{Destructor - example}

  \Highlight[opacity=0.15,TealDrop]{d begin}{d end}
  
  \begin{lstlisting}[escapeinside={(*}{*)},morekeywords={MemoryManager}]
class MemoryManager
{
public:
  MemoryManager() = default;
  MemoryManager(const MemoryManager &);

  ~MemoryManager() (*\tikzsupermark{d begin}*)
  {
    delete d_ptr;
  } (*\tikzsupermark{d end}*)

private:
  double * d_ptr;
};
  \end{lstlisting}

\end{frame}

\begin{frame}
  \frametitle{Operator overloading}

  It is also possible to define how your class behaves together with all the operators of C++

  \vspace{1em}
  What should \lstinline[morekeywords={Coordinate}]!Coordinate+Coordinate! do?

  \vspace{1em}
  What should \lstinline[morekeywords={Coordinate}]!Coordinate*Coordinate! do?

  \vspace{1em}
  What about \lstinline[morekeywords={Coordinate}]!++Coordinate! ?
  
\end{frame}

\begin{frame}[fragile]
  \frametitle{Operator overloading - arithmetic}

  \begin{lstlisting}[escapeinside={(*}{*)},morekeywords={Coordinate},basicstyle=\ttfamily\fontsize{9pt}{9pt}\selectfont]
class Coordinate
{
public:
  Coordinate operator+(const Coordinate & rhs) const
  {
    auto result = *this; (*\tikzmark{this}*)
    result.x += rhs.x;
    result.y += rhs.y;
    result.z += rhs.z;
    
    return result;
  }
};
  \end{lstlisting}

  \vspace{.2em}
  {\fontspec{Yanone Kaffeesatz Regular}
    For member function declarations the calling object is always on the left hand
    side of the operator for two variable operators
  }

  \begin{onlyenv}<2>
    \nointerlineskip
    \begin{tikzpicture}[overlay,remember picture]
      \draw[pointy arrow, line width=.5] ([yshift=.25ex] this) -- +(1cm,0)
      node [right,scale=0.6,align=left,yshift=-1ex] {%
        {\color{FeebleWeek}this} is a pointer to the\\object calling the function
      };
    \end{tikzpicture}
  \end{onlyenv}
  
\end{frame}

\begin{frame}[fragile]
  \frametitle{Operator overloading - arithmetic}

  \begin{lstlisting}[escapeinside={(*}{*)},morekeywords={Coordinate},basicstyle=\ttfamily\fontsize{9pt}{9pt}\selectfont]
class Coordinate
{
friend Coordinate operator+(
  const Coordinate &, const Coordinate &);
};

Coordinate operator+(
  const Coordinate &lhs, const Coordinate &rhs)
{
  auto result = lsh;
  result.x += rhs.x;
  result.y += rhs.y;
  result.z += rhs.z;
  
  return result;
}
  \end{lstlisting}
  
\end{frame}

\begin{frame}[fragile]
  \frametitle{Order matters}

  Just as in mathematics, argument order matters

  \vspace{.3em}
  \begin{lstlisting}[escapeinside={(*}{*)},morekeywords={Coordinate}]
class Coordinate
{
public:
  Coordinate operator*(double);
};

int main()
{
  Coordinate distance {4.5, 9.0};

  auto twice = distance * 2; (*\tikzmark{twice}*)
  auto thrice = 3 * distance; (*\tikzmark{thrice}*)
}
  \end{lstlisting}

  \nointerlineskip
  \begin{tikzpicture}[overlay,remember picture]
    \coordinate (twice begin) at ([yshift=.35ex] twice);
    \draw[pointy arrow] (twice begin) -- +(1.5cm,0)
      coordinate (twice end)
      node[right, scale=0.75] {OK};
    \coordinate (thrice begin) at ([yshift=.35ex] thrice);
    \draw[pointy arrow] (thrice begin) -- (thrice begin -| twice end)
      node[right, scale=0.75] {{\color{Marty}Error:} not defined};
  \end{tikzpicture}

\end{frame}

\begin{frame}[fragile]
  \frametitle{Operator overloading - assignment}

  The assignment operator is also automatically generated by the compiler if not explicitly declared

  \vspace{.5em}
  \begin{lstlisting}[escapeinside={(*}{*)},morekeywords={Coordinate}]
class Coordinate
{
public:
  Coordinate& operator=(const Coordinate &);
  Coordinate& operator=(Coordinate &&);
};
  \end{lstlisting}

  \vspace{.5em}
  Should return a reference to {\color{FeebleWeek}this}
  
\end{frame}

\begin{frame}[fragile]
  \frametitle{Operator overloading - stream}

  \only<1>{
    Change printing behaviour by overloading the\\bitshift operator for stream objects
  }

  \begin{onlyenv}<2>
  \begin{lstlisting}[escapeinside={(*}{*)},morekeywords={Coordinate,ostream}]
class Coordinate
{
friend std::ostream& operator<<(
  std::ostream&, const Coordinate &);
};

std::ostream& operator<<(
  std::ostream & out, const Coordinate & c)
{
  out << "{" << c.x << ","
    << c.y << "," << c.z << "}";
  
  return out;
}
  \end{lstlisting}
  \end{onlyenv}
  
\end{frame}

\begin{frame}[fragile]
  \frametitle{Pre- and post increment}

  {\large Pre increment} \; \lstinline!++object!\\[5pt]
  Returns the value the object has {\color{FeebleWeek}after} it has been incremented


  \vspace{1em}
  {\large Post increment} \; \lstinline!object++!\\[5pt]
  Returns the value the object had {\color{Marty}before} it was incremented
  
\end{frame}

\begin{frame}[fragile]
  \frametitle{Operator overloading - increment}

  \nointerlineskip
  \begin{tikzpicture}[overlay,remember picture]
    \path (pic cs:pre increment begin,{(0,0)}) coordinate (a);
    \path (pic cs:pre increment end,{(0,0)}) coordinate (b);
    \fill[opacity=0.25,AtomicBikini] (a -| current page.north west) -- ++(0,.9ex) -- ++(\paperwidth,0)
    -- (b -| current page.north east) -- ++(0,-.4ex) -- ++(-\paperwidth,0) -- cycle;
    \coordinate (a edge) at (a -| current page.east);
    \coordinate (b edge) at (b -| current page.east);
    \node at ($(a edge) !.5! (b edge)$) [anchor=east,xshift=-1cm,scale=0.75] {Pre increment};
  \end{tikzpicture}

  \begin{lstlisting}[escapeinside={(*}{*)},morekeywords={Counter},basicstyle=\ttfamily\fontsize{9pt}{9pt}\selectfont]
class Counter
{
public:
  Counter& operator++() (*\tikzsupermark{pre increment begin}*)
  {
    ++count;
    return *this;
  } (*\tikzsupermark{pre increment end}*)

  Counter operator++(int)
  {
    auto before = *this;
    ++count;
    return before;
  }

private:
  unsigned count;
};
  \end{lstlisting}
  
\end{frame}

\begin{frame}[fragile]
  \frametitle{Operator overloading - increment}

  \nointerlineskip
  \begin{tikzpicture}[overlay,remember picture]
    \path (pic cs:post increment begin,{(0,0)}) coordinate (a);
    \path (pic cs:post increment end,{(0,0)}) coordinate (b);
    \fill[opacity=0.15,Marty] (a -| current page.north west) -- ++(0,1ex) -- ++(\paperwidth,0)
    -- (b -| current page.north east) -- ++(0,-.4ex) -- ++(-\paperwidth,0) -- cycle;
    \coordinate (a edge) at (a -| current page.east);
    \coordinate (b edge) at (b -| current page.east);
    \node at ($(a edge) !.5! (b edge)$) [anchor=east,xshift=-1cm,scale=0.75] {Post increment};
  \end{tikzpicture}

  \begin{lstlisting}[escapeinside={(*}{*)},morekeywords={Counter},basicstyle=\ttfamily\fontsize{9pt}{9pt}\selectfont]
class Counter
{
public:
  Counter& operator++()
  {
    ++count;
    return *this;
  }

  Counter operator++(int) (*\tikzsupermark{post increment begin}*)
  {
    auto before = *this;
    ++count;
    return before;
  } (*\tikzsupermark{post increment end}*)

private:
  unsigned count;
};
  \end{lstlisting}
  
\end{frame}

\begin{frame}[fragile]
  \frametitle{And many more...}

  All the other operators can be overloaded as well

  \vspace{.25em}
  \begin{itemize}
    \setlength\itemsep{.5em}
    \item Reference {\color{Tropiteal}\&} and dereference {\color{Tropiteal}*}
    \item Arithmetic assignment {\ttfamily\color{Tropiteal}+= -= *= /=}
    \item Call operator {\color{Tropiteal}()}
    \item Element operator {\color{Tropiteal}[]}
    \item Cast operator
  \end{itemize}

  \vspace{.5em}
  except for member access {\large\color{Tropiteal}.}
  
\end{frame}

\begin{frame}[fragile]
  \frametitle{{\color{FeebleWeek}const}-ness}

  Member functions that leave the object unchanged should be marked const

  \vspace{1em}
  \begin{lstlisting}[escapeinside={(*}{*)},morekeywords={Coordinate}]
class Coordinate
{
public:
  double norm() const;
};
  \end{lstlisting}

  \vspace{1em}
  Constant instances of a class can only call methods that are marked const
  
\end{frame}

\begin{frame}[fragile]
  \frametitle{{\color{FeebleWeek}static} methods in classes}

  \begin{onlyenv}<1>
    Just as with functions, a static member of a class transcends the individual class instances
    
    \vspace{1em}
    Static methods can only use static variables and call other static methods
    
    \vspace{1em}
    Access static methods as if it was in a namespace

    \vspace{.5em}
    \begin{lstlisting}[morekeywords={SomeClass},basicstyle=\ttfamily]
SomeClass::staticMethod();
    \end{lstlisting}
  \end{onlyenv}

  \begin{onlyenv}<2>
  \begin{lstlisting}[escapeinside={(*}{*)},morekeywords={InstanceCounter}]
class InstanceCounter
{
public:
  InstanceCounter() { ++count; }
  ~InstanceCounter() { --count; }

  static unsigned getCount()
  {
    return count;
  }

private:
  static unsigned count;
};
  \end{lstlisting}
  \end{onlyenv}

  \begin{onlyenv}<3-4>
  \begin{lstlisting}[escapeinside={(*}{*)},morekeywords={InstanceCounter}]
unsigned InstanceCounter::count = 0;

int main()
{
  InstanceCounter i1;
  InstanceCounter::getCount(); (*\tikzmark{count 1}*)
  {
    InstanceCounter i1, i2, i3, i4;
    InstanceCounter::getCount(); (*\tikzmark{count 2}*)
  }
  InstanceCounter::getCount(); (*\tikzmark{count 3}*)
}
  \end{lstlisting}

  \only<4>{
  \nointerlineskip
  \begin{tikzpicture}[overlay,remember picture]
    \coordinate (a) at ([yshift=.3ex] count 1);
    \draw[pointy arrow] (a) -- +(2cm,0)
      coordinate (aend)
      node[right, scale=0.75] {1};
    \coordinate (b) at ([yshift=.3ex] count 2);
    \draw[pointy arrow] (b) -- (b -| aend)
      node[right, scale=0.75] {5};
    \coordinate (c) at ([yshift=.3ex] count 3);
    \draw[pointy arrow] (c) -- (c -| aend)
      node[right, scale=0.75] {1};
  \end{tikzpicture}
  }

  \end{onlyenv}
  
\end{frame}

\begin{frame}
  \frametitle{The rule of %
  \tikz[baseline={(three.base)}]{
    \node[inner sep=2pt] (three) {three};
    \draw[line width=2pt,Marty] ([yshift=.1cm] three.south west) -- ([yshift=-.2cm] three.north east);
  } five}

  It is generally a good idea to explicitly define

  \vspace{.5em}
  \begin{itemize}
    \setlength\itemsep{.5em}
    \item Destructor
    \item Copy constructor
    \item Move constructor {\footnotesize\color{Tropiteal}\{C++11\}}
    \item Copy assignment operator
    \item Move assignment operator {\footnotesize\color{Tropiteal}\{C++11\}}
  \end{itemize}
  
\end{frame}

\section{Programming Practices}

\frame[plain]{\sectionpage}

\begin{frame}
  \frametitle{Good Programming Practices}

  \begin{itemize}
    \setlength\itemsep{0.5em}
    \item Be wary of memory leaks
    \item Continue to use {\color{FeebleWeek}const} consistently
    \item Assign values to built in types at construction
    \item Respect encapsulation
    \item Follow the rule of five
  \end{itemize}
  
\end{frame}


\section{Recap}

\frame[plain]{\sectionpage}

\begin{frame}
  \frametitle{Recap Day 3}

  \begin{itemize}
    \setlength\itemsep{0.75em}
    \item Use {\color{FeebleWeek}new} and {\color{Marty}delete} to manage memory dynamically
    \item Classes can have member variables and member functions
    \item Member functions are bound to the class instance
    \item Classes have three levels of access levels
  \end{itemize}
  
\end{frame}

\begin{frame}
  \frametitle{Recap Day 3}

  \begin{itemize}
    \setlength\itemsep{0.75em}
    \item Constructors manipulate how new objects are created
    \item Destructors define what happens when they go out of scope
    \item Operators can be overloaded to define how they work with your classes
  \end{itemize}
  
\end{frame}

\end{document}
