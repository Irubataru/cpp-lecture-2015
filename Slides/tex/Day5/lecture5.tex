\documentclass[14pt,a4paper,dvipsnames,usenames]{beamer}
\usepackage[T1]{fontenc}
\usepackage{url}
\usepackage{xcolor}
\usepackage{amsmath,varwidth}
\usepackage{tikz,tikz-uml,xparse,url}
\usepackage{listings,algpseudocode}
\usepackage[quiet]{mathspec}
\usepackage{varwidth}

\setmainfont[
  ItalicFont={Yanone Kaffeesatz Light},
  Scale=1.3,
  LetterSpace=2.0
]{Yanone Kaffeesatz Bold}

\setmonofont{Hack}

% --------------------- %
% Beamer theme settings %
% --------------------- %

\usetheme{CodeCourse}
\setbeamertemplate{navigation symbols}{} %Remove beamer navigation symbols
\setbeamertemplate{bibliography item}{\insertbiblabel}

% ---------------------------- %
% Colour definitions for theme %
% ---------------------------- %

\definecolor{Tropiteal}{RGB}{0,168,198}
\definecolor{TealDrop}{RGB}{64,192,203}
\definecolor{WhiteTrash}{RGB}{249,242,231}
\definecolor{AtomicBikini}{RGB}{174,226,57}
\definecolor{FeebleWeek}{RGB}{143,190,0}
\definecolor{ICantExpress}{RGB}{28,20,13}
\definecolor{Love}{RGB}{209,67,52}
\definecolor{UnrealFoodPills}{RGB}{250,105,0}
\definecolor{Marty}{RGB}{250,42,0}
\definecolor{ConradGold}{RGB}{255,221,0}

\definecolor{sbase03}{HTML}{002B36}
\definecolor{sbase02}{HTML}{073642}
\definecolor{sbase01}{HTML}{586E75}
\definecolor{sbase00}{HTML}{657B83}
\definecolor{sbase0}{HTML}{839496}
\definecolor{sbase1}{HTML}{93A1A1}
\definecolor{sbase2}{HTML}{EEE8D5}
\definecolor{sbase3}{HTML}{FDF6E3}
\definecolor{syellow}{HTML}{B58900}
\definecolor{sorange}{HTML}{CB4B16}
\definecolor{sred}{HTML}{DC322F}
\definecolor{smagenta}{HTML}{D33682}
\definecolor{sviolet}{HTML}{6C71C4}
\definecolor{sblue}{HTML}{268BD2}
\definecolor{scyan}{HTML}{2AA198}
\definecolor{sgreen}{HTML}{859900}

% -------------------------------------------------- %
% Beamer notes to be itemized rather than enumerated %
% -------------------------------------------------- %

\makeatletter
\def\beamer@setupnote{%
  \gdef\beamer@notesactions{%
    \beamer@outsideframenote{%
      \beamer@atbeginnote%
      \beamer@notes%
      \ifx\beamer@noteitems\@empty\else
      \begin{itemize}\itemsep=0pt\parskip=0pt%
        \beamer@noteitems%
      \end{itemize}%
      \fi%
      \beamer@atendnote%
    }%
    \gdef\beamer@notesactions{}%
  }
}
\makeatother

% -------------------------- %
% Standard LaTeX definitions %
% -------------------------- %

\DeclareGraphicsExtensions{.pdf}
\renewcommand\textbullet{\ensuremath{\bullet}}

% ----------------------------------- %
% Various tikz styles and definitions %
% ----------------------------------- %

\usetikzlibrary{calc,intersections,positioning,matrix,chains,scopes}
\usetikzlibrary{decorations.pathmorphing,decorations.pathreplacing,shapes}
\newcommand{\tikzmark}[1]{\tikz[overlay,remember picture] \node (#1) {};}

\tikzumlset{fill class=RoyalBlue!20}

\tikzstyle{startnode} = [circle, draw, fill=Bittersweet!20, minimum size=10pt, inner sep = 0]
\tikzstyle{stopnode} = [startnode]
\tikzstyle{process} = [rectangle, minimum width=2cm, minimum height=.7cm, text centered, draw, fill=OliveGreen!20, align=center, font=\scriptsize]
\tikzstyle{dangerprocess} = [process, fill=BrickRed!20, draw=BrickRed, thick]
\tikzstyle{flowarrow} = [->, >=stealth, thick]

% ---------------------------------- %
% My default listings C++ code style %
% ---------------------------------- %

\lstset{%
  language=C++, basicstyle=\scriptsize\ttfamily, 
  keywordstyle=\color{OliveGreen}, identifierstyle=\color{RoyalBlue}, 
  commentstyle=\color{Brown}, stringstyle=\color{Bittersweet}, showstringspaces=false,  
  breaklines=true, prebreak=\mbox{{\color{Bittersweet}\tiny\ $\hookleftarrow$}}, 
  postbreak=\mbox{{\color{Bittersweet}\tiny$\to$\ }}, tabsize=5,
  morekeywords={%
    MyClass,unique_ptr,shared_ptr,weak_ptr,auto_ptr,
    make_unique, make_shared, default_delete,
    dynamic_pointer_cast, const_pointer_cast,
    scoped_ptr,QSharedPointer,QScopedPointer,QWeakPointer,
    ftring, SmartPtr, Shape, Circle, ShapeFactory, CircleFactory,
    Derived, Base
  }
}

\def\inline{\lstinline[basicstyle=\ttfamily\normalsize]}

% ---------------------------------------------------------------------------------------------------- %
% Various command for inline code instead of lstinline so that I don't have to update the keyword list %
% ---------------------------------------------------------------------------------------------------- %

\newcommand{\object}[1]{{\ttfamily \color{OliveGreen}#1}}
\newcommand{\function}[1]{{\ttfamily \color{RoyalBlue}#1}}
\newcommand{\std}[1]{{\ttfamily {\color{RoyalBlue} std::}{\color{OliveGreen}#1}}}
\newcommand{\boost}[1]{{\ttfamily {\color{RoyalBlue} boost::}{\color{OliveGreen}#1}}}

\newcommand{\uniqueptr}{\std{unique\_ptr}}
\newcommand{\sharedptr}{\std{shared\_ptr}}
\newcommand{\weakptr}{\std{weak\_ptr}}
\newcommand{\autoptr}{\std{auto\_ptr}}

\usepackage{array}

% ------------------------------------------------ %
% tikz functions to place markers on beamer slides %
% ------------------------------------------------ %

\makeatletter
\tikzset{%
  remember picture with id/.style={%
    remember picture,
    overlay,
    save picture id=#1,
  },
  save picture id/.code={%
    \edef\pgf@temp{#1}%
    \immediate\write\pgfutil@auxout{%
      \noexpand\savepointas{\pgf@temp}{\pgfpictureid}}%
  },
  if picture id/.code args={#1#2#3}{%
    \@ifundefined{save@pt@#1}{%
      \pgfkeysalso{#3}%
    }{
      \pgfkeysalso{#2}%
    }
  }
}

\def\savepointas#1#2{%
  \expandafter\gdef\csname save@pt@#1\endcsname{#2}%
}

\def\tmk@labeldef#1,#2\@nil{%
  \def\tmk@label{#1}%
  \def\tmk@def{#2}%
}

\tikzdeclarecoordinatesystem{pic}{%
  \pgfutil@in@,{#1}%
  \ifpgfutil@in@%
    \tmk@labeldef#1\@nil
  \else
    \tmk@labeldef#1,(0pt,0pt)\@nil
  \fi
  \@ifundefined{save@pt@\tmk@label}{%
    \tikz@scan@one@point\pgfutil@firstofone\tmk@def
  }{%
  \pgfsys@getposition{\csname save@pt@\tmk@label\endcsname}\save@orig@pic%
  \pgfsys@getposition{\pgfpictureid}\save@this@pic%
  \pgf@process{\pgfpointorigin\save@this@pic}%
  \pgf@xa=\pgf@x
  \pgf@ya=\pgf@y
  \pgf@process{\pgfpointorigin\save@orig@pic}%
  \advance\pgf@x by -\pgf@xa
  \advance\pgf@y by -\pgf@ya
  }%
}
\newcommand\tikzsupermark[2][]{%
\tikz[remember picture with id=#2] #1;}
\makeatother

\newcommand{\MarkText}[3][]{%
\begin{tikzpicture}[overlay,remember picture]%
  \path (pic cs:#2,{(0,\paperheight)}) +(0,.7\baselineskip) coordinate (a);
  \path (pic cs:#3,{(0,-\paperheight)}) +(0,-.3\baselineskip) coordinate (b);
  \draw [ultra thick,
    if picture id={#2}{red}{line width=1cm,green,opacity=.5},
    #1]
  (a) -- ($(b)!(a)!($(b)+(1,0)$)$);
\end{tikzpicture}%
}%

\newcommand{\tikzmark}[1]{\tikz[overlay,remember picture] \coordinate (#1);}

% -------------- %
% Special frames %
% -------------- %

\newcommand{\LiveFrame}{%
\begin{frame}[plain]
  \nointerlineskip
  \begin{tikzpicture}[overlay,remember picture]
    \fill[FeebleWeek] (current page.north east) rectangle (current page.south west);
    \node at ([yshift=-.25em] current page.center) [font=\LARGE, WhiteTrash] {Live Example};
  \end{tikzpicture}
\end{frame}
}

\newcommand{\CPPEleven}{%
\nointerlineskip
\begin{tikzpicture}[overlay,remember picture]
  \node[anchor=south east] at ( current page.south east) [color=Tropiteal] {\{C++11\}};
\end{tikzpicture}
}

\newcommand{\CPPFourteen}{%
\nointerlineskip
\begin{tikzpicture}[overlay,remember picture]
  \node[anchor=south east] at ( current page.south east) [color=Tropiteal] {\{C++14\}};
\end{tikzpicture}
}

\newcommand{\BadPractice}{%
\nointerlineskip
\begin{tikzpicture}[overlay,remember picture]
  \node[anchor=south east] at ( current page.south east) [color=Marty] {Bad Practice};
\end{tikzpicture}
}

\newcommand{\ClassLine}{%
\nointerlineskip
\begin{tikzpicture}[overlay,remember picture]
  \draw[transform canvas={yshift=-1.7cm},thin] (current page.north west) -- (current page.north east);
\end{tikzpicture}
}

\newcommand{\Library}[2][0pt]{%
\nointerlineskip
\begin{tikzpicture}[overlay,remember picture]
  \node[anchor=south east] at ([xshift=#1] current page.south east) [color=Marty,scale=0.9]
    [font=\fontspec{Yanone Kaffeesatz Regular}] {\#include<#2>};
\end{tikzpicture}
}

% ----------------------------------------- %
% Various Commands for inline codecolouring %
% ----------------------------------------- %

\newcommand{\inobj}[1]{%
  {\color{Tropiteal}#1}}

\newcommand{\inkey}[1]{%
  {\color{FeebleWeek}#1}}

\newcommand{\innorm}[1]{%
  {\color{ICantExpress}#1}}

\newcommand{\std}[1]{%
  \inobj{std}\innorm{::}\inkey{#1}}

% ----------------------------------------------------- %
% Various tikz based functions to style up the document %
% ----------------------------------------------------- %

\newcommand{\underlinemark}[3][line width=1.5pt, FeebleWeek]{%
\draw[transform canvas={yshift=-.15cm}, #1]
  ([xshift=-1pt] #2) -- ([xshift=2pt] #3);
}

\newcommand{\StrikeThrough}[2][Marty]{%
  \tikz[baseline=(strokethrough.base)]{%
    \node[inner sep=0pt, outer sep=0pt] (strokethrough) {\strut{}#2};
    \draw[line width=2pt,#1] ([xshift=-3pt] strokethrough.west) -- ([xshift=3pt] strokethrough.east);
  }
}

\newcommand{\Highlight}[3][opacity=0.4,AtomicBikini]{%
\nointerlineskip
\begin{tikzpicture}[overlay,remember picture]
  \path (pic cs:#2,{(0,0)}) coordinate (a);
  \path (pic cs:#3,{(0,0)}) coordinate (b);
  \fill[#1] (a -| current page.north west) -- ++(0,1.4ex) -- ++(\paperwidth,0)
  -- (b -| current page.north east) -- ++(0,-.4ex) -- ++(-\paperwidth,0) -- cycle;
\end{tikzpicture}
}

\newcommand{\Dimtext}[2]%
{
  { \transparent{0.7}
  \begin{tikzpicture}[overlay, remember picture]
    \fill[white] ( #1 -| current page.north west) -- ++(0,.8em) -- ++(\paperwidth,0) -- (#2 -| current page.north east)
   -- ++(0,-.5em) -- ++(-\paperwidth,0) -- cycle;
  \end{tikzpicture}
  }
}

% -------------- %
% Array commands %
% -------------- %

\newcolumntype{L}[1]{>{\raggedright\let\newline\\\arraybackslash\hspace{0pt}}m{#1}}
\newcolumntype{C}[1]{>{\centering\let\newline\\\arraybackslash\hspace{0pt}}m{#1}}
\newcolumntype{R}[1]{>{\raggedleft\let\newline\\\arraybackslash\hspace{0pt}}m{#1}}


\usefonttheme{serif}

%\setbeameroption{show notes}

\title[C++ Day5]{Introduction to the C++\newline{}Programming Language\newline{}\newline{}\fontsize{16pt}{16pt}\selectfont{}Day 5}
\author{\texorpdfstring{%
    Aleksandra Rylund Glesaaen\newline\fontsize{12pt}{12pt}\selectfont\texttt{aleksandra@glesaaen.com}%
  }{%
    Aleksandra Rylund Glesaaen}}
\date{October 2nd 2015}

\begin{document}

\begin{frame}
\titlepage
\end{frame}

\begin{frame}
  \frametitle{What will we learn?}

  \begin{itemize}
    \setlength\itemsep{.5em}
    \item \StrikeThrough{Basic C++ syntax}
    \item \StrikeThrough{Control structures}
    \item \StrikeThrough{Functions}
    \item \StrikeThrough{Structs and classes}
    \item Templates and STL \hspace{.5cm}{\color{Marty}(Thursday and today)}
    \item Exceptions \hspace{.5cm}{\color{Marty}(today)}
  \end{itemize}
\end{frame}

\begin{frame}
  \frametitle{Today's topics}

  \tableofcontents
  
\end{frame}

\section{Standard Template Library}

\frame[plain]{\sectionpage}

\begin{frame}[fragile]
  \frametitle{What is the STL?}

  A compilation of template classes and functions with a consistent interface design, it contains

  \begin{itemize}
    \setlength\itemsep{.4em}
    \item Container classes
    \item Iterators
    \item Generic algorithms
    \item Smart pointers {\footnotesize\color{Tropiteal}\{C++11\}}
    \item Random number generation {\footnotesize\color{Tropiteal}\{C++11\}}
    \item ...and more
  \end{itemize}
  
\end{frame}

\begin{frame}[fragile]
  \frametitle{Motivation}

  \begin{center}
  \begin{tikzpicture}
    \node [use as bounding box,font=\fontspec{Yanone Kaffeesatz Regular}] (quote) {%
      \begin{minipage}{0.85\textwidth}
        Using the STL is fun, but using it effectively is outrageous fun, the kind of fun
        where they have to drag you away from the keyboard, because you just can't believe
        the good time you're having.
      \end{minipage}
    };
    \node at (quote.south east) [font=\fontspec{Yanone Kaffeesatz Regular},Tropiteal] {Scott Meyers};
  \end{tikzpicture}
  \end{center}
\end{frame}

\begin{frame}[fragile]
  \frametitle{Containers}

  We have seen two ways of storing larger chunks of data in C++ so far

  \vspace{.5em}
  \begin{itemize}
    \setlength\itemsep{.3em}
    \item arrays 
    \item linked lists
  \end{itemize}

  \vspace{.5em}
  The STL contain these two as well as many more containers you can use for all your data storage needs
  
\end{frame}

\begin{frame}[fragile]
  \frametitle{Standard \underline{Template} Library}

  All the storage containers are templates

  \vspace{.5em}
  \begin{itemize}
    \setlength\itemsep{.5em}
    \item \lstinline[morekeywords={vector}]!std::vector<Penguin>!
    \item \lstinline[morekeywords={list}]!std::list<Song>!
    \item \lstinline[morekeywords={stack}]!std::stack<Card>!
    \item \lstinline[morekeywords={array}]!std::array<Hedgehog,10>!
    \item \lstinline[morekeywords={map,string}]!std::map<Coordinate,Treasure>!
  \end{itemize}
  
\end{frame}

\begin{frame}[fragile]
  \frametitle{Sequential containers}

  Containers where the data is stored one after another

  \vspace{.5em}
  \begin{itemize}
    \setlength\itemsep{.5em}
    \item \lstinline!std::vector! \tikzmark{vector}
    \item \lstinline!std::array! {\footnotesize\color{Tropiteal}\{C++11\}} \tikzmark{array}
    \item \lstinline!std::list!
    \item \lstinline!std::stack! \tikzmark{fifo}
    \item \lstinline!std::queue!
  \end{itemize}

  \nointerlineskip
  \begin{tikzpicture}[overlay,remember picture]
    \coordinate (vector begin) at ([yshift=1.25ex] vector);
    \coordinate (array end) at ([yshift=-.5ex] array);
    \coordinate (layer1) at ([xshift=.5cm] current page.north);

    \draw[decorate,decoration={brace,amplitude=5pt},line width=1pt]
      (layer1 |- vector begin) -- (layer1 |- array end)
      node[midway, right=.5em, scale=0.75,align=left] {Can (and should)\\replace C arrays};

    \draw[pointy arrow] ([shift={(.5cm,.5ex)}] fifo) -- +(1.5cm,0)
      node[right,scale=0.75] {First in, first out};
  \end{tikzpicture}
  
\end{frame}

\begin{frame}[fragile]
  \frametitle{vector and array}

  Elements are also stored sequential in memory

  \vspace{.5em}
  Can access the elements in the arrays through:
  
  \vspace{.3em}
  \begin{itemize}
    \setlength\itemsep{.5em}
    \item The access operator {\ttfamily\color{Tropiteal}[]}
    \item The \lstinline!at()! function \tikzmark{at function}
  \end{itemize}

  \vspace{.5em}
  The vector class has almost no overhead and is always preferred to dynamic C arrays

  \nointerlineskip
  \begin{tikzpicture}[overlay,remember picture]
    \draw[pointy arrow,line width=.5pt] ([shift={(.5cm,.5ex)}] at function) -- +(1.5cm,0)
      node[right,scale=0.6] {Includes bounds check};
  \end{tikzpicture}

\end{frame}

\begin{frame}[fragile]
  \frametitle{\inkey{vector}}
  \fontspec{Yanone Kaffeesatz Regular}

  {\large Container that does sequential dynamic arrays}

  \vspace{.5cm}
  {\large\color{Tropiteal}Notable functions}\\[2pt]

  \hspace*{-.37cm}
  \begin{tikzpicture}
  \matrix (functions) [
    matrix of nodes,
    column 1/.style={anchor=base west},
    column 2/.style={anchor=base west,text width=6.5cm,inner xsep=0pt,align=left},
    column sep=.3cm
  ] {
  {\color{ICantExpress!80!WhiteTrash}(constructor)}(\inkey{size\_t}) &
  sets the initial size \\
  \inkey{operator}[](\inkey{size\_t}) & access nth element \\
  \inobj{at}(\inkey{size\_t}) & access nth element w/ bounds check \\
  \inobj{resize}(\inkey{size\_t}) & resize the vector \\
  \inobj{front}(), \inobj{back}() & access first/last element \\
  \inobj{size}() & get current size of vector \\
  };

  \draw[Tropiteal] (functions-1-1.north west) -- (functions-1-2.north east);
  \draw[Tropiteal] (functions-2-1.north west) -- (functions-2-2.north east);
  \draw[Tropiteal] (functions-3-1.north west) -- (functions-3-2.north east);
  \draw[Tropiteal] (functions-4-1.north west) -- (functions-4-2.north east);
  \draw[Tropiteal] (functions-5-1.north west) -- (functions-5-2.north east);
  \draw[Tropiteal] (functions-6-1.north west) -- (functions-6-2.north east);
  \end{tikzpicture}

  \Library{vector}

  \ClassLine
  
\end{frame}

\begin{frame}[fragile]
  \frametitle{Linked lists}

  Elements are not sequential in memory

  \vspace{1cm}
  No direct access of individual elements, we need to navigate through the list structure

  \vspace{1cm}
  Deleting and inserting are both really cheap
  
  \Library{list}

\end{frame}

\begin{frame}[fragile]
  \frametitle{Initialiser lists}

  A convenient way of initialising containers is by listing their initial content

  \vspace{.5cm}
  \begin{lstlisting}[escapeinside={(*}{*)}]
std::vector<int> lucky_numbers {12, 5, 42};
std::list<char> the_word {'b', 'i', 'r', 'd'};
  \end{lstlisting}

  \vspace{.5cm}
  Can create similar constructors for our own classes by using the \std{initializer\_list} container

  \CPPEleven

  \Library[-2cm]{initializer\_list}
  
\end{frame}

\begin{frame}[fragile]
  \frametitle{C++11 constructors}

  {\Large\color{Marty}Note}

  \vspace{.2cm}
  Using C++11 constructor notation {\large\color{Tropiteal}\{\}} will pick out initialiser list constructors first

  \vspace{.5cm}
  \begin{lstlisting}[escapeinside={(*}{*)},basicstyle=\fontsize{12pt}{12pt}\selectfont\ttfamily]
std::vector zero_vector {0}; (*\tikzmark{constr 11}*)
std::vector zero_vector (0); (*\tikzmark{constr pre}*)
  \end{lstlisting}

  \nointerlineskip
  \begin{tikzpicture}[overlay,remember picture]
    \coordinate (text) at ([yshift=-1.5cm] constr 11);
    \only<1>{
    \draw[pointy arrow] ([yshift=.4ex] constr 11) .. controls +(1cm,0) and +(1cm,0) .. (text)
      node[left,scale=.75] {Vector of length 1, with element 0};
    }
    \only<2>{
    \draw[pointy arrow] ([yshift=.4ex] constr pre) .. controls +(.5cm,0) and +(.5cm,0) .. (text)
      node[left,scale=.75] {Vector of length 0};
    }
  \end{tikzpicture}
  
  \CPPEleven
  
\end{frame}

\begin{frame}[fragile]
  \frametitle{Associative containers}

  The elements are sorted, and searches are very quick

  \vspace{.5em}
  \begin{itemize}
    \setlength\itemsep{.5em}
    \item \lstinline!std::set!\\
      {\fontspec{Yanone Kaffeesatz Regular}
        collection of unique elements, sorted
      }
    \item \lstinline!std::map!\\
      {\fontspec{Yanone Kaffeesatz Regular}
        collection of key-value pairs, keys unique and sorted
      }
  \end{itemize}
  
\end{frame}

\begin{frame}[fragile]
  \frametitle{Complexity}

    \begin{tikzpicture}[remember picture,
      good/.style={
        draw, scale=0.75, fill=AtomicBikini,
        text width=1.5cm,align=center
      },
      med/.style={
        draw, scale=0.75, fill=AtomicBikini!50!ConradGold,
        text width=1.5cm,align=center
      },
      ok/.style={
        draw, scale=0.75, fill=ConradGold,
        text width=1.5cm,align=center
      }]
      \matrix [
        nodes={font=\fontspec{Yanone Kaffeesatz Regular}},
        column sep=.75cm,
        ] (complexity)
      {
                         & \node {Access};     & \node {Search};     & \node {Insert};     \\
         \node {Array};  & \node[good] {O(1)}; & \node[ok] {O(n)};   & \node[ok] {O(n)};   \\
         \node {Stack};  & \node[ok] {O(n)};   & \node[ok] {O(n)};   & \node[good] {O(1)}; \\
         \node {List};   & \node[ok] {O(n)};   & \node[ok] {O(n)};   & \node[good] {O(1)}; \\
         \node {Map};    & \node {-};          & \node[good] {O(1)}; & \node[good] {O(1)}; \\
         \node {Binary tree}; & \node[med] {O(log(n))}; & \node[med] {O(log(n))}; & \node[med] {O(log(n))}; \\
      };
    \end{tikzpicture}

  \nointerlineskip
  \begin{tikzpicture}[overlay,remember picture]
    \node[anchor=south east] at ( current page.south east) [scale=0.6] {*best case};
  \end{tikzpicture}
  
\end{frame}

\begin{frame}[fragile]
  \frametitle{What container to choose}

  \begin{center}\Large
    99\% of the time you will\\use {\color{FeebleWeek}vector} or {\color{FeebleWeek}array}
  \end{center}
  
\end{frame}

\begin{frame}[fragile]
  \frametitle{What container to choose}

  Otherwise make a careful decision based on what you need the container for

  \vspace{.5em}
  \begin{itemize}
    \setlength\itemsep{.5em}
    \item Are you going to insert elements in the middle of the container?
    \item What are your iterator needs?
    \item Do you need a fixed ordering?
    \item Is memory consistency important?
  \end{itemize}

\end{frame}

\begin{frame}[fragile]
  \frametitle{Iterators}

  For many of the containers in STL, direct access to an element is not possible, we somehow have to traverse the container
  structure, iterate if you will

\end{frame}

\begin{frame}[fragile]
  \frametitle{Iterators}

  An iterator points to an element of a container

  \vspace{.5cm}
  \hspace{1cm} \lstinline!*iterator! 

  \vspace{1cm}
  and it can move to the next element of the container

  \vspace{.5cm}
  \hspace{1cm} \lstinline!++iterator!

\end{frame}

\begin{frame}[fragile]
  \frametitle{Iterator example}

  \begin{lstlisting}[deletekeywords={set}]
#include<set>
#include<cctype>
  \end{lstlisting}
  \begin{lstlisting}[]
int main()
{
  std::set<char> lower_set {'a', 'q', 'k', 'p'};
  std::set<char> upper_set;

  for (auto it  = lower_set.begin();
            it != lower_set.end();   ++it)
    upper_set.insert(std::toupper(*it));
}
  \end{lstlisting}
\end{frame}

\begin{frame}[fragile]
  \frametitle{Iterator interface}

  Most of the containers in STL have iterators,\\and their interface is uniform

  \vspace{.5em}
  \begin{lstlisting}[escapeinside={(*}{*)},basicstyle=\ttfamily]
container.begin(); (*\tikzmark{begin iterator}*)
container.end(); (*\tikzmark{end iterator}*)

++iterator; (*\tikzmark{advance mark}*)
 *iterator; (*\tikzmark{dereference mark}*)
  \end{lstlisting}

  \nointerlineskip
  \begin{tikzpicture}[overlay,remember picture]
    \coordinate (begin) at ([yshift=.5ex] begin iterator);
    \coordinate (end) at ([yshift=.5ex] end iterator);
    \coordinate (advance) at ([yshift=.5ex] advance mark);
    \coordinate (dereference) at ([yshift=.5ex] dereference mark);
    \coordinate (layer1) at ([xshift=2cm] current page.north);
    \coordinate (layer2) at ([xshift=.5cm] current page.north);

    \only<1>{
    \draw[pointy arrow] (begin) -- (begin -| layer1)
      node[right,scale=0.75,align=left,yshift=1ex] {First element\\of the container};
    \draw[pointy arrow] (end) -- (end -| layer1)
      node[right,scale=0.75,align=left,yshift=-.8ex] {One past the\\final element};
    }

    \only<2>{
    \draw[pointy arrow] (advance) -- (advance -| layer2)
      node[right,scale=0.75,align=left] {Move to the next element};
    \draw[pointy arrow] (dereference) -- (dereference -| layer2)
      node[right,scale=0.75,align=left] {Value of current element};
    }
  \end{tikzpicture}
\end{frame}

\begin{frame}[fragile]
  \frametitle{Range based for loops}
  
  A cleaner way of iterating through containers
  
  \vspace{.75cm}
  \begin{lstlisting}[morekeywords={Employee}]
std::vector<Employee> employees;

//...

for (auto & worker : employees) {
  worker.work();
}
  \end{lstlisting}

  \vspace{.5cm}
  But it is only syntactic sugar

  \CPPEleven

\end{frame}

\begin{frame}[fragile]
  \frametitle{Streams}

  Streams are standardised input/output objects in C++

  \vspace{1em}
  It is possible to create your own input- or output stream objects that
  have other sinks and sources than the two we have used so far
  
\end{frame}

\begin{frame}[fragile]
  \fontspec{Yanone Kaffeesatz Regular}

  \begin{center}
  \begin{tikzpicture}[
    class/.style={
      draw, rounded corners,
      line width=.5pt,
      minimum width=2.5cm,
      fill=TealDrop!60!WhiteTrash
    },
    inherit/.style={
      line width=1pt,
      {Triangle[open]}-,
      shorten <=3pt
    }]
    \matrix (ios) [
      matrix of nodes,
      nodes in empty cells,
      column sep=1.5cm,
      row sep=1.2cm,
      ]
    {
      & & & & \\
      & & & & \\
      & & & & \\
      & & & & \\
      & & & & \\
      & & & & \\
    };
    \node[class] at (ios-1-3) (base) {ios\_base};
    \node[class] at (ios-2-3) (ios cl) {ios}
      edge[inherit,shorten <=1pt] (base);
    \only<1>{
    \node[class] at (ios-3-2) (istream) {istream}
      edge[inherit] (ios cl);
    \node[class] at (ios-3-4) (ostream) {ostream}
      edge[inherit] (ios cl);
    }
    \only<2->{
    \node[class] at (ios-3-2) (istream) {istream}
      edge[inherit,FeebleWeek] (ios cl);
    \node[class] at (ios-3-4) (ostream) {ostream}
      edge[inherit,FeebleWeek] (ios cl);
    }
    \node[class] at (ios-4-3) (iostream) {iostream}
      edge[inherit] (ostream)
      edge[inherit] (istream);

    \node[class] at (ios-5-1) {istringstream}
      edge[inherit] (istream);
    \node[class] at (ios-5-3) {stringstream}
      edge[inherit,shorten <=1pt] (iostream);
    \node[class] at (ios-5-5) {ostringstream}
      edge[inherit] (ostream);
    \node[class] at (ios-6-1) {ifstream}
      edge[inherit,bend right=35] (istream);
    \node[class] at (ios-6-3) {fstream}
      edge[inherit,bend right=67] (iostream);
    \node[class] at (ios-6-5) {ofstream}
      edge[inherit,bend left=35] (ostream);

    \only<2>{
      \node [above=.3cm of ostream] [scale=0.85,FeebleWeek] {virtual};
      \node [above=.3cm of istream] [scale=0.85,FeebleWeek] {virtual};

    }
  \end{tikzpicture}
  \end{center}

  \nointerlineskip
  \begin{tikzpicture}[overlay,remember picture]
    \node[anchor=north west, inner sep=.3cm] at (current page.north west)
      [font=\fontspec{Yanone Kaffeesatz Bold}\large]
      {I/O Class Hierarchy};
  \end{tikzpicture}
  
\end{frame}

\begin{frame}[fragile]
  \frametitle{\color{ICantExpress}{\color{FeebleWeek}ifstream}, {\color{FeebleWeek}ofstream}, {\color{FeebleWeek}fstream}}
  \fontspec{Yanone Kaffeesatz Regular}

  {\large Stream object for files}

  \vspace{.5cm}
  {\large\color{Tropiteal}Notable functions}\\[2pt]

  \hspace*{-.37cm}
  \begin{tikzpicture}
  \matrix (functions) [
    matrix of nodes,
    column 1/.style={anchor=base west},
    column 2/.style={anchor=base west,text width=6.5cm,inner xsep=0pt,align=left},
    column sep=.5cm
  ] {
  {\color{ICantExpress!80!WhiteTrash}(constructor)}(\std{string}) &
  opens the file with the given filename \\
  {\color{FeebleWeek}operator}<<( ... ) & writes to the file\\
  {\color{FeebleWeek}operator}>>( ... ) & reads from the file\\
  {\color{Tropiteal}open}(\std{string}) &
  opens the file with the given filename \\
  {\color{Tropiteal}close}( ) &  closes the file buffer\\
  {\color{Tropiteal}is\_open}( ) & checks if the file was opened correctly\\
  };

  \draw[Tropiteal] (functions-1-1.north west) -- (functions-1-2.north east);
  \draw[Tropiteal] (functions-2-1.north west) -- (functions-2-2.north east);
  \draw[Tropiteal] (functions-3-1.north west) -- (functions-3-2.north east);
  \draw[Tropiteal] (functions-4-1.north west) -- (functions-4-2.north east);
  \draw[Tropiteal] (functions-5-1.north west) -- (functions-5-2.north east);
  \draw[Tropiteal] (functions-6-1.north west) -- (functions-6-2.north east);
  \end{tikzpicture}

  \ClassLine

  \Library{fstream}

\end{frame}

\begin{frame}[fragile]
  \frametitle{File streams - example}
  \begin{lstlisting}[escapeinside={(*}{*)},morekeywords={ofstream}]
std::ofstream to_file_stream {"file.txt"};

if (!to_file_stream) {
  std::cerr << "Could not open file";
  return 1;
}

to_file_stream << "Hello world" << std::endl;
to_file_stream.close();
  \end{lstlisting}
\end{frame}

\begin{frame}[fragile]

  \frametitle{\color{ICantExpress}\large{\color{FeebleWeek}istringstream}, {\color{FeebleWeek}ostringstream}, {\color{FeebleWeek}stringstream}}
  \fontspec{Yanone Kaffeesatz Regular}

  {\large Stream object for the \std{string} class}

  \vspace{.75cm}
  {\large\color{Tropiteal}Notable functions}\\[2pt]

  \hspace*{-.25cm}
  \begin{tikzpicture}
  \matrix (functions) [
    matrix of nodes,
    column 1/.style={anchor=base west},
    column 2/.style={anchor=base west,text width=6.5cm,inner xsep=0pt,align=left},
    column sep=.5cm
  ] {
  {\color{ICantExpress!80!WhiteTrash}(constructor)}(\std{string}) &
  set initial value of the string object \\
  {\color{FeebleWeek}operator}<<( ... ) & writes to the string\\
  {\color{FeebleWeek}operator}>>( ... ) & reads from the string\\
  {\color{Tropiteal}str}( ) & access the underlying string object\\
  {\color{Tropiteal}str}(\std{string}) & change value of the string object\\
  };

  \draw[Tropiteal] (functions-1-1.north west) -- (functions-1-2.north east);
  \draw[Tropiteal] (functions-2-1.north west) -- (functions-2-2.north east);
  \draw[Tropiteal] (functions-3-1.north west) -- (functions-3-2.north east);
  \draw[Tropiteal] (functions-4-1.north west) -- (functions-4-2.north east);
  \draw[Tropiteal] (functions-5-1.north west) -- (functions-5-2.north east);
  \end{tikzpicture}
  
  \ClassLine

  \Library{sstream}

\end{frame}

\begin{frame}[fragile]
  \frametitle{Stream example}

  \begin{lstlisting}[escapeinside={(*}{*)}]
void sayHello(std::ostream & os)
{
  os << "Hello!" << std::endl;
}

int main()
{
  std::ofstream ofs {"file.txt"};
  sayHello(ofs);
  ofs.close();

  std::ostringstream oss;
  sayHello(oss);

  auto hello_string = oss.str();
}
  \end{lstlisting}
  
\end{frame}

\begin{frame}[fragile]
  \frametitle{Function objects}

  Objects with an overloaded call operator {\large\color{Tropiteal}()}

  \begin{overlayarea}{\textwidth}{3cm}

    \begin{onlyenv}<1>
      \begin{lstlisting}[escapeinside={(*}{*)},basicstyle=\fontsize{9pt}{9pt}\selectfont\ttfamily]
struct Greater
{
  bool operator() (const double a, const double b)
  {
    return a > b;
  }
};

auto greater = Greater {};
      \end{lstlisting}
    \end{onlyenv}

    \begin{onlyenv}<2>
      \begin{lstlisting}[escapeinside={(*}{*)},basicstyle=\fontsize{9pt}{9pt}\selectfont\ttfamily]
auto greater = [](const double a, const double b)
{
  return a > b;
};
      \end{lstlisting}
    \end{onlyenv}

  \end{overlayarea}
  
\end{frame}

\begin{frame}[fragile]
  \frametitle{The functional library}

  The STL has a uniform interface for these

  \vspace{.5cm}
  \begin{lstlisting}[basicstyle=\ttfamily]
template <class R, class... Args>
class function<R(Args...)> {...};
  \end{lstlisting}

  \vspace{.5cm}
  Which can bind to anything with the correct call operator

  \Library{functional}
  
\end{frame}

\begin{frame}[fragile]
  \frametitle{The functional library - example}

  \begin{lstlisting}[escapeinside={(*}{*)},basicstyle=\fontsize{8pt}{8pt}\selectfont\ttfamily]
void printInt(int i)
{
  std::cout << i;
}

struct IntPrint
{
  void operator() (int i)
  {
    std::cout << i;
  }
};
  \end{lstlisting} 
  \begin{lstlisting}[escapeinside={(*}{*)},basicstyle=\fontsize{8pt}{8pt}\selectfont\ttfamily,morekeywords={function,IntPrint}]
int main()
{
  std::function<void(int)> f1 = printInt;
  std::function<void(int)> f2 = IntPrint {};
  std::function<void(int)> f3 = [](int i){printInt(i);};
}
  \end{lstlisting} 
\end{frame}

\begin{frame}[fragile]
  \frametitle{Argument capture}

  \begin{onlyenv}<1>
  \begin{lstlisting}[escapeinside={(*}{*)},basicstyle=\ttfamily]
[a,&b](*\tikzmark{capture}*) (/* args */) {/* body */};
  \end{lstlisting}

  \nointerlineskip
  \begin{tikzpicture}[overlay,remember picture]
    \draw[pointy arrow] ([shift={(-2ex,-.5ex)}] capture) .. controls +(0,-1cm) and +(-1cm,0) .. +(1.5cm,-1cm)
      node[right,scale=0.95] {Capture {\color{Tropiteal}b} by reference};
    \draw[pointy arrow] ([shift={(-4ex,1.5ex)}] capture) .. controls +(0,1cm) and +(-1cm,0) .. +(1.5cm,1cm)
      node[right,scale=0.95] {Capture {\color{Tropiteal}a} by value};
  \end{tikzpicture}
  \end{onlyenv}

  \begin{onlyenv}<2>
  \begin{lstlisting}[escapeinside={(*}{*)},basicstyle=\ttfamily]
[&](*\tikzmark{capture all}*) (/* args */) {/* body */};
  \end{lstlisting}

  \nointerlineskip
  \begin{tikzpicture}[overlay,remember picture]
    \draw[pointy arrow] ([shift={(-1.3ex,-.5ex)}] capture all) -- +(0,-1cm)
      node[below,xshift=2.5cm] {Capture everything by {\color{FeebleWeek}reference}};
  \end{tikzpicture}
  \end{onlyenv}

  \begin{onlyenv}<3>
  \begin{lstlisting}[escapeinside={(*}{*)},basicstyle=\ttfamily]
[=](*\tikzmark{capture all val}*) (/* args */) {/* body */};
  \end{lstlisting}

  \nointerlineskip
  \begin{tikzpicture}[overlay,remember picture]
    \draw[pointy arrow] ([shift={(-1.3ex,1.5ex)}] capture all val) -- +(0,1cm)
      node[above,xshift=2.5cm] {Capture everything by {\color{Marty}value}};
  \end{tikzpicture}
  \end{onlyenv}

  \CPPEleven
  
\end{frame}

\begin{frame}[fragile]
  \frametitle{Closures in C++}

  \only<1>{
  A closure is the concept of storing values inside of functions

  \vspace{1cm}
  We can use the lambda function capture for that
  }

  \begin{onlyenv}<2->
  \begin{lstlisting}[escapeinside={(*}{*)},basicstyle=\fontsize{8pt}{8pt}\selectfont\ttfamily,morekeywords={function}]
std::function<std::string(std::string)>
Surround(std::string surr)
{
  return [surr](std::string expr)
  {
    return surr[0] + expr + surr[1];
  };
}

int main()
{
  auto square_brackets = Surround("[]");
  auto quotation_marks = Surround("\"\"");

  std::cout << square_brackets("Hello") << std::endl; (*\tikzmark{square hello}*)
  std::cout << quotation_marks("Hello") << std::endl; (*\tikzmark{quotation hello}*)
}
  \end{lstlisting}

  \only<3>{
  \nointerlineskip
  \begin{tikzpicture}[overlay,remember picture]
    \draw[pointy arrow] ([shift={(-4ex,1.5ex)}] square hello) -- +(0,1cm)
      node [scale=0.75,above] {Prints {\color{Marty}[Hello]}};
    \draw[pointy arrow] ([shift={(-4ex,-.5ex)}] quotation hello) .. controls +(0,-.5cm) and +(.5cm,0) .. +(-.5cm,-.75cm)
      node [scale=0.75,left] {Prints {\color{Marty}"Hello"}};
  \end{tikzpicture}
  }

  \CPPEleven

  \end{onlyenv}

\end{frame}

\begin{frame}[fragile]
  \frametitle{Algorithms}

  There is a large number of commonly used algorithms in the STL

  \vspace{1cm}
  Uniform interface that go well with the iterators

  \Library{algorithm}
  
\end{frame}

\begin{frame}[fragile]
  \frametitle{Algorithms}

  \begin{overlayarea}{\textwidth}{5cm}
  \begin{onlyenv}<1>
  \begin{lstlisting}[]
template <class Itt, class T>
  \end{lstlisting}\vspace*{-1ex}
  \begin{lstlisting}[morekeywords={Itt,T}]
Itt find(Itt begin, Itt end, const T& value);
  \end{lstlisting}

  \vspace{1em}
  Find the first element equal to {\color{Tropiteal}value} in a container,\\
  returns an iterator pointing to the element, or {\color{Tropiteal}end} if not found
  \end{onlyenv}

  \begin{onlyenv}<2>
  \begin{lstlisting}[]
template <class Itt, class Unary>
  \end{lstlisting}\vspace*{-1ex}
  \begin{lstlisting}[morekeywords={Itt,Unary}]
Unary for_each(Itt begin, Itt end, Unary f);
  \end{lstlisting}

  \vspace{1em}
  Apply the function {\color{Tropiteal}f} to every element in the container\\
  returns the final state of the function object {\color{Tropiteal}f}
  \end{onlyenv}

  \begin{onlyenv}<3>
  \begin{lstlisting}[]
template <class Itt>
  \end{lstlisting}\vspace*{-1ex}
  \begin{lstlisting}[morekeywords={Itt}]
void sort(Itt begin, Itt end);
  \end{lstlisting}

  \vspace{1em}
  Sorts the range specified by {\color{Tropiteal}begin} and {\color{Tropiteal}end}
  \end{onlyenv}

  \begin{onlyenv}<4>
  \begin{lstlisting}[]
template <class Itt, class Compare>
  \end{lstlisting}\vspace*{-1ex}
  \begin{lstlisting}[morekeywords={Itt,Compare}]
void sort(Itt begin, Itt end, Compare compare);
  \end{lstlisting}

  \vspace{1em}
  Sorts the range specified by {\color{Tropiteal}begin} and {\color{Tropiteal}end} using the supplied comparison operator
  \end{onlyenv}
  \end{overlayarea}
  
  \Library{algorithm}

\end{frame}

\begin{frame}[fragile]
  \frametitle{Less than operator is king}

  STL uses the less than operator for all comparisons

  \vspace{.75em}
  Equality:\\
  \begin{lstlisting}[basicstyle=\ttfamily]
!(a < b) and !(b < a)
  \end{lstlisting}

  \vspace{.75em}
  Inequality:\\
  \begin{lstlisting}[basicstyle=\ttfamily]
(a < b) or (b < a)
  \end{lstlisting}

  \vspace{.75em}
  It is important to implement a proper less than
  
\end{frame}

\begin{frame}[fragile]
  \frametitle{Capture in algorithms}

  \begin{lstlisting}[escapeinside={(*}{*)}]
std::vector<double> earnings {};

// ...

double total {0.};

std::for_each(earnings.begin(), earnings.end(),
  [&total](auto val){ total += val; });
  \end{lstlisting}

  \CPPEleven

\end{frame}

\begin{frame}[fragile]
  \frametitle{Smart pointers}

  We discussed the dangers of dynamic memory management using raw pointers on day two

  \vspace{1cm}
  The smart pointer library in STL is here to rescue us

  \vspace{1cm}
  Smart pointers act as if they were pointers, but provide additional functionality

  \CPPEleven

  \Library[-2cm]{memory}
  
\end{frame}

\begin{frame}[fragile]
  \frametitle{The big question}

  \begin{lstlisting}[escapeinside={(*}{*)},basicstyle=\ttfamily,morekeywords={SmartPointer,Type}]
SmartPointer<Type> p {};
auto q = p; (*\tikzmark{assign}*)
  \end{lstlisting}

  \nointerlineskip
  \begin{tikzpicture}[overlay,remember picture]
    \draw[pointy arrow] ([shift={(-4.2ex,-.5ex)}] assign) -- +(0,-1cm)
      node[below,xshift=1cm,Marty] {What happens here?};
  \end{tikzpicture}

  \CPPEleven

  \Library[-2cm]{memory}
  
\end{frame}

\begin{frame}[fragile]
  \frametitle{std{\color{ICantExpress}::}{\color{FeebleWeek}unique\_ptr}}

  The object completely own the resource

  \vspace{1em}
  Copying is disallowed, can only move

  \begin{onlyenv}<1>
  \vspace{1em}
  \begin{lstlisting}[escapeinside={(*}{*)},basicstyle=\ttfamily,morekeywords={Type}]
std::unique_ptr<Type> p {};
auto q = p; (*\tikzmark{assign}*)
  \end{lstlisting}

  \nointerlineskip
  \begin{tikzpicture}[overlay,remember picture]
    \draw[pointy arrow] ([yshift=.3ex] assign) -- +(1cm,0)
      node[right,scale=0.75,Marty] {Compile error};
  \end{tikzpicture}
  \end{onlyenv}

  \begin{onlyenv}<2>
  \vspace{1em}
  \begin{lstlisting}[escapeinside={(*}{*)},basicstyle=\ttfamily,morekeywords={Type}]
std::unique_ptr<Type> p {};
auto q = std::move(p); (*\tikzmark{move}*)
  \end{lstlisting}

  \nointerlineskip
  \begin{tikzpicture}[overlay,remember picture]
    \draw[pointy arrow] ([yshift=.3ex] move) -- +(1cm,0)
      node[right,scale=0.75,FeebleWeek] {OK};
  \end{tikzpicture}
  \end{onlyenv}
  
  \CPPEleven

  \Library[-2cm]{memory}
  
\end{frame}

\begin{frame}[fragile]
  \frametitle{std{\color{ICantExpress}::}{\color{FeebleWeek}shared\_ptr}}

  Keeps a count of all the references to the resource

  \vspace{1em}
  Only deletes the resource when all hooks are gone

  \vspace{1em}
  \begin{lstlisting}[escapeinside={(*}{*)},basicstyle=\ttfamily,morekeywords={Type}]
std::shared_ptr<Type> p {}; (*\tikzmark{construct}*)
auto q = p; (*\tikzmark{assign}*)
  \end{lstlisting}

  \nointerlineskip
  \begin{tikzpicture}[overlay,remember picture]
    \coordinate (layer1) at ([xshift=3cm] current page.north);
    \node[right,scale=0.75,yshift=.7ex] at (construct -| layer1) {.{\color{Tropiteal}use\_count}() = 1};
    \node[right,scale=0.75,yshift=.7ex] at (assign -| layer1) {.{\color{Tropiteal}use\_count}() = 2};
  \end{tikzpicture}
  
  \CPPEleven

  \Library[-2cm]{memory}
  
\end{frame}

\begin{frame}[fragile]
  \frametitle{Guideline}

  \begin{center}
  \begin{tikzpicture}
    \node [use as bounding box,font=\fontspec{Yanone Kaffeesatz Regular}] (quote) {%
      \begin{minipage}{0.85\textwidth}
        Don't use explicit new, delete and owning * pointers, except in rare
        cases encapsulated inside the implementation of low-level data structures.
      \end{minipage}
    };
    \node at (quote.south east) [font=\fontspec{Yanone Kaffeesatz Regular},Tropiteal] {Herb Sutter};
  \end{tikzpicture}
  \end{center}

\end{frame}

\begin{frame}[fragile]
  \frametitle{Other libraries}

  \begin{itemize}
    \setlength\itemsep{.5em}
    \item Random number generation
    \item Duration
    \item Regular expressions
    \item Thread support
    \item Atomic operations
  \end{itemize}
  
\end{frame}

\section{Exceptions}

\frame[plain]{\sectionpage}

\begin{frame}[fragile]
  \frametitle{Motivating example}
  
  \begin{lstlisting}[escapeinside={(*}{*)},morekeywords={size_t}]
template <class Type>
class Vector
{
public:
  Type& operator[](const std::size_t index)
  {
    if (index >= size) {



    } (*\tikzmark{exception}*)

    // ... 
  }
};
  \end{lstlisting}

  \nointerlineskip
  \begin{tikzpicture}[overlay,remember picture]
    \draw[pointy arrow] ([shift={(2cm,.5cm)}] exception) -- +(330:2cm)
      node[below right,scale=0.75] {What do you do here?};
  \end{tikzpicture}

\end{frame}

\begin{frame}[fragile]
  \frametitle{The old style}

  C\; programmers mostly use error flags for this

  \vspace{.5em}
  \begin{lstlisting}[basicstyle=\ttfamily\fontsize{9pt}{8pt}\selectfont]
template <class Type>
  \end{lstlisting} \vspace*{-1ex}
  \begin{lstlisting}[basicstyle=\ttfamily\fontsize{9pt}{8pt}\selectfont,morekeywords={Type,Vector,size_t}]
int accessVector(
  Type& res, const Vector<Type> &v, std::size_t i)
{
  if (i >= v.size()) {
    return 1;
  }

  res = v[i];
  return 0;
}
  \end{lstlisting}

  \vspace{.5em}
  {\color{Marty}Disadvantage}: They can be ignored
  
\end{frame}

\begin{frame}[fragile]
  \frametitle{Exception detection - {\color{FeebleWeek}throw}}

  \only<1>{
  When an exception is detected, we\, {\Large\color{Marty}throw}
  }

  \only<2->{
  What do we throw?

  \vspace{.5em}
  \uncover<3>{
    An object that describes the error we encountered
  }
  }
  
\end{frame}

\begin{frame}[fragile]
  \frametitle{Back to the vector}

  \begin{lstlisting}
struct OutOfRangeError {};
  \end{lstlisting}
  \begin{lstlisting}[escapeinside={(*}{*)},morekeywords={size_t,OutOfRangeError}]
template <class Type>
class Vector
{
public:
  Type& operator[](const std::size_t index)
  {
    if (index >= size) {
      throw OutOfRangeError {};
    }

    // ... 
  }
};
  \end{lstlisting}
  
\end{frame}

\begin{frame}[fragile]
  \frametitle{Exception handling - {\color{FeebleWeek}catch}}

  \begin{onlyenv}<1>
  When an exception is thrown, the code will move outwards until it is caught
  
  \begin{center}
    \begin{tikzpicture}[
      every node/.style={font=\fontspec{Yanone Kaffeesatz Regular},scale=0.8},
      scale=0.8]
      \draw[line width=.5pt] (0,0) rectangle (8cm,-5cm);
      \draw[line width=.5pt] (.5cm,-.75cm) rectangle (7.5cm,-4.5cm);
      \draw[line width=.5pt] (1cm,-1.5cm) rectangle (7cm,-4cm);
      \draw[line width=.5pt] (1.5cm,-2.25cm) rectangle (6.5cm,-3.5cm);
      \node[anchor=north west,inner sep=1pt] at (.2cm, -.1cm)    [scale=0.75] {Scope 1};
      \node[anchor=north west,inner sep=1pt] at (.7cm, -.85cm)   [scale=0.75] {Scope 2};
      \node[anchor=north west,inner sep=1pt] at (1.2cm, -1.6cm)  [scale=0.75] {Scope 3};
      \node[anchor=north west,inner sep=1pt] at (1.7cm, -2.35cm) [scale=0.75] {Scope 4};

      \node[fill,circle, inner sep=0pt, minimum size=5pt,Marty] at (2cm,-3.05cm) (throw) {};
      \node[right=.1cm of throw,FeebleWeek,yshift=.1ex] {throw}; 
      \draw[-{Stealth},Marty,line width=1pt] (throw.west) -- +(-1.7cm,0);
    \end{tikzpicture}
  \end{center}

  If the exception isn't caught, the program terminates
  \end{onlyenv}

  \begin{onlyenv}<2>

  Exception handling is done by {\color{FeebleWeek}try}-{\color{FeebleWeek}catch} blocks

  \vspace{.5em}
  \begin{lstlisting}[escapeinside={(*}{*)},basicstyle=\ttfamily\fontsize{12pt}{12pt}\selectfont,morekeywords={ExceptionType}]
try {
(*\tikzmark{detection}*)

} catch (ExceptionType & err) {
(*\tikzmark{handling}*)

}
  \end{lstlisting}

  \nointerlineskip
  \begin{tikzpicture}[overlay,remember picture]
    \node[anchor=west,shift={(.5cm,-.4ex)}] at (detection) [scale=.8] {executing code};
    \node[anchor=west,shift={(.5cm,-.4ex)}] at (handling) [scale=.8] {exception handling};
  \end{tikzpicture}

  The catch block is only executed if an exception of the corresponding type is thrown in the try block
  \end{onlyenv}

  \begin{onlyenv}<3>
    Can do multiple catch statements

    \vspace{.5em}
    \begin{lstlisting}[escapeinside={(*}{*)},morekeywords={exception,runtime_error}]
try {

} catch (std::runtime_error & e) {

} catch (std::exception & e) {

} catch (...) {

}
    \end{lstlisting}

    \vspace{.5em}
    As with if-else, first match is executed
  \end{onlyenv}
  
\end{frame}

\begin{frame}[fragile]
  \frametitle{Exception classes}

  One should implement exception classes through inheritance

  \vspace{1em}
  That way the handler doesn't have to know about everything that can go
  wrong inside the try-block

  \vspace{1em}
  Should always catch by reference to avoid slicing
  
\end{frame}

\begin{frame}[fragile]
  \frametitle{\inkey{exception}}
  \fontspec{Yanone Kaffeesatz Regular}

  {\large Base exception type in the standard library}

  \vspace{.75cm}
  {\large\color{Tropiteal}Notable functions}\\[2pt]

  \hspace*{-.37cm}
  \begin{tikzpicture}
  \matrix (functions) [
    matrix of nodes,
    column 1/.style={anchor=base west,text width=3.5cm,align=left},
    column 2/.style={anchor=base west,text width=6.5cm,inner xsep=0pt,align=left},
    column sep=.5cm
  ] {
    \inkey{virtual} \inobj{what}() & returns an explanatory cstring \\
  };

  \draw[Tropiteal] (functions-1-1.north west) -- (functions-1-2.north east);
  \end{tikzpicture}
  
  \vfill

  \ClassLine

  \Library{exception}

\end{frame}

\begin{frame}[fragile]
  \fontspec{Yanone Kaffeesatz Regular}

  \vspace{.75cm}
  \begin{center}
  \begin{tikzpicture}[
    %scale=.8, every node/.style={scale=.8},
    class/.style={
      draw, rounded corners,
      line width=.5pt,
      minimum width=3.5cm,
      fill=TealDrop!60!WhiteTrash
    },
    inherit/.style={
      line width=1pt,
      {Triangle[open]}-,
      shorten <=3pt
    }]
    \matrix (except) [
      matrix of nodes,
      nodes in empty cells,
      column sep=2cm,
      row sep=1.5cm,
      ]
    {
      & & \\
      & & \\
      & & \\
      & & \\
    };
    \node[class] at (except-1-2) (base) {exception};
    \node[class] at (except-2-1) (logic) {logic\_error}
      edge[inherit] (base);
    \node[class] at (except-2-3) (runtime) {runtime\_error}
      edge[inherit] (base);
    \node[class] at (except-3-1) {invalid\_argument}
      edge[inherit] (logic);
    \node[class] at (except-4-1) {out\_of\_range}
      edge[inherit,bend left=73] (logic);
    \node[class] at (except-3-3) {underflow\_error}
      edge[inherit] (runtime);
    \node[class] at (except-4-3) {overflow\_error}
      edge[inherit,bend right=73] (runtime);
  \end{tikzpicture}
  \end{center}

  \nointerlineskip
  \begin{tikzpicture}[overlay,remember picture]
    \node[anchor=north west, inner sep=.3cm] at (current page.north west)
      [font=\fontspec{Yanone Kaffeesatz Bold}\large]
      {Exception Class Hierarchy};
  \end{tikzpicture}

  \Library{stdexcept}
  
\end{frame}

\begin{frame}[fragile]
  \frametitle{Inheriting from \std{exception}}

  \begin{minipage}{1.2\textwidth}
  \begin{lstlisting}[escapeinside={(*}{*)},morekeywords={noexcept,override,exception}]
class Error : public std::exception
{
public:
  Error(std::string err)
    : error_message {std::move(err)} {}

  virtual const char* what() const noexcept override
  {
    return error_message.c_str();
  }

private:
  std::string error_message;
};
  \end{lstlisting}
  \end{minipage}
  
\end{frame}

\begin{frame}[fragile]
  \frametitle{Performance impact}

  Biggest complaint from C users: {\only<1>{\color{Marty}}exceptions are slow}

  \uncover<2>{
  \vspace{.75cm}
  Well, then you are using them wrong...

  \vspace{.5cm}
  Never use exceptions to steer program flow

  \vspace{.5cm}
  If the code doesn't throw, there is no overhead
  }

  
\end{frame}

\section{Where to go from here}

\frame[plain]{\sectionpage}

\begin{frame}[plain]

  \nointerlineskip
  \begin{tikzpicture}[overlay,remember picture]

    \coordinate (layer1) at ([yshift=-1cm] current page.west);

    \fill[TealDrop] (current page.north west) rectangle (current page.south east);

    \fill[WhiteTrash] decorate [decoration={random steps, segment length=5pt,amplitude=3pt}]%
      {([xshift=1cm] layer1) -- ++(1.5cm,1.5cm) -- ++(.5cm, .1cm) -- ([xshift=4.1cm] layer1)} -- cycle;

    \fill[Tropiteal] (layer1) rectangle (current page.south east);
    \node[anchor=base west,inner sep=0pt,scale=1.2] at ([xshift=5cm] layer1) [WhiteTrash] {Only the tip of the iceberg};
  \end{tikzpicture}
  
\end{frame}

\begin{frame}[fragile]
  \frametitle{boost}

  The best multi-functional library for C++ out there

  \vspace{.5em}
  \begin{columns}
  \column{.4\textwidth}
  \begin{itemize}
    \setlength\itemsep{.5em}
    \item Filesystem
    \item Iostreams
    \item Iterator
    \item Multi-Array
    \item Multiprecision
  \end{itemize}
  \column{.5\textwidth}
  \begin{itemize}
    \setlength\itemsep{.5em}
    \item Phoenix
    \item Program Options
    \item Property Tree
    \item System
    \item ...and many more
  \end{itemize}
  \end{columns}
  
\end{frame}

\begin{frame}[fragile]
  \frametitle{Fun with templates}

  There is so much one can do with templates

  \vspace{.5em}
  \begin{itemize}
    \setlength\itemsep{.75em}
    \item Metaprogramming
    \item Variadic templates
    \item Expression templates
  \end{itemize}
  
\end{frame}

\begin{frame}[fragile]
  \frametitle{Design patterns}

  Design patterns are simple and elegant solutions to specific problems
  one often encounters when coding

  \vspace{1cm}
  Design patterns are often language independent, and is an
  indispensable tool to any good programmer
  
\end{frame}

\begin{frame}[fragile]
  \frametitle{All the details}

  {\color{Tropiteal}cppreferen.com} \:is my most visited webpage

  \vspace{1cm}
  Scott Meyers' "Effective ..." series is {\large really} good

  \vspace{1cm}
  Herb Sutter's "Guru of the Week" is very enlightening


  
\end{frame}

\begin{frame}[fragile]
  \frametitle{Language development}

  There is a lot happening these days

  \vspace{1cm}
  \hspace*{-.5cm}
  \begin{tikzpicture}
    \node (github) {\includegraphics[width=2cm]{Figures/Octocat.png}};
    \node [right=0pt of github] [Tropiteal,font=\ttfamily] {isocpp/CppCoreGuidelines};
  \end{tikzpicture}
  
\end{frame}

\section{Programming Practices}

\frame[plain]{\sectionpage}

\begin{frame}
  \frametitle{Good Programming Practices}

  \begin{itemize}
    \setlength\itemsep{0.75em}
    \item prefer \std{array} to static arrays
    \item prefer \std{vector} to dynamic arrays
    \item prefer algorithms calls to hand-written loops
    \item avoid the use of "dumb" pointers
  \end{itemize}
  
\end{frame}

\begin{frame}
  \frametitle{Good Programming Practices}

  When it comes to exceptions

  \vspace{.5em}
  \begin{itemize}
    \setlength\itemsep{0.75em}
    \item throw by value
    \item catch by reference
    \item re-throw if necessary
    \item inherit from \std{exception}
  \end{itemize}
  
\end{frame}

\section{Recap}

\frame[plain]{\sectionpage}

\begin{frame}
  \frametitle{Recap Day 5}

  \begin{itemize}
    \setlength\itemsep{0.75em}
    \item Lots of useful tools in the STL
    \vspace{.25em}
    \begin{itemize}
      \setlength\itemsep{.25em}
      \item Containers
      \item Streams
      \item Smart pointers
      \item ++
    \end{itemize}
    \item Iterators give a generic way of working with all the different containers
    \item Algorithms for all your programming needs
  \end{itemize}

\end{frame}

\begin{frame}
  \frametitle{Recap Day 5}

  \begin{itemize}
    \setlength\itemsep{0.75em}
    \item Use {\color{FeebleWeek}throw} to signal an exception
    \item {\color{FeebleWeek}try}-{\color{FeebleWeek}catch} blocks to handle them
    \item One can inherit from \std{exception} to\\make a uniform exception interface
  \end{itemize}

\end{frame}

\begin{frame}[plain]

  \nointerlineskip
  \begin{tikzpicture}[overlay,remember picture]
    \fill[ICantExpress] (current page.north east) rectangle (current page.south west);
    \node[scale=1.3] at (current page.center) [font=\LARGE, ConradGold] {%
      \begin{varwidth}{.5\textwidth}
      \scalebox{0.97}{May the}\\
      \scalebox{1.3}{FORCE}\\
      be with\\[5pt]
      \scalebox{1.95}{YOU}
      \end{varwidth}
    };
  \end{tikzpicture}
  
\end{frame}

\end{document}
