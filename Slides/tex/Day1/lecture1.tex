\documentclass[14pt,a4paper,dvipsnames,usenames]{beamer}
\usepackage[T1]{fontenc}
\usepackage{url}
\usepackage{xcolor}
\usepackage{amsmath,varwidth}
\usepackage{tikz,tikz-uml,xparse}
\usepackage{listings,algpseudocode}
\usepackage[quiet]{mathspec}

\setmainfont[
  ItalicFont={Yanone Kaffeesatz Light},
  Scale=1.3,
  LetterSpace=2.0
]{Yanone Kaffeesatz Bold}

\setmonofont{Hack}

% --------------------- %
% Beamer theme settings %
% --------------------- %

\usetheme{CodeCourse}
\setbeamertemplate{navigation symbols}{} %Remove beamer navigation symbols
\setbeamertemplate{bibliography item}{\insertbiblabel}

% ---------------------------- %
% Colour definitions for theme %
% ---------------------------- %

\definecolor{Tropiteal}{RGB}{0,168,198}
\definecolor{TealDrop}{RGB}{64,192,203}
\definecolor{WhiteTrash}{RGB}{249,242,231}
\definecolor{AtomicBikini}{RGB}{174,226,57}
\definecolor{FeebleWeek}{RGB}{143,190,0}
\definecolor{ICantExpress}{RGB}{28,20,13}
\definecolor{Love}{RGB}{209,67,52}
\definecolor{UnrealFoodPills}{RGB}{250,105,0}
\definecolor{Marty}{RGB}{250,42,0}
\definecolor{ConradGold}{RGB}{255,221,0}

\definecolor{sbase03}{HTML}{002B36}
\definecolor{sbase02}{HTML}{073642}
\definecolor{sbase01}{HTML}{586E75}
\definecolor{sbase00}{HTML}{657B83}
\definecolor{sbase0}{HTML}{839496}
\definecolor{sbase1}{HTML}{93A1A1}
\definecolor{sbase2}{HTML}{EEE8D5}
\definecolor{sbase3}{HTML}{FDF6E3}
\definecolor{syellow}{HTML}{B58900}
\definecolor{sorange}{HTML}{CB4B16}
\definecolor{sred}{HTML}{DC322F}
\definecolor{smagenta}{HTML}{D33682}
\definecolor{sviolet}{HTML}{6C71C4}
\definecolor{sblue}{HTML}{268BD2}
\definecolor{scyan}{HTML}{2AA198}
\definecolor{sgreen}{HTML}{859900}

% -------------------------------------------------- %
% Beamer notes to be itemized rather than enumerated %
% -------------------------------------------------- %

\makeatletter
\def\beamer@setupnote{%
  \gdef\beamer@notesactions{%
    \beamer@outsideframenote{%
      \beamer@atbeginnote%
      \beamer@notes%
      \ifx\beamer@noteitems\@empty\else
      \begin{itemize}\itemsep=0pt\parskip=0pt%
        \beamer@noteitems%
      \end{itemize}%
      \fi%
      \beamer@atendnote%
    }%
    \gdef\beamer@notesactions{}%
  }
}
\makeatother

% -------------------------- %
% Standard LaTeX definitions %
% -------------------------- %

\DeclareGraphicsExtensions{.pdf}
\renewcommand\textbullet{\ensuremath{\bullet}}

% ----------------------------------- %
% Various tikz styles and definitions %
% ----------------------------------- %

\usetikzlibrary{calc,intersections,positioning,matrix,chains,scopes}
\usetikzlibrary{decorations.pathmorphing,decorations.pathreplacing,shapes}
\newcommand{\tikzmark}[1]{\tikz[overlay,remember picture] \node (#1) {};}

\tikzumlset{fill class=RoyalBlue!20}

\tikzstyle{startnode} = [circle, draw, fill=Bittersweet!20, minimum size=10pt, inner sep = 0]
\tikzstyle{stopnode} = [startnode]
\tikzstyle{process} = [rectangle, minimum width=2cm, minimum height=.7cm, text centered, draw, fill=OliveGreen!20, align=center, font=\scriptsize]
\tikzstyle{dangerprocess} = [process, fill=BrickRed!20, draw=BrickRed, thick]
\tikzstyle{flowarrow} = [->, >=stealth, thick]

% ---------------------------------- %
% My default listings C++ code style %
% ---------------------------------- %

\lstset{%
  language=C++, basicstyle=\scriptsize\ttfamily, 
  keywordstyle=\color{OliveGreen}, identifierstyle=\color{RoyalBlue}, 
  commentstyle=\color{Brown}, stringstyle=\color{Bittersweet}, showstringspaces=false,  
  breaklines=true, prebreak=\mbox{{\color{Bittersweet}\tiny\ $\hookleftarrow$}}, 
  postbreak=\mbox{{\color{Bittersweet}\tiny$\to$\ }}, tabsize=5,
  morekeywords={%
    MyClass,unique_ptr,shared_ptr,weak_ptr,auto_ptr,
    make_unique, make_shared, default_delete,
    dynamic_pointer_cast, const_pointer_cast,
    scoped_ptr,QSharedPointer,QScopedPointer,QWeakPointer,
    ftring, SmartPtr, Shape, Circle, ShapeFactory, CircleFactory,
    Derived, Base
  }
}

\def\inline{\lstinline[basicstyle=\ttfamily\normalsize]}

% ---------------------------------------------------------------------------------------------------- %
% Various command for inline code instead of lstinline so that I don't have to update the keyword list %
% ---------------------------------------------------------------------------------------------------- %

\newcommand{\object}[1]{{\ttfamily \color{OliveGreen}#1}}
\newcommand{\function}[1]{{\ttfamily \color{RoyalBlue}#1}}
\newcommand{\std}[1]{{\ttfamily {\color{RoyalBlue} std::}{\color{OliveGreen}#1}}}
\newcommand{\boost}[1]{{\ttfamily {\color{RoyalBlue} boost::}{\color{OliveGreen}#1}}}

\newcommand{\uniqueptr}{\std{unique\_ptr}}
\newcommand{\sharedptr}{\std{shared\_ptr}}
\newcommand{\weakptr}{\std{weak\_ptr}}
\newcommand{\autoptr}{\std{auto\_ptr}}

\usepackage{array}

% ------------------------------------------------ %
% tikz functions to place markers on beamer slides %
% ------------------------------------------------ %

\makeatletter
\tikzset{%
  remember picture with id/.style={%
    remember picture,
    overlay,
    save picture id=#1,
  },
  save picture id/.code={%
    \edef\pgf@temp{#1}%
    \immediate\write\pgfutil@auxout{%
      \noexpand\savepointas{\pgf@temp}{\pgfpictureid}}%
  },
  if picture id/.code args={#1#2#3}{%
    \@ifundefined{save@pt@#1}{%
      \pgfkeysalso{#3}%
    }{
      \pgfkeysalso{#2}%
    }
  }
}

\def\savepointas#1#2{%
  \expandafter\gdef\csname save@pt@#1\endcsname{#2}%
}

\def\tmk@labeldef#1,#2\@nil{%
  \def\tmk@label{#1}%
  \def\tmk@def{#2}%
}

\tikzdeclarecoordinatesystem{pic}{%
  \pgfutil@in@,{#1}%
  \ifpgfutil@in@%
    \tmk@labeldef#1\@nil
  \else
    \tmk@labeldef#1,(0pt,0pt)\@nil
  \fi
  \@ifundefined{save@pt@\tmk@label}{%
    \tikz@scan@one@point\pgfutil@firstofone\tmk@def
  }{%
  \pgfsys@getposition{\csname save@pt@\tmk@label\endcsname}\save@orig@pic%
  \pgfsys@getposition{\pgfpictureid}\save@this@pic%
  \pgf@process{\pgfpointorigin\save@this@pic}%
  \pgf@xa=\pgf@x
  \pgf@ya=\pgf@y
  \pgf@process{\pgfpointorigin\save@orig@pic}%
  \advance\pgf@x by -\pgf@xa
  \advance\pgf@y by -\pgf@ya
  }%
}
\newcommand\tikzsupermark[2][]{%
\tikz[remember picture with id=#2] #1;}
\makeatother

\newcommand{\MarkText}[3][]{%
\begin{tikzpicture}[overlay,remember picture]%
  \path (pic cs:#2,{(0,\paperheight)}) +(0,.7\baselineskip) coordinate (a);
  \path (pic cs:#3,{(0,-\paperheight)}) +(0,-.3\baselineskip) coordinate (b);
  \draw [ultra thick,
    if picture id={#2}{red}{line width=1cm,green,opacity=.5},
    #1]
  (a) -- ($(b)!(a)!($(b)+(1,0)$)$);
\end{tikzpicture}%
}%

\newcommand{\tikzmark}[1]{\tikz[overlay,remember picture] \coordinate (#1);}

% -------------- %
% Special frames %
% -------------- %

\newcommand{\LiveFrame}{%
\begin{frame}[plain]
  \nointerlineskip
  \begin{tikzpicture}[overlay,remember picture]
    \fill[FeebleWeek] (current page.north east) rectangle (current page.south west);
    \node at ([yshift=-.25em] current page.center) [font=\LARGE, WhiteTrash] {Live Example};
  \end{tikzpicture}
\end{frame}
}

\newcommand{\CPPEleven}{%
\nointerlineskip
\begin{tikzpicture}[overlay,remember picture]
  \node[anchor=south east] at ( current page.south east) [color=Tropiteal] {\{C++11\}};
\end{tikzpicture}
}

\newcommand{\CPPFourteen}{%
\nointerlineskip
\begin{tikzpicture}[overlay,remember picture]
  \node[anchor=south east] at ( current page.south east) [color=Tropiteal] {\{C++14\}};
\end{tikzpicture}
}

\newcommand{\BadPractice}{%
\nointerlineskip
\begin{tikzpicture}[overlay,remember picture]
  \node[anchor=south east] at ( current page.south east) [color=Marty] {Bad Practice};
\end{tikzpicture}
}

\newcommand{\ClassLine}{%
\nointerlineskip
\begin{tikzpicture}[overlay,remember picture]
  \draw[transform canvas={yshift=-1.7cm},thin] (current page.north west) -- (current page.north east);
\end{tikzpicture}
}

\newcommand{\Library}[2][0pt]{%
\nointerlineskip
\begin{tikzpicture}[overlay,remember picture]
  \node[anchor=south east] at ([xshift=#1] current page.south east) [color=Marty,scale=0.9]
    [font=\fontspec{Yanone Kaffeesatz Regular}] {\#include<#2>};
\end{tikzpicture}
}

% ----------------------------------------- %
% Various Commands for inline codecolouring %
% ----------------------------------------- %

\newcommand{\inobj}[1]{%
  {\color{Tropiteal}#1}}

\newcommand{\inkey}[1]{%
  {\color{FeebleWeek}#1}}

\newcommand{\innorm}[1]{%
  {\color{ICantExpress}#1}}

\newcommand{\std}[1]{%
  \inobj{std}\innorm{::}\inkey{#1}}

% ----------------------------------------------------- %
% Various tikz based functions to style up the document %
% ----------------------------------------------------- %

\newcommand{\underlinemark}[3][line width=1.5pt, FeebleWeek]{%
\draw[transform canvas={yshift=-.15cm}, #1]
  ([xshift=-1pt] #2) -- ([xshift=2pt] #3);
}

\newcommand{\StrikeThrough}[2][Marty]{%
  \tikz[baseline=(strokethrough.base)]{%
    \node[inner sep=0pt, outer sep=0pt] (strokethrough) {\strut{}#2};
    \draw[line width=2pt,#1] ([xshift=-3pt] strokethrough.west) -- ([xshift=3pt] strokethrough.east);
  }
}

\newcommand{\Highlight}[3][opacity=0.4,AtomicBikini]{%
\nointerlineskip
\begin{tikzpicture}[overlay,remember picture]
  \path (pic cs:#2,{(0,0)}) coordinate (a);
  \path (pic cs:#3,{(0,0)}) coordinate (b);
  \fill[#1] (a -| current page.north west) -- ++(0,1.4ex) -- ++(\paperwidth,0)
  -- (b -| current page.north east) -- ++(0,-.4ex) -- ++(-\paperwidth,0) -- cycle;
\end{tikzpicture}
}

\newcommand{\Dimtext}[2]%
{
  { \transparent{0.7}
  \begin{tikzpicture}[overlay, remember picture]
    \fill[white] ( #1 -| current page.north west) -- ++(0,.8em) -- ++(\paperwidth,0) -- (#2 -| current page.north east)
   -- ++(0,-.5em) -- ++(-\paperwidth,0) -- cycle;
  \end{tikzpicture}
  }
}

% -------------- %
% Array commands %
% -------------- %

\newcolumntype{L}[1]{>{\raggedright\let\newline\\\arraybackslash\hspace{0pt}}m{#1}}
\newcolumntype{C}[1]{>{\centering\let\newline\\\arraybackslash\hspace{0pt}}m{#1}}
\newcolumntype{R}[1]{>{\raggedleft\let\newline\\\arraybackslash\hspace{0pt}}m{#1}}


\usefonttheme{serif}

\pgfdeclarelayer{background}
\pgfdeclarelayer{foreground}
\pgfsetlayers{background,main,foreground}

%\setbeameroption{show notes}

\title[C++ Day1]{Introduction to the C++\newline{}Programming Language\newline{}\newline{}\fontsize{16pt}{16pt}\selectfont{}Day 1}
\author{\texorpdfstring{%
    Aleksandra Rylund Glesaaen\newline\fontsize{12pt}{12pt}\selectfont\texttt{aleksandra@glesaaen.com}%
  }{%
    Aleksandra Rylund Glesaaen}}
\date{September 28th 2015}

\begin{document}

\begin{frame}
\titlepage
\end{frame}

\section{Introduction}

\frame[plain]{\sectionpage}

\begin{frame}
  \frametitle{Who am I?}

  \begin{itemize}
    \setlength\itemsep{1em}
    \item Aleksandra Rylund \tikzmark{name begin}Glesaaen\tikzmark{name end}
    \item "A lattice guy"
    \item A mostly self-taught obsessive perfectionist
    \item Someone who believes IT should be taught by IT people
      {\fontsize{8pt}{8pt}\selectfont{}\{so you should all take a different course\}}
  \end{itemize}

  \uncover<2>{
  \begin{tikzpicture}[overlay,remember picture]
    \underlinemark{name begin}{name end}
    \draw[<-,>=stealth] ($(name end) + (.2,-.1)$) .. controls +(1,0) and +(0,1) .. +(3,1)
      node[below, font=\footnotesize, align=center] {weird\\Norwegian\\name};
  \end{tikzpicture}}

\end{frame}

\begin{frame}[fragile]
  \frametitle{Course material}
  
  \hspace*{-.5cm}
  \begin{tikzpicture}
    \node (github) {\includegraphics[width=2cm]{Figures/Octocat.png}};
    \node [right=0pt of github] [font=\ttfamily] {Irubataru/cpp-lecture-2015};
  \end{tikzpicture}
  
\end{frame}

\begin{frame}
  \frametitle{What will we learn?}

  \begin{itemize}
    \setlength\itemsep{.5em}
    \item Basic C++ syntax \uncover<2->{\hspace{.5cm}{\only<2>{\color{Tropiteal}}\only<3>{\color{Marty}}(today)}}
    \item Control structures \uncover<2->{\hspace{.5cm}{\only<2>{\color{Tropiteal}}\only<3>{\color{Marty}}(today)}}
    \item Functions \uncover<2->{\hspace{.5cm}{\color{Tropiteal}(Tuesday)}}
    \item Structs and classes \uncover<2->{\hspace{.25cm}{\color{Tropiteal}(Wednesday and Thursday)}}
    \item Templates and STL \uncover<2->{\hspace{.5cm}{\color{Tropiteal}(Thursday and Friday)}}
    \item Exceptions \uncover<2->{\hspace{.5cm}{\color{Tropiteal}(Friday)}}
  \end{itemize}
\end{frame}

\begin{frame}[plain]
  \nointerlineskip
  \begin{tikzpicture}[overlay,remember picture]
    \only<1>{
    \fill[FeebleWeek] (current page.north east) rectangle (current page.south west);
    \node at ([yshift=-.25em] current page.center) [font=\Large, WhiteTrash] (warn1) {%
      Look at the person sitting next to you
    };
    }
    \only<2>{
    \fill[Marty] (current page.north east) rectangle (current page.south west);
    \node at ([yshift=-.25em] current page.center) [font=\LARGE, WhiteTrash] (warn2) {%
      One of you won't make it
    };
    }
  \end{tikzpicture}
\end{frame}

\begin{frame}[fragile]
  \frametitle{Today's topics}

  \tableofcontents
  
\end{frame}

\begin{frame}
  \frametitle{What is C?}

  \begin{minipage}{0.85\textwidth} \large
    A relic from the 70s, 80s and 90s that has had a huge influence on most\\modern programming languages.
  \end{minipage}

\end{frame}

\begin{frame}
  \frametitle{What is C?}

  Notable features

  \vspace{.5em}

  \begin{itemize}
    \setlength\itemsep{.5em}
    \item It is a procedural language
    \item It is statically typed
    \item It has low-level access to memory
    \item Readable syntax {\fontsize{8pt}{8pt}\selectfont{}(in my opinion)}
  \end{itemize}

\end{frame}

\begin{frame}
  \frametitle{What is C++?}

  \begin{tikzpicture}
    \node (quote) [align=left,font=\Large] {%
      Anything you can do, I can do better.\\
      I can do anything better than you.
    };
    \node [right=of quote.south, shift={(.5cm,-.2cm)}, color=TealDrop] {\emph{Annie Get Your Gun}};
  \end{tikzpicture}

  \vspace{1em}

  \uncover<2>{
    C++ is a language built on top of the C programming language
  }

\end{frame}

\begin{frame}
  \frametitle{What is C++?}

  Additional features

  \vspace{.5em}

  \begin{itemize}
    \setlength\itemsep{.5em}
    \item Classes and inheritance
    \item Templates
    \item Exceptions
    \item A huge standard library
    \item ... and it is in active development
  \end{itemize}

\end{frame}

\begin{frame}[fragile]
  \frametitle{Versions of C++}

  C++ is constantly evolving, hence there are many standards

  \vspace{.5em}
  \begin{itemize}
    \setlength\itemsep{.5em}
    \item C++98
    \item { \only<2>{\color{Marty}} C++03}\tikzmark{cpp03}
    \item { \only<3>{\color{Tropiteal}}C++11}\tikzmark{cpp11}
    \item { \only<3>{\color{Tropiteal}}C++14}\tikzmark{cpp14}
    \item C++17
  \end{itemize}

  \nointerlineskip
  \begin{tikzpicture}[overlay,remember picture]
    \only<2>{
      \draw[line width=1pt,<-,>=stealth] ([shift={(.2em,.4ex)}] cpp03) -- +(1.5cm,0)
        node[right,scale=0.8] {Current standard of many compilers};
    }
    \only<3>{
    \draw[transform canvas={xshift=.75em},decorate,decoration={brace,amplitude=5pt},line width=2pt]
      ([yshift=1.25ex] cpp11) -- ([yshift=-.25ex] cpp14)
      coordinate [midway] (brace mid);
    \node[right=of brace mid,scale=.8] {We will use these};
    }
  \end{tikzpicture}
  
\end{frame}

\begin{frame}[fragile]
  \frametitle{The slides}

  \begin{center}
  \begin{columns}
    \column{.45\textwidth}
    Things that are bad practice will be marked
    \vspace{.2cm}
    \begin{center}
    \begin{tikzpicture}
      \node[Marty,scale=1.2] {Bad Practice};
    \end{tikzpicture}
    \end{center}
    \vspace{1cm}
    \column{.45\textwidth}
    Things new to C++11 or C++14 will me marked
    \begin{center}
    \begin{tikzpicture}
      \node[Tropiteal,scale=1.2] {\{C++11\}};
    \end{tikzpicture}
    \end{center} \vspace*{-1.5cm}
    \begin{center}
    \begin{tikzpicture}
      \node[Tropiteal,scale=1.2] {\{C++14\}};
    \end{tikzpicture}
    \end{center}
  \end{columns}
  \end{center}
  
\end{frame}

\section{Syntax and structure}

\frame[plain]{\sectionpage}

\begin{frame}[fragile]
  \frametitle{Hello World in C++}

  \begin{lstlisting}[escapeinside={(*}{*)}]
#include<iostream> (*\tikzmark{include}*)

int main() (*\tikzmark{main-func}*)
{
  std::cout(*\tikzmark{cout}*) << "Hello World"(*\tikzmark{string}*) << std::endl;(*\tikzmark{endl}*)
}
  \end{lstlisting}

  \only<2>{
  \nointerlineskip
  \begin{tikzpicture}[overlay,remember picture, every node/.style={scale=0.8}, every path/.style={line width=1pt}]
    \draw[<-,>=stealth] ([shift={(-2,.8em)}] include) .. controls +(0,.75cm) and +(-.75,0) .. +(2,1)
      node [right] {Include external libraries};
    \draw[<-,>=stealth] ([yshift=.2em] main-func) -- +(1,0) node [right] {The main function};
    \draw[<-,>=stealth] ([shift={(-1cm,-.2cm)}] cout) -- +(0,-1) node [below, xshift=2cm] {Built in terminal stream object};
    \draw[<-,>=stealth] ([shift={(-1cm,-.2cm)}] string) .. controls +(0,-1) and +(-.75,0) .. +(1,-.5)
      node [right] {A string literal};
  \end{tikzpicture}
  }
 
\end{frame}

\begin{frame}
  \frametitle{A program in C++}

  In essence all C++ programs consist of two things

  \vspace{1em}

  \begin{enumerate}
    \setlength\itemsep{.5em}
    \item Sentences
    \item Blocks
  \end{enumerate}
 
\end{frame}

\begin{frame}[fragile]
  \frametitle{A program in C++}

  \Highlight{sentence begin}{sentence begin}

  \begin{enumerate}
    \item Sentences
  \end{enumerate}

  \vspace{1em}

  \begin{lstlisting}[escapeinside={(*}{*)}]
#include<iostream>

int main()
{
  (*\tikzsupermark{sentence begin}*)std::cout << "Hello World" << std::endl;
}
  \end{lstlisting}

  \vspace{1em}

  \begin{minipage}[b][2ex]{\textwidth}
    A complete instruction ending with a \hskip5pt {\large\color{Tropiteal};}
  \end{minipage}
 
\end{frame}

\begin{frame}[fragile]
  \frametitle{A program in C++}

  \Highlight{block begin}{block end}

  \begin{enumerate}
    \setcounter{enumi}{1}
    \item Blocks
  \end{enumerate}

  \vspace{1em}

  \begin{lstlisting}[escapeinside={(*}{*)}]
#include<iostream>

int main()
(*\tikzsupermark{block begin}*){
  std::cout << "Hello World" << std::endl;
(*\tikzsupermark{block end}*)}
  \end{lstlisting}

  \vspace{1em}

  \begin{minipage}[b][2ex]{\textwidth}
    A group of instructions inside of a pair of \hskip5pt {\large\color{Tropiteal}\{\}}
  \end{minipage}

\end{frame}

\section{Types and variables}

\frame[plain]{\sectionpage}

\begin{frame}[fragile]
  \frametitle{What is a variable?}

  A variable is simply a named location in memory

  \bigskip

  \begin{lstlisting}[escapeinside={(*}{*)}]
int main()
{
  int n = 5;
  std::cout << &n << std::endl; //(*\tikzmark{membegin}*)0x7fff27ea5464(*\tikzmark{memend}*)
}
  \end{lstlisting}

  \begin{tikzpicture}[overlay,remember picture]
    \underlinemark{membegin}{memend}
    \coordinate (memcenter) at ($(membegin) !.5! (memend)$);
    \draw[<-,>=stealth] ([yshift=-.25cm] memcenter) -- +(0,-1)
      node[below,font=\footnotesize,align=center] {name of location\\in memory};
  \end{tikzpicture}
 
\end{frame}

\begin{frame}
  \frametitle{What is a variable?}

  Its data type tells the compiler two important things

  \vspace{1em}

  \begin{enumerate}
    \setlength\itemsep{1em}
    \item How much memory the variable needs
    \item The allowed operations on the variable
  \end{enumerate}
\end{frame}

\begin{frame}[fragile]
  \frametitle{Variable initialisation}

  \begin{onlyenv}<1-5>
  \begin{lstlisting}[escapeinside={(*}{*)}, basicstyle=\ttfamily\large]
(*\tikzmark{typebegin}*)double(*\tikzmark{typeend}*) (*\tikzmark{namebegin}*)rate_of_decay(*\tikzmark{nameend}*) = (*\tikzmark{litbegin}*)0.75(*\tikzmark{litend}*);
  \end{lstlisting}
  \end{onlyenv}

  \begin{onlyenv}<6>
  \begin{lstlisting}[escapeinside={(*}{*)}, basicstyle=\ttfamily\large]
double rate_of_decay (0.75);
  \end{lstlisting}
  \end{onlyenv}
 
  \begin{onlyenv}<7>
  \begin{lstlisting}[escapeinside={(*}{*)}, basicstyle=\ttfamily\large]
double rate_of_decay {0.75};
  \end{lstlisting}
  \end{onlyenv}

  \begin{onlyenv}<8>
  \begin{lstlisting}[escapeinside={(*}{*)}, basicstyle=\ttfamily\large]
double rate_of_decay;
  \end{lstlisting}
  \end{onlyenv}

  \nointerlineskip
  \begin{tikzpicture}[overlay,remember picture]
    \only<2>{
      \underlinemark[line width=1.5pt, Marty]{typebegin}{typeend};
      \coordinate (typecenter) at ($(typebegin) !.5! (typeend)$);
      \draw[<-,>=stealth,line width=1pt] ([yshift=-.25cm] typecenter) .. controls +(0,-.75) and +(-.75,0) ..  +(1,-1)
        node[right, anchor=north west, yshift=.8em, scale=0.9] {%
          \begin{minipage}{9cm}
            Data type of the variable

            \medskip
            Tells the compiler:

            \begin{itemize}
              \item \lstinline!rate_of_decay! \hskip3pt is a \hskip3pt \lstinline!double!
              \item It needs 8 bytes of memory {\fontsize{8pt}{8pt}\selectfont{}\{usually\}}
            \end{itemize}
          \end{minipage}
        };
    }
    \only<3>{
      \underlinemark[transform canvas={yshift=-.1cm}, line width=1.5pt, Marty]{namebegin}{nameend};
      \coordinate (namecenter) at ($(namebegin) !.5! (nameend)$);
      \draw[<-,>=stealth,line width=1pt] ([yshift=.8cm] namecenter) -- +(0,.5)
        node[above, anchor=south, scale=0.9] {%
          \begin{minipage}{8cm}
            Name of the variable

            \smallskip
            Your hook to the newly allocated memory
          \end{minipage}
        };
    }
    \only<4>{
      \underlinemark[line width=1.5pt, Marty]{litbegin}{litend};
      \coordinate (litcenter) at ($(litbegin) !.5! (litend)$);
      \draw[<-,>=stealth,line width=1pt] ([yshift=.-.25cm] litcenter) ..controls +(0,-.75) and +(.75,0) .. +(-1,-1)
        node[left, anchor=north east, yshift=.8em, scale=0.9] {%
          \begin{minipage}{6cm}
            \flushright
            A \hspace{.25em} \lstinline!double! \hspace{.25em} literal which will in this case be placed in the allocated memory
            slot
          \end{minipage}
        };
    }
  \end{tikzpicture}

  \nointerlineskip
  \begin{tikzpicture}[overlay,remember picture, every node/.style={scale=1.25}]
    \only<5>{
      \node at ([yshift=1.2cm] current page.center) {Assignment Style};
    }
    \only<6>{
      \node at ([yshift=1.2cm] current page.center) {C++03 Constructor Style};
    }
    \only<7>{
      \node at ([yshift=1.2cm] current page.center) {C++11 Constructor Style};
    }
    \only<8>{
      \node at ([yshift=1.2cm] current page.center) {Undefined Declaration Style};
    }
  \end{tikzpicture}

  \only<8>{
    \nointerlineskip
    \begin{tikzpicture}[overlay,remember picture]
      \node[anchor=south east] at ( current page.south east) [color=Marty, align=center]
      {Bad Practice\\[-.2em]{\fontsize{8pt}{8pt}\selectfont\{for built in types\}}};
    \end{tikzpicture}
  }

  \only<7>{
    \nointerlineskip
    \begin{tikzpicture}[overlay,remember picture]
      \node[anchor=south east] at ( current page.south east) [color=Tropiteal, align=center]
      {\{C++11\}};
    \end{tikzpicture}
  }
 
\end{frame}

\begin{frame}[fragile]
  \frametitle{Naming variables}

  \begin{center}
    \verb/([_a-zA-Z])[_a-zA-Z0-9]*/
  \end{center}

  \only<2>{

    \vspace{1em}
    Exceptions

    \vspace{.5em}
    \begin{itemize}
      \setlength\itemsep{.5em}
      \item Keywords defined by the language
      \item Names starting with \_ or \_\_ are reserved
    \end{itemize}

    \vspace{.5em}
    Keywords: \lstinline!int!, \lstinline!float!, \lstinline!while!, \lstinline!const!, \lstinline!false!, ...
  }
 
\end{frame}

\begin{frame}[fragile]
  \frametitle{Naming variables}

  One should find a system and stick to it

  \vspace{1em}

  \begin{overlayarea}{\textwidth}{4cm}
  \begin{onlyenv}<1>
  E.g. mixed style

  \vspace{.5em}
  {\renewcommand{\arraystretch}{1.5}
    \begin{tabular}{ll}
    Variables: & \lstinline!snake_case! \\
    Functions: & \lstinline!mixedCase! \\
    Classes: & \lstinline!CamelCase!
    \end{tabular}
  }
  \end{onlyenv}

  \begin{onlyenv}<2>
  E.g. Stroustrup style

  \vspace{.5em}
  {\renewcommand{\arraystretch}{1.5}
    \begin{tabular}{ll}
    Variables: & \lstinline!snake_case! \\
    Functions: & \lstinline!Mixed_case! \\
    Classes: & \lstinline!Mixed_case!
    \end{tabular}
  }
  \end{onlyenv}
  \end{overlayarea}

\end{frame}

\begin{frame}[fragile]
  \frametitle{Built in data types}

  Basically four built in data types in C++

  \vspace{1em}
  {\renewcommand{\arraystretch}{1.5}
    \begin{tabular}{|l|l|l|} \hline
      Boolean: & \lstinline!bool! & \lstinline!true!, \lstinline!false!\\
      Character: & \lstinline!char! & \lstinline!'c'!, \lstinline!'#'!, \lstinline!'7'!, ... \\
      Integer: & \lstinline!int! & \lstinline!0!, \lstinline!12!, \lstinline!-42!, ... \\
      Floating point: \hspace{.2em} & \lstinline!float! \hspace{.2em} & \lstinline!0.0!, \lstinline!1.33!, \lstinline!-4.11!, ... \\ \hline
    \end{tabular}
  }
 
\end{frame}

\begin{frame}[fragile]
  \frametitle{Type qualifiers}

  \large Type qualifiers manipulate the built in types

\end{frame}

\begin{frame}[fragile]
  \frametitle{Type qualifiers}

  Manipulate memory size
  \begin{center}
    \begin{tabular}{cc}
      \lstinline!short! & \lstinline!long!
    \end{tabular}
  \end{center}

  \vspace{1em}

  Manipulate value range
  \begin{center}
    \begin{tabular}{cc}
      \lstinline!signed! & \lstinline!unsigned!
    \end{tabular}
  \end{center}

\end{frame}

\begin{frame}[fragile]
  \frametitle{Type qualifiers}

  \footnotesize
  \vspace*{-1em}
  \begin{center}
    {\renewcommand{\arraystretch}{1.4}
    \begin{tabular}{|L{5cm}|L{3cm}|}\hline
      {\color{Tropiteal}Type} & {\color{Tropiteal}Size$^*$ \fontsize{8pt}{8pt}\selectfont (minimum)} \\ \hline
      \lstinline!short int! & 2 byte \\
      \lstinline!int! & 2 byte \\
      \lstinline!long int! & 4 byte \\
      \lstinline!long long int! & 8 byte \\ \hline
      \lstinline!float! & 4 byte \\
      \lstinline!double! & 8 byte \\
      \lstinline!long double! & 10 byte \\ \hline
    \end{tabular}
    }
  \end{center}

  \nointerlineskip
  \begin{tikzpicture}[overlay,remember picture]
    \node[anchor=south east] at (current page.south east) [scale=0.6, Tropiteal] {$^*$ depending on compiler and architecture};
  \end{tikzpicture}

\end{frame}

\begin{frame}[fragile]
  \frametitle{Type qualifiers}

  \vspace*{-.5em}
  \footnotesize
  \begin{center}
    {\renewcommand{\arraystretch}{1.4}
    \begin{tabular}{|l|L{7.5cm}|}\hline
      {\color{Tropiteal}Type} & {\color{Tropiteal}Value range} \\ \hline
      \lstinline!int! & {\ttfamily -32,768 to 32,767} \\
      \lstinline!long int! & {\ttfamily -2,147,483,648 to 2,147,483,647} \\
      \lstinline!unsigned int! & {\ttfamily 0 to 65,535}  \\ \hline
      \lstinline!float! & {\ttfamily ± 1.175,494,3 10\textsuperscript{-38} to\newline
        ± 3.402,823,4 10\textsuperscript{38} }\\
      \lstinline!double! & {\ttfamily ± 2.225,073,858,507,201,4 10\textsuperscript{-308} to\newline
        ± 1.797,693,134,862,315,7 10\textsuperscript{308} }\\ \hline
    \end{tabular}
    }
  \end{center}

\end{frame}

\begin{frame}[fragile]
  \frametitle{Literals}

  Explicit values whose type are syntax dependent

  \vspace{.5em}

  \begin{itemize}
    \setlength\itemsep{.5em}
    \item Integers are just numbers: \hspace{.25em} \lstinline!5!
    \item Floats have a decimal point: \hspace{.25em} \lstinline!4.5!
    \item Characters are surrounded by ' ': \hspace{.25em} \lstinline!'c'!
    \item Booleans are either \lstinline!true! \hspace{.1em} or \hspace{.1em} \lstinline!false!
    \item C strings are surrounded by " ": \hspace{.25em} \lstinline!"Hello"!
    \item Function literals: \lstinline![](int){/*...*/};!\tikzmark{function lit}
  \end{itemize}

  \nointerlineskip
  \begin{tikzpicture}[overlay,remember picture]
    \node[anchor=west] at ([yshift=.4ex] function lit) [scale=0.8,Tropiteal] {\{C++11\}};
  \end{tikzpicture}
 
\end{frame}

\begin{frame}[fragile]
  \frametitle{Literals}

  One can add qualifiers to literals as well

  \footnotesize
  \begin{center}
    {\renewcommand{\arraystretch}{1.4}
    \begin{tabular}{|L{2cm}|L{4cm}|}\hline
      {\color{Tropiteal}Literal\strut{}} & {\color{Tropiteal}Type} \\ \hline
      \lstinline!42u! & \lstinline!unsigned int! \\
      \lstinline!167l! & \lstinline!long! \\
      \lstinline!5.62! & \lstinline!double! \\
      \lstinline!1.0e-2! & \lstinline!double! \\
      \lstinline!4.12f! & \lstinline!float! \\ \hline
    \end{tabular}
    }
  \end{center}
  
\end{frame}

\begin{frame}
  \frametitle{Operators}

  Operators in programming are much the same as operators in mathematics %Style

  \vspace{.5em}

  \begin{itemize}
    \setlength\itemsep{.5em}
    \item Arithmetic operators: \hspace{.25em} {\ttfamily\bfseries\color{Tropiteal} + - * / \%}
    \item Logic operators: \hspace{.25em} {\ttfamily\bfseries\color{Tropiteal} and \&\& or || !}
    \item Comparison operators: \hspace{.25em} {\ttfamily\bfseries\color{Tropiteal} == < != > <= >=}
    \item Combined operators:  \hspace{.25em} {\ttfamily\bfseries\color{Tropiteal} += -= *= /=}
    \item Others: \hspace{.25em} {\ttfamily\bfseries\color{Tropiteal} << >> = ++ -- ? \& :: ->}
  \end{itemize}
  
\end{frame}

\begin{frame}[fragile]
  \frametitle{Operators}

  Operators have different precedence levels

  \vspace{1em}
  \begin{itemize}
    \setlength\itemsep{1em}
    \item {\ttfamily\bfseries 5+12*7/4-2} \hspace{.25em}  
      \only<2->{\tikz[baseline=-4pt] \draw[<-,>=stealth,line width=1pt] (0,0) -- (1,0); \hspace{.25em} {\color{Tropiteal} 24}}
    \item {\ttfamily\bfseries 4>3 and 7==8 or 16<=72} \hspace{.25em}
      \only<3->{\tikz[baseline=-4pt] \draw[<-,>=stealth,line width=1pt] (0,0) -- (1,0); \hspace{.25em} {\color{Tropiteal} true}}
  \end{itemize}

  \vspace{1em}
  \uncover<3->{
    Whitespace does nothing in C++
  }
  
\end{frame}

\begin{frame}[fragile]
  \frametitle{Operators}

  Expressions can be grouped with \, {\color{Tropiteal}\large (\;)}

  \vspace{1em}
  \begin{itemize}
    \setlength\itemsep{1em}
    \item {\ttfamily\bfseries {\color{Tropiteal}(5+12)}*7/4-2}
      \only<2->{\tikz[baseline=-4pt] \draw[<-,>=stealth,line width=1pt] (0,0) -- (1,0); \hspace{.25em} {\color{Tropiteal} 27}}
    \item {\ttfamily\bfseries {\color{Tropiteal}(5+12)}*7/{\color{Marty}(4-2)}}
      \only<3->{\tikz[baseline=-4pt] \draw[<-,>=stealth,line width=1pt] (0,0) -- (1,0); \hspace{.25em} {\color{Tropiteal} 59}}
  \end{itemize}

  
\end{frame}

\begin{frame}[fragile]
  \frametitle{Type casting}

  C++ is {\color{FeebleWeek}statically} typed, but not {\color{Marty}strongly} typed

  \vspace{1em}

  One can change between the types using casting

  \vspace{1em}

  \begin{lstlisting}[escapeinside={(*}{*)}]
int n = 5;
double cn = (double)n;(*\tikzmark{c cast}*)
double cppn = static_cast<double>(n);(*\tikzmark{cpp cast}*)
  \end{lstlisting}

  \vspace{1em}

  But you can never change the type of the variable \lstinline!n!

  \nointerlineskip
  \begin{tikzpicture}[overlay,remember picture,every path/.style={line width=1pt}]
    \coordinate (c text begin) at ($(c cast) + (.5em,.4ex)$);
    \coordinate (c text end) at ($(c text begin) + (4cm,0)$);
    \draw[<-,>=stealth] (c text begin) -- (c text end) node[right,scale=0.6] {C style cast};

    \coordinate (cpp text begin) at ($(cpp cast) + (.5em,.4ex)$);
    \coordinate (cpp text end) at (cpp text begin -| c text end);
    \draw[<-,>=stealth] (cpp text begin) -- (cpp text end) node[right,scale=0.6] {C++ style cast};
  \end{tikzpicture}
  
\end{frame}

\begin{frame}[fragile]
  \frametitle{Variable qualifiers: {\color{FeebleWeek}const} }
  
  Constants are declared with \lstinline!const!

  \vspace{1em}

  \begin{lstlisting}[escapeinside={(*}{*)},basicstyle=\ttfamily\normalsize]
const int size_of_arrays = 100;

/* ... */

size_of_arrays *= 2;(*\tikzmark{error}*)
  \end{lstlisting}

  \nointerlineskip
  \begin{tikzpicture}[overlay,remember picture,every path/.style={line width=1pt}]
    \coordinate (error begin) at ($(error) + (.5em,.4ex)$);
    \coordinate (error end) at ($(error begin) + (2cm,0)$);
    \draw[<-,>=stealth] (error begin) -- (error end) node[right,scale=0.8,Marty] {Compile error};
  \end{tikzpicture}
  
\end{frame}

\begin{frame}[fragile]
  \frametitle{Variable qualifiers: {\color{FeebleWeek}const} }

  \begin{center}
    \begin{tikzpicture}
      \node (mutable) [font=\ttfamily\bfseries,FeebleWeek,scale=1.5] {mutable};
      \node[right=2.5cm of mutable] (const) [font=\ttfamily\bfseries,FeebleWeek,scale=1.5] {const};

      \draw[line width=4pt,-{Latex[length=.5cm]},Marty] ([shift={(.70,.175)}] mutable.east) -- +(1.4cm,0);
      \draw[line width=4pt,{Latex[length=.5cm]}-,Marty] ([shift={(.40,-.175)}] mutable.east) -- +(1.4cm,0);
    \end{tikzpicture}
  \end{center}

  \only<2>{
    \vspace{1em}
    In {\color{Tropiteal}C++} variables are mutable by default

    \vspace{1em}
    Different from languages like {\color{Marty}rust}
  }
  
\end{frame}

\begin{frame}[fragile]
  \frametitle{Pointers}

  \only<1>{
  Variables are still just named locations in memory
  \\[.5em]
  We can work with these locations using pointers
  }

  \begin{onlyenv}<2>
  \vspace*{-2em}
  \begin{center}
    \begin{tikzpicture}[remember picture]
      \matrix[table,anchor=east,row 1/.style={nodes={draw=none,scale=1},color=Tropiteal}] (memory)
      {
         Value & Address \\
         & \\
         256 & 0x78d2 \\
         & \\
         0x78d2 & 0xa041 \\
         & \\
      };
      \node[right=.2cm of memory.east,rotate=270,xshift=-1.25cm] {Memory};
    \end{tikzpicture}
  \end{center}

  \vspace*{-.5em}
  \begin{lstlisting}[escapeinside={(*}{*)},basicstyle=\normalsize\ttfamily]
int val(*\tikzmark{val}*) = 256;
int * ptr(*\tikzmark{ptr}*) = &val;
  \end{lstlisting}

  \nointerlineskip
  \begin{tikzpicture}[overlay,remember picture]
    \draw[line width=1pt,->,>=stealth] 
    ([shift={(-1.75ex,1.25ex)}] val) .. controls +(0,2cm) and +(-2cm,0) .. ([xshift=-.25em] memory-3-1.west);
    \draw[line width=1pt,->,>=stealth] 
    ([shift={(-1.75ex,-.25ex)}] ptr) .. controls +(0,-3cm) and +(7cm,0) .. ([xshift=.25em] memory-5-2.east);
  \end{tikzpicture}

  \end{onlyenv}

  \begin{onlyenv}<3>
  \vspace*{-2em}
  \begin{center}
    \begin{tikzpicture}[remember picture]
      \matrix[table,anchor=east,row 1/.style={nodes={draw=none,scale=1},color=Tropiteal}] (memory2)
      {
         Value & Address \\
         & \\
         256 & 0x78d2 \\
         ? & 0x78d3 \\
         0x78d2 & 0xa041 \\
         0x78d3 & 0xb6d2 \\
      };
      \node[right=.2cm of memory2.east,rotate=270,xshift=-1.25cm] {Memory};
    \end{tikzpicture}
  \end{center}

  \vspace*{-.5em}
  \begin{lstlisting}[escapeinside={(*}{*)},basicstyle=\normalsize\ttfamily]
int * ptr(*\tikzmark{ptr2}*) = &val;
int * ptrpp(*\tikzmark{ptrpp}*) = ptr + 1;
  \end{lstlisting}

  \nointerlineskip
  \begin{tikzpicture}[overlay,remember picture]
    \draw[line width=1pt,->,>=stealth] 
    ([shift={(-1.75ex,1.25ex)}] ptr2) .. controls +(0,2cm) and +(-2cm,0) .. ([xshift=-.25em] memory2-5-1.west);
    \draw[line width=1pt,->,>=stealth] 
    ([shift={(-2.75ex,-.25ex)}] ptrpp) .. controls +(0,-3cm) and +(7cm,0) .. ([xshift=.25em] memory2-6-2.east);
  \end{tikzpicture}

  \end{onlyenv}
 
\end{frame}

\begin{frame}[fragile]
  \frametitle{Pointers}

  You can access the value of the memory the pointer points to by {\color{FeebleWeek}dereferencing} it

  \vspace{2cm}

  \begin{lstlisting}[escapeinside={(*}{*)},basicstyle=\ttfamily]
int * ptr = &val;
int half_val = *ptr / 2;
  \end{lstlisting}

  \vspace{.5em}

  This is why the type of the pointer is important, so the program knows how many bits to read.

  \nointerlineskip
  \begin{tikzpicture}[overlay,remember picture]
    \node[FeebleWeek] (operator) at (current page.center) [scale=2] {*};
    \draw[line width=1pt,<-,>=stealth] ([yshift=.15cm] operator.east) -- +(2cm,0) 
      node[right,align=center,scale=.8] {Dereference\\operator};
  \end{tikzpicture}
  
\end{frame}

\begin{frame}[fragile]
  \frametitle{The null pointer}

  Sometimes it is useful to be able to say that the pointer doesn't point to anything

  \vspace{.5cm}
  \begin{lstlisting}[escapeinside={(*}{*)},basicstyle=\ttfamily]
double * ptr = NULL; (*\tikzmark{null}*)
double * ptr = nullptr; (*\tikzmark{nullptr}*)
  \end{lstlisting}

  \nointerlineskip
  \begin{tikzpicture}[overlay,remember picture]
    \draw[pointy arrow] ([yshift=.5ex] null) -- +(1.5cm,0)
      coordinate (layer1)
      node[right,scale=0.75] {Old style};
    \draw[transform canvas={yshift=.5ex},pointy arrow] (nullptr) -- (nullptr -| layer1)
      node[right,scale=0.75] {New style {\color{Tropiteal}\{C++11\}}};
  \end{tikzpicture}
\end{frame}

\begin{frame}[fragile]
  \frametitle{References}

  A reference works as an alias for a variable

  \vspace{1em}
  \begin{lstlisting}[escapeinside={(*}{*)},basicstyle=\ttfamily]
int value = 134;
int & ref  = value;
int value_half = ref / 2;
&value == &ref;(*\tikzmark{equal check}*)
  \end{lstlisting}

  \vspace{1em}
  In many languages assignment automatically creates references and not copies, e.g. {\color{Tropiteal}JavaScript}

  \nointerlineskip
  \begin{tikzpicture}[overlay,remember picture]
    \draw[line width=1pt,<-,>=stealth] ([shift={(.5em, .4ex)}] equal check) -- +(3cm,0)
      node[right,FeebleWeek] {true};
  \end{tikzpicture}
  
\end{frame}

\begin{frame}[fragile]
  \frametitle{What's the deal with {\color{Marty}*} and {\color{Marty}\&} ?}

  In type specifications:

  \begin{itemize}
    \setlength\itemsep{0.5em}
    \item {\color{Marty}*} \:declares a pointer
    \item {\color{Marty}\&} \:declares a reference
  \end{itemize}

  \vspace{0.5em}
  As operators

  \begin{itemize}
    \setlength\itemsep{0.5em}
    \item {\color{Marty}*} \:converts pointer to reference
    \item {\color{Marty}\&} \:converts reference to pointer {\fontsize{8pt}{8pt}\selectfont \{sort of\}}
  \end{itemize}
  
\end{frame}

\begin{frame}[fragile]
  \frametitle{The {\color{FeebleWeek}auto} type}

  When assigning

  \begin{center}
    \begin{tikzpicture}
      \node[draw,line width=1pt,FeebleWeek] (var) {type of variable};
      \node[right= .1cm of var] (eq) {=};
      \node[draw,line width=1pt,Tropiteal] [right=.1cm of eq] (rhs) {type of expression};
      \node[right= .1cm of rhs] {;};
    \end{tikzpicture}
  \end{center}

  These types are generally the same

  \vspace{1em}

  {\ttfamily\bfseries\color{FeebleWeek}auto} \; = \; whatever the type on the right is

  \CPPEleven
\end{frame}

\begin{frame}[fragile]
  \frametitle{The {\color{FeebleWeek}auto} type}

  {\color{FeebleWeek}auto} only picks up the base type

  \vspace{1em}

  \begin{lstlisting}[escapeinside={(*}{*)},basicstyle=\ttfamily]
const char & char_ref = some_char;
(*\tikzmark{auto begin}*)auto(*\tikzmark{auto end}*) var = char_ref;
  \end{lstlisting}


  \vspace{2em}

  But you can use type qualifiers on it

  \nointerlineskip
  \begin{tikzpicture}[overlay,remember picture]
    \underlinemark[line width=1.5pt, Marty]{auto begin}{auto end}
    \coordinate (auto middle) at ($(auto begin) !.5! (auto end)$);
    \draw[<-,>=stealth,line width=1pt] ([yshift=-.25cm] auto middle) .. controls +(0,-1) and +(-1,0) .. +(2,-.75)
      node[right] [] {\lstinline!char!};
  \end{tikzpicture}

  \CPPEleven
\end{frame}

\begin{frame}[fragile]
  \frametitle{{\LARGE C} \hspace{.2em} Style Arrays}

  \begin{columns}
    \column{0.6\textwidth}
      One can create a list of objects like this:
    
      \vspace{1em}
    
      \begin{lstlisting}[escapeinside={(*}{*)},basicstyle=\ttfamily]
int array[10];
      \end{lstlisting}
    
      \vspace{1em}
    
      Places 10 integers consecutively in memory
    \column{0.4\textwidth}

    \vspace*{-1.5cm}
    \begin{center}
      \begin{tikzpicture}
        \matrix[table,anchor=east,row 1/.style={nodes={draw=none,scale=1},color=Tropiteal}]
        {
           Memory \\
           \lstinline!a[0]! \\
           \lstinline!a[1]! \\
           \lstinline!a[2]! \\
           \lstinline!a[3]! \\
           \lstinline!a[4]! \\
           \lstinline!a[5]! \\
           \lstinline!a[6]! \\
           \lstinline!a[7]! \\
           ... \\
        };
      \end{tikzpicture}
    \end{center}
    \end{columns}
\end{frame}

\begin{frame}[fragile]
  \frametitle{{\LARGE C} \hspace{.2em} Style Arrays}

  As a type qualifier {\large\color{Tropiteal} []} decides how much memory should be reserved

  \begin{lstlisting}[escapeinside={(*}{*)},basicstyle=\ttfamily]
int fibonacci_numbers[10];
  \end{lstlisting}

  \vspace{1em}

  As an operator {\large\color{Tropiteal} []} is used to access the various memory positions

  \begin{lstlisting}[escapeinside={(*}{*)},basicstyle=\ttfamily]
fibonacci_numbers[5] = 8;
fibonacci_numbers[6] = 13;
  \end{lstlisting}

\end{frame}

\begin{frame}[fragile]
  \frametitle{{\LARGE C} \hspace{.2em} Style Arrays}

  The array is actually just a pointer in disguise

  \begin{center}
    \begin{lstlisting}[escapeinside={(*}{*)},basicstyle=\ttfamily]
   array[0] == *array
    \end{lstlisting}
  \end{center}

  The {\large\color{Tropiteal} []} operator is also just a shorthand

  \begin{center}
    \begin{lstlisting}[escapeinside={(*}{*)},basicstyle=\ttfamily]
   array[n] == *(array + n)
    \end{lstlisting}
  \end{center}
  
\end{frame}

\begin{frame}[fragile]
  \frametitle{{\LARGE C} \hspace{.2em} Style Arrays}

  One can initialise the array with an initialiser list

  \vspace{.5cm}
  \begin{lstlisting}[escapeinside={(*}{*)},basicstyle=\ttfamily]
int lucky[](*\tikzmark{size}*) = {4, 12, 42, 7};
  \end{lstlisting}

  \nointerlineskip
  \begin{tikzpicture}[overlay,remember picture]
    \draw[pointy arrow] ([shift={(-1ex,-.75ex)}] size) -- +(0,-1cm)
      node[below,xshift=2cm,scale=0.75] {Size of the array inferred from context};
  \end{tikzpicture}
  
\end{frame}

\begin{frame}[fragile]
  \frametitle{{\LARGE C} \hspace{.2em} Style Strings}

  A C style string is simply an array of characters

  \vspace{.5cm}
  \begin{lstlisting}[escapeinside={(*}{*)}]
char message[] = "How are you all doing?";
  \end{lstlisting}

  \vspace{.5cm}
  The final entry is always the null character \lstinline!'\0'!

  \vspace{.5cm}
  \footnotesize
  ({\color{Tropiteal}message} has 23 elements)
  
\end{frame}

\section{Control Structures}

\frame[plain]{\sectionpage}

\begin{frame}[fragile]
  \frametitle{Controlling program flow}

  So far we have learned to write programs that execute "in a straight line"

  \vspace{.5em}
  \begin{itemize}
    \setlength\itemsep{0.5em}
    \item<2-> What if we want to {\color{Marty}branch}?
    \item<3-> What if we want to {\color{FeebleWeek}repeat}?
  \end{itemize}

  \vspace{1em}
  \uncover<4->{
    Then we need control structures
  }

  \nointerlineskip
  \begin{tikzpicture}[overlay,remember picture]
    \coordinate (begin) at ([xshift=-1.5cm] current page.east);
    \only<1> {
      \draw[line width=3pt,->,>=stealth,Tropiteal] ([yshift=-1cm] begin) -- +(0,-1.5cm);
    }

    \only<2> {
      \draw[line width=3pt,->,>=stealth,Marty] ([yshift=-1cm] begin) -- ++(0,-.5cm)
        .. controls +(0,-.3cm) and +(0,.3cm) .. ++(+.3cm,-.5cm) -- ++(0,-.5cm);
      \draw[line width=6pt,->,>=stealth,WhiteTrash] ([yshift=-1cm] begin) -- ++(0,-.5cm)
        .. controls +(0,-.3cm) and +(0,.3cm) .. ++(-.3cm,-.5cm) -- ++(0,-.5cm);
      \draw[line width=3pt,->,>=stealth,Marty] ([yshift=-1cm] begin) -- ++(0,-.5cm)
        .. controls +(0,-.3cm) and +(0,.3cm) .. ++(-.3cm,-.5cm) -- ++(0,-.5cm);
    }

    \only<3-> {
      \draw[line width=3pt,-{Stealth[bend,length=10pt]},shorten >=.2cm,FeebleWeek] ([yshift=-1cm] begin) -- ++(0,-.5cm)
        .. controls +(0,-.3cm) and +(0,.3cm) .. ++(-.3cm,-.5cm) -- ++(0,-.5cm)
        .. controls +(0,-.2cm) and +(0,-.2cm) .. ++(.6cm,0) -- ++(0,.5cm)
        .. controls +(0,.3cm) and +(0,-.3cm) .. ++(-.3cm,.5cm);
    }
  \end{tikzpicture}
 
\end{frame}

\begin{frame}[fragile]
  \frametitle{Conditionals: {\color{FeebleWeek}if}, {\color{FeebleWeek}else} }
  
  \begin{lstlisting}[escapeinside={(*}{*)},basicstyle=\ttfamily\large]
  if ((*\tikzmark{if condition begin}*)       (*\tikzmark{if condition end}*)) {
  (*\tikzmark{true block}*)

  } else {
  (*\tikzmark{false block}*)

  }
  \end{lstlisting}

  \nointerlineskip
  \begin{tikzpicture}[overlay,remember picture]
    \node[anchor=south west] at ([shift={(.1cm,-.25cm)}] if condition begin) {condition};

    \draw [-{Stealth[bend]}, line width=1pt,FeebleWeek]
      ([shift={(1cm,-.2cm)}] if condition begin) .. controls +(0,-1cm) and +(1cm,0) .. ([yshift=-.25cm] true block)
      node [midway,sloped,yshift=.05cm,fill=WhiteTrash,inner sep=2pt] {true};

    \draw [-{Stealth[bend]}, line width=1pt,Marty]
      ([shift={(2cm,-.2cm)}] if condition begin) .. controls +(0,-1cm) and +(0,1cm) .. ++(.25cm,-2cm)
      .. controls +(0,-2cm) and +(1cm,0) .. ([yshift=-.25cm] false block)
      node [midway,sloped,yshift=.05cm,fill=WhiteTrash,inner sep=2pt] {false};
  \end{tikzpicture}
  
\end{frame}

\begin{frame}[fragile]
  \frametitle{Conditionals: {\color{FeebleWeek}if}, {\color{FeebleWeek}else} }

  Can also do multiple statements

  \vspace{.5em}
  \begin{lstlisting}[escapeinside={(*}{*)},basicstyle=\ttfamily]
  if ((*\tikzmark{if first cond}*)        ) {

  } else if ((*\tikzmark{if second cond}*)        ) {

  } else {

  }
  \end{lstlisting}

  \vspace{.5em}
  It chooses the {\color{Marty}first} condition that matches

  \nointerlineskip
  \begin{tikzpicture}[overlay,remember picture]
    \node[anchor=south west] at ([shift={(.1cm,-.25cm)}] if first cond) {condition};
    \node[anchor=south west] at ([shift={(.1cm,-.25cm)}] if second cond) {condition};
  \end{tikzpicture}
  
\end{frame}

\begin{frame}[fragile]
  \frametitle{Conditionals: {\color{FeebleWeek}switch}}

  \begin{lstlisting}[escapeinside={(*}{*)},basicstyle=\ttfamily\large]
  switch ((*\tikzmark{switch var}*)      ) {
    case (*\tikzmark{case var1}*)    :
    (*\tikzmark{case block1}*)
    case (*\tikzmark{case var2}*)    :
    (*\tikzmark{case block2}*)
    default:
    (*\tikzmark{case default}*)
  }
  \end{lstlisting}

  \nointerlineskip
  \begin{tikzpicture}[overlay,remember picture]
    \node[anchor=south west] at ([shift={(.1cm,-.25cm)}] switch var) (var) {variable};
    \node[anchor=south west] at ([shift={(-.35cm,-.25cm)}] case var1) {value \#1};
    \node[anchor=south west] at ([shift={(-.35cm,-.25cm)}] case var2) {value \#2};

    \draw [-{Stealth[bend]}, line width=1pt,FeebleWeek]
      ([xshift=-.4cm] var.south east) .. controls +(1cm,-1cm) and +(1cm,0) .. ([yshift=.1cm] case block1)
      node [pos=0.7,sloped,yshift=.05cm,fill=WhiteTrash,inner sep=2pt,scale=0.8] {case 1};

    \draw [-{Stealth[bend]}, line width=1pt,Tropiteal]
      ([xshift=-.2cm] var.south east) .. controls +(2.5cm,-2cm) and +(3cm,0) .. ([yshift=.1cm] case block2)
      node [pos=0.8,sloped,yshift=.05cm,fill=WhiteTrash,inner sep=2pt,scale=0.8] {case 2};

    \draw [-{Stealth[bend]}, line width=1pt,Marty]
      (var.south east) .. controls +(3.5cm,-2.166cm) and + (5cm,0) .. ([yshift=.1cm] case default)
      node [pos=0.85,sloped,yshift=.05cm,fill=WhiteTrash,inner sep=2pt,scale=0.8] {no case};
  \end{tikzpicture}
  
\end{frame}

\begin{frame}[fragile]
  \frametitle{Loops: {\color{FeebleWeek}for}}

  \begin{lstlisting}[escapeinside={(*}{*)},basicstyle=\ttfamily\large]
for ((*\tikzmark{for init}*)      ;(*\tikzmark{for test}*)      ;(*\tikzmark{for update}*)      ) {

(*\tikzmark{for block}*)


}
  \end{lstlisting}

  \nointerlineskip
  \begin{tikzpicture}[overlay,remember picture]
    \node[anchor=south west] at ([shift={(0,-.25cm)}] for init) (init) {initialise};
    \node[anchor=south west] at ([shift={(-.1cm,-.25cm)}] for test) (check) {condition};
    \node[anchor=south west] at ([shift={(.2cm,-.38cm)}] for update) (update) {update};

    \draw[-{Stealth[bend]},line width=1pt,Tropiteal]
      (init.north) .. controls +(0,1cm) and +(0,1cm) .. ([xshift=-.2cm] check.north);
    \draw[-{Stealth[bend]},line width=1pt,Tropiteal]
      (update.north) .. controls +(0,1cm) and +(0,1cm) .. ([xshift=.2cm] check.north);

    \draw [-{Stealth[bend]}, line width=1pt,FeebleWeek]
      ([xshift=-.2cm] check.south) .. controls +(0,-2cm) and +(1cm,0) .. ([yshift=-.25cm] for block)
      node [midway,sloped,yshift=.05cm,fill=WhiteTrash,inner sep=2pt] {true};

    \draw [-{Stealth[bend]}, line width=1pt,Marty]
      ([xshift=.2cm] check.south) .. controls +(0,-3cm) and +(1cm,0) .. ([yshift=-3cm] for block)
      node [midway,sloped,yshift=.05cm,fill=WhiteTrash,inner sep=2pt] {false};

    \draw[line width=3pt,WhiteTrash]
      ([yshift=-1.25cm] for block) .. controls +(3cm,0) and +(0,-2cm) .. (update.south);

    \draw[-{Stealth[bend]},line width=1pt,Tropiteal]
      ([yshift=-1.25cm] for block) .. controls +(3cm,0) and +(0,-2cm) .. (update.south);

  \end{tikzpicture}
  
\end{frame}

\begin{frame}[fragile]
  \frametitle{Loops: {\color{FeebleWeek}for} - example}

  \begin{lstlisting}[basicstyle=\ttfamily]
int numbers[100];

for (auto i = 0; i < 100; ++i) {
  numbers[i] = i;
}
  \end{lstlisting}

\end{frame}

\begin{frame}[fragile]
  \frametitle{Loops: {\color{FeebleWeek}while}}

  \begin{lstlisting}[escapeinside={(*}{*)},basicstyle=\ttfamily\large]
  while ((*\tikzmark{while check}*)      ) {
  
  (*\tikzmark{while block}*)


  }
  \end{lstlisting}
  

  \nointerlineskip
  \begin{tikzpicture}[overlay,remember picture]
    \node[anchor=south west] at ([shift={(-.1cm,-.25cm)}] while check) (check) {condition};

    \draw [-{Stealth[bend]}, line width=1pt,FeebleWeek]
      ([xshift=-.2cm] check.south) .. controls +(0,-2cm) and +(1cm,0) .. ([yshift=-.25cm] while block)
      node [pos=0.6,sloped,yshift=.05cm,fill=WhiteTrash,inner sep=2pt] {true};

    \draw [-{Stealth[bend]}, line width=1pt,Marty]
      ([xshift=.2cm] check.south) .. controls +(0,-3cm) and +(1cm,0) .. ([yshift=-3cm] while block)
      node [pos=0.6,sloped,yshift=.05cm,fill=WhiteTrash,inner sep=2pt] {false};

    \draw[line width=3pt,WhiteTrash]
      ([yshift=-1.25cm] while block) .. controls +(3cm,0) and +(1cm,-2cm) .. ([xshift=1cm] check.south);

    \draw[-{Stealth[bend]},line width=1pt,Tropiteal]
      ([yshift=-1.25cm] while block) .. controls +(3cm,0) and +(1cm,-2cm) .. ([xshift=1cm] check.south);
  \end{tikzpicture}
\end{frame}

\begin{frame}[fragile]
  \frametitle{Loops: {\color{FeebleWeek}while} - example}

  \begin{lstlisting}[basicstyle=\ttfamily\fontsize{13pt}{13pt}\selectfont]
// Calculate 10!
unsigned factorial = 1;
unsigned counter = 10;

while (counter > 1) {
  factorial *= counter;
  --counter;
}

std::cout << "10! = " << factorial;
  \end{lstlisting}
  
\end{frame}

\begin{frame}[fragile]
  \frametitle{Exiting loops}

  There are two commands for altering loop flow
  
  \vspace{.5em}
  {\color{Marty}break} and continue

  \vspace{.5em}
  \begin{lstlisting}[escapeinside={(*}{*)},basicstyle=\ttfamily\large]
  while ((*\tikzmark{while check}*)      ) {

    break;(*\tikzmark{break tag}*)
  (*\tikzmark{while box}*)}
  \end{lstlisting}

  \vspace{.5em}
  {\color{Marty}break} exits the loop

  \nointerlineskip
  \begin{tikzpicture}[overlay,remember picture]
    \node[anchor=south west] at ([shift={(-.1cm,-.25cm)}] while check) (check) {condition};
    \draw[-{Stealth[bend]},line width=1pt,Marty] ([yshift=.5ex] break tag) .. controls +(1cm,0) and +(4cm,0) .. ([yshift=-.5cm] while box);
  \end{tikzpicture}
  
\end{frame}

\begin{frame}[fragile]
  \frametitle{Exiting loops}

  There are two commands for altering loop flow
  
  \vspace{.5em}
  break and {\color{FeebleWeek}continue}

  \vspace{.5em}
  \begin{lstlisting}[escapeinside={(*}{*)},basicstyle=\ttfamily\large]
  while ((*\tikzmark{while check}*)      ) {

    continue;(*\tikzmark{continue tag}*)
  (*\tikzmark{while box}*)}
  \end{lstlisting}

  \vspace{.5em}
  {\color{FeebleWeek}continue} jumps back to the loop update

  \nointerlineskip
  \begin{tikzpicture}[overlay,remember picture]
    \node[anchor=south west] at ([shift={(-.1cm,-.25cm)}] while check) (check) {condition};
    \draw[-{Stealth[bend]},line width=1pt,FeebleWeek]
    ([yshift=.5ex] continue tag) .. controls +(1cm,0) and +(.5cm,-1cm) ..  (check.south east);
  \end{tikzpicture}
  
\end{frame}

\section{Crash Introduction to IO}

\frame[plain]{\sectionpage}

\begin{frame}[fragile]
  \frametitle{Streams}

  In C++ IO is handled by something called streams

  \vspace{.5em}
  We use the shift operators to interact with the stream objects

  \vspace{1em}
  \only<1>{
    Original meaning
  }
  \only<2>{
    Meaning adopted by stream objects
  }

  \vspace{.5em}
  \begin{tikzpicture}
    \node[draw, line width=2pt, FeebleWeek] [inner sep=3pt] (left obj1) {object1};
    \node[right=0pt of left obj1] (left shift) [font=\ttfamily,Tropiteal,scale=1.5] {<<};
    \node[right=0pt of left shift] [draw, line width=2pt, Marty] [inner sep=3pt] (left obj2) {object2};

    \node[below=0.25em of left shift] (right shift) [font=\ttfamily,Tropiteal,scale=1.5] {>>};
    \node[left=0pt of right shift] [draw, line width=2pt, FeebleWeek] [inner sep=3pt] (right obj1) {object1};
    \node[right=0pt of right shift] [draw, line width=2pt, Marty] [inner sep=3pt] (right obj2) {object2};

    \only<1>{
    \draw[<-,>=stealth,line width=1pt] ([xshift=.4cm] left obj2.east) -- +(1cm,0)
      node[right,scale=0.8,anchor=west] {left shift operator};
    \draw[<-,>=stealth,line width=1pt] ([xshift=.4cm] right obj2.east) -- +(1cm,0)
      node[right,scale=0.8,anchor=west] {right shift operator};
    }

    \only<2>{
    \draw[<-,>=stealth,line width=1pt] ([xshift=.4cm] left obj2.east) -- +(1cm,0)
      node[right,scale=0.8,anchor=west,align=left] {object2 {\color{Tropiteal}writes to} object1};
    \draw[<-,>=stealth,line width=1pt] ([xshift=.4cm] right obj2.east) -- +(1cm,0)
      node[right,scale=0.8,anchor=west,align=left] {object2 {\color{Tropiteal}reads from} object1};
    }
    
  \end{tikzpicture}

\end{frame}

\begin{frame}[fragile]
  \frametitle{Standard in and out}

  The {\ttfamily\bfseries\color{Tropiteal}iostream} library includes two convenient stream objects

  \vspace{1em}
  {\large{\color{Tropiteal}std}::{\color{Tropiteal}cout}} for writing to console

  \vspace{.5em}
  {\large{\color{Tropiteal}std}::{\color{Tropiteal}cin}} for reading from keyboard

  \Library{iostream}

\end{frame}

\begin{frame}[fragile]
  \frametitle{Standard in and out - example}

  \begin{lstlisting}[escapeinside={(*}{*)}]
#include<iostream> (*\tikzmark{include}*)

int main()
{
  double input_from_user {0.};

  std::cin >> input_from_user;

  std::cout << "You wrote: \"" << input_from_user
    << "\"(*\tikzmark{char escape}*)" << (*\tikzmark{endl begin}*)std::endl(*\tikzmark{endl end}*);
}
  \end{lstlisting}

  \only<2>{
  \nointerlineskip
  \begin{tikzpicture}[overlay,remember picture]
    \draw[<-,>=stealth,line width=1pt] ([yshift=.35ex] include) -- +(1.5cm,0)
      node[right, align=left, scale=0.8] {%
        To gain access to\\
        {\color{Tropiteal}std}::{\color{Tropiteal}cout} and {\color{Tropiteal}std}::{\color{Tropiteal}cin}
      };

      \draw[<-,>=stealth,line width=1pt] ([shift={(-.6ex,-.5ex)}] char escape) -- +(0,-.5cm)
        node[below,scale=0.8] {Character escape};

      \coordinate (endl middle) at ($(endl begin) !.5! (endl end)$);
      \draw[<-,>=stealth,line width=1pt] ([yshift=-.25em] endl middle) .. controls +(0,-.5cm) and +(-1cm,0) .. +(2cm, -1cm)
        node[right,scale=0.8,align=left] {Flush stream and\\create newline};
  \end{tikzpicture}
  }

\end{frame}

\section{Coding Environments}

\frame[plain]{\sectionpage}

\LiveFrame

\section{Programming Practices}

\frame[plain]{\sectionpage}

\begin{frame}[fragile]
  \frametitle{Good Programming Practices}

  \begin{itemize}
    \setlength\itemsep{0.5em}
    \item {\color{Marty}Do not} use global variables {\fontsize{9pt}{9pt}\selectfont (constants are OK)}
    \item {\color{FeebleWeek}Always} initialise built in functions with a default value
    \item {\color{FeebleWeek}Always} use descriptive variable names \\
      {\fontsize{9pt}{9pt}\selectfont (and stay away from magic numbers)}
    \item Use {\color{FeebleWeek}const} consistently
  \end{itemize}
  
\end{frame}

\begin{frame}[fragile]
  \frametitle{Good Programming Practices}

  Stay away from C functionality that has been superseded

  \vspace{.5em}
  \begin{itemize}
    \setlength\itemsep{0.5em}
    \item Macros, especially {\ttfamily\color{sorange}\#define}
    \item Pointers to \lstinline!void!
    \item \lstinline!NULL! for empty pointers
    \item \lstinline!printf! and \lstinline!scanf! for IO
  \end{itemize}

\end{frame}

\section{Recap}

\frame[plain]{\sectionpage}

\begin{frame}[fragile]
  \frametitle{Recap Day 1}

  \begin{itemize}
    \setlength\itemsep{0.5em}
    \item A C++ program consists of {\color{Marty}sentences} and {\color{Marty}blocks}
    \item The type of the variable tells the compiler
    \vspace{.3em}
    \begin{itemize}
    \setlength\itemsep{0.5em}
      \item How much memory the variable needs
      \item What actions are allowed on the variable
    \end{itemize}
    \item There are four basic types in C++
    \vspace{.3em}
    \begin{itemize}
    \setlength\itemsep{0.5em}
      \item \lstinline!bool!, \lstinline!char!, \lstinline!int!, \lstinline!float!
    \end{itemize}
    \item Types can be modified with qualifiers
  \end{itemize}
  
\end{frame}

\begin{frame}[fragile]
  \frametitle{Recap Day 1}

  \begin{itemize}
    \setlength\itemsep{0.5em}
    \item Use pointers to examine memory locations
    \item References if you want to alias a variable
    \item Use {\color{FeebleWeek}if} and {\color{FeebleWeek}switch} to make branches in your code
    \item Use {\color{Marty}for} and {\color{Marty}while} to repeat stuff
  \end{itemize}
  
\end{frame}

\end{document}
